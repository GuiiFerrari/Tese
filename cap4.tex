%\documentclass[12pt]{report}
\documentclass[a4paper,12pt,oneside]{book}
%\usepackage[a4paper, margin=2cm, footskip=15pt]{geometry}
\usepackage[a4paper,top=3.5cm,left=3cm,right=3cm,bottom=2.5cm]{geometry}
\usepackage{times} %pacote da fonte Times New Roman
\usepackage[utf8]{inputenc}
\usepackage[export]{adjustbox}
\usepackage{caption}
\usepackage{subcaption}
\usepackage{setspace}
\usepackage{indentfirst}
\usepackage{float}
\usepackage{amsmath}
\usepackage{pgfplots}
% \usepackage{unicode-math}
\newcommand\mathplus{+}
% \usepackage{subfigure}
% \usepackage{algorithm}
% \usepackage[ruled,vlined,lined,linesnumbered,spanish, onelanguage]{algorithm2e}
\usepackage[ruled,vlined,lined,linesnumbered,portuguese, onelanguage]{algorithm2e}
% \usepackage{algorithmic}
\usepackage{algpseudocode}
\renewcommand{\baselinestretch}{1.3}
\usepackage[backend=bibtex, style=numeric, sorting=none]{biblatex}
\addbibresource{sample.bib}
%\renewcommand{\listalgorithm}{Lista de Algoritmos}%
%\renewcommand{\algorithm}{Algoritmo}%
%\renewcommand{\algocf@typo}{}%
%\renewcommand{\@algocf@procname}{Procedimento}
%\renewcommand{\@algocf@funcname}{Fun\c{c}\~{a}o}
\usepackage[scr=rsfs]{mathalpha}
\DeclareMathOperator{\sinc}{sinc}
\DeclareMathOperator{\rect}{rect}

\usepackage{hyperref}
\hypersetup{colorlinks=false, pdftitle={Tese}}
% \usepackage[backend=biber]{biblatex}
% \addbibresource{sample.bib}





\begin{document}
\pagestyle{plain}
\captionsetup[figure]{name={Figura}}
\renewcommand*\contentsname{Sumário}
\renewcommand{\listfigurename}{Lista de figuras}
\renewcommand{\listoftables}{Tabelas}
\renewcommand{\listofalgorithms}{Lista de algoritmos}
\makeatletter
\renewcommand{\@chapapp}{Capítulo}
\makeatother



\thispagestyle{empty}
\begin{center}

%%%%%%%%%%%%%%%%%%%%%%%%%%%%%%%%%%%%%%%%%%%%%%%%%%%%%%%%%%%%%%%%%%%%%%%%%%%%%%%%%%%%%    
%\Título da Tese \ Thesis title
	{\fontsize{16}{16} \selectfont Universidade de S\~ao Paulo \\}
	\vspace{0.1cm}
	{\fontsize{16}{16} \selectfont Instituto de F\'{i}sica}
    \vspace{2.2cm}

	{\fontsize{22}{22}\selectfont Estudo da reação de breakup $^4$He($^{17}$F,$^{16}$O+p)$^4$He usando o alvo ativo pAT-TPC: uma abordagem usando técnicas de Machine Learning\par}
    \vspace{2cm}

%%%%%%%%%%%%%%%%%%%%%%%%%%%%%%%%%%%%%%%%%%%%%%%%%%%%%%%%%%%%%%%%%%%%%%%%%%%%%%%%%%%%%    
%\Nome do Autor \ Author's name

    {\fontsize{18}{18}\selectfont Guilherme Ferrari Fortino\par}

    \vspace{1.4cm}

\end{center}

%%%%%%%%%%%%%%%%%%%%%%%%%%%%%%%%%%%%%%%%%%%%%%%%%%%%%%%%%%%%%%%%%%%%%%%%%%%%%%%%%%%%%%    
%\Orientador e coorientador (se existir) \ Supervisor and co-supervisor (if there is one)
\leftskip 6cm
\begin{flushright}	
\leftskip 6cm
Orientador: Prof. Dr.  Valdir Guimarães\underline{ \hskip 5cm  } 
\leftskip 6cm
    %Se não houve coorientador, comente a linha abaixo \ If there is no co-advisor, comment line below
Coorientador: Dr. Juan Carlos Zamora Cardona  \underline{ \hskip 5cm  } 
\end{flushright}	

\vspace{0.6cm}    

%%%%%%%%%%%%%%%%%%%%%%%%%%%%%%%%%%%%%%%%%%%%%%%%%%%%%%%%%%%%%%%%%%%%%%%%%%%%%%%%%%%%%    
% Grau Acadêmico \ Degree

\par
\leftskip 6cm
\noindent {Disserta\c{c}\~{a}o de mestrado apresentada ao Instituto de F\'{i}sica da Universidade de S\~{a}o Paulo, como requisito parcial para a obten\c{c}\~{a}o do t\'{i}tulo de Mestre(a) em Ci\^{e}ncias.}
\par
\leftskip 0cm
\vskip 2cm

%%%%%%%%%%%%%%%%%%%%%%%%%%%%%%%%%%%%%%%%%%%%%%%%%%%%%%%%%%%%%%%%%%%%%%%%%%%%%%%%%%%%    
% Banca Examinadora -- Primeiro nome é do presidente ou do presidente da banca \ Examining committee -- The first name must be the supervisor's name or the examination committee president's name.

\noindent Banca Examinadora: \\
\noindent Prof(a). Dr(a). Nome do(a) Professor(a) - Orientador (institui\c{c}\~{a}o de trabalho) \\
Prof(a). Dr(a). Nome do(a) Professor(a) (institui\c{c}\~{a}o de trabalho) \\
Prof(a). Dr(a). Nome do(a) Professor(a) (institui\c{c}\~{a}o de trabalho) \\
\vspace{1.8cm}


%%%%%%%%%%%%%%%%%%%%%%%%%%%%%%%%%%%%%%%%%%%%%%%%%%%%%%%%%%%%%%%%%%%%%%%%%%%%%%%%%%%%    
%Data \ Date
\begin{center}
    {S\~ao Paulo \\  2022}
\end{center}%\centering
    
\clearpage

\tableofcontents
\listoffigures
\listoftables
\listofalgorithms
\newpage

\chapter{Introdução}

\chapter{O experimento}\label{PATTPC}

\par Esse capítulo irá descrever sobre a parte experimental desse trabalho que foi desenvolvido na Universidade de Notre Dame em Outubro de 2019. Nesse experimento o feixe radioativo $^{17}$F foi produzido e selecionado por rigidez magnética pelo sistema TWINSOL (TWIN SOLenids)\cite{twinsol}. O alvo ativo \textit{prototype Active Target - Time Projection} Chamber (pAT-TPC) \cite{pattpc} foi usando como alvo e detector simultaneamente. Na continuação, será descrito como foi feito o sistema de produção do feixe secundário $^{17}$F e a detecção das reações induzidas por esse feixe no alvo ativo.

% \par Primeiro será descrita a produção do feixe de $^{17}$F com foco na descrição do sistema de produção de feixes radioativos (TWINSOL). Na seção seguinte será apresentado o detector pAT-TPC.


% O objetivo deste experimento é fazer medidas do \textit{breakup} do  gasoso, com o prototype Active Target - Time Projection Chamber (pAT-TPC). Esse capítulo irá descrever a produção do feixe e a 

% \par O prototype Active Target - Time Projection Chamber (pAT-TPC) é um detector de alvo-ativo que usa o volume de gás tanto quanto alvo quanto como detector. Seu grande volume ativo e capacidade de rastreio (\textit{tracking}) fornecem boa resolução energética e  angular, o que torna o pAT-TPC adequado para trabalhar com feixes exóticos de baixa intensidade\cite{pattpc, pattpc2}. O feixe secundário de $^{17}$F usado pelo pAT-TPC é produzido pelo \textit{Twin Solenoids} (TWINSOL) \cite{twinsol}. A figura \ref{fig:twinsol+pattpc} mostra os dois sistemas acoplados.

% \par Essa seção irá descrever brevemente o pAT-TPC, bem como o TWINSOL. Descrições mais detalhadas podem ser encontradas nas referências \cite{pattpc, twinsol}.

% \section{Produção do feixe de $^{17}$F}
\section{Produção do feixe secundário $^{17}$F usando o sistema TWINSOL}

\par A produção do feixe radioativo $^{17}$F foi feita a partir da reação do feixe estável de $^{16}$O com um alvo de deutério gasoso. O feixe primário $^{16}$O foi produzido e acelerado por um acelerador tipo Tandem, que possui uma tensão terminal de até 10 MV\cite{KOLATA1989503, NDtandem}. O feixe foi conduzido até a câmara alvo de produção, no início do sistema TWINSOL, onde partículas emergentes de reações nucleares surgiram. A figura \ref{fig:twinsol+pattpc} mostra o sistema TWINSOL acoplado com o pAT-TPC.

\begin{figure}[H]
    \centering
    \includegraphics[scale = 0.35]{figs/poster3.jpeg}
    \caption{Sistema TWINSOL à esquerda e pAT-TPC à direita. O feixe estável de $^{16}$O entra à esquerda do TWINSOL, para então ser produzido o feixe secundário $^{17}$F que irá ser conduzido até o alvo ativo pAT-TPC. Todo o sistema está localizado na University of Notre Dame.}
    \label{fig:twinsol+pattpc}
\end{figure}

\par O TWINSOL é um sistema de produção de feixes radioativos em voo que possui dois solenoides supercondutores alinhados que são usados para produzir, coletar, transportar, focar e analisar feixes estáveis e radioativos. O sistema se baseia na seleção de partículas a partir da sua rigidez magnética ($B\rho$)\cite{twinsol, ribras_leo, zamora_mater}. Cada solenoide possui 30 cm de raio interno e 1 m de comprimento\cite{twinsol}. O fato de ser um solenoide finito faz com que surjam efeitos de borda na componente radial do campo magnético do solenoide, cujo efeito faz com que o solenoide seja capaz de focalizar partículas. Para entender melhor o efeito de borda no campo magnético, e consequentemente o funcionamento do TWINSOL, simulações computacionais usando a biblioteca GEANT4\cite{geant4} foram feitas usando a geometria do sistema ``irmão" do TWINSOL, o Radioactive Ion Beams in Brasil (RIBRAS), que também possui dois solenoides supercondutores alinhados\cite{ribras_leo, ribras}. A simulação da figura \ref{fig:sim_ribras} mostra os dois solenoides de cor verde focalizando as partículas de cor laranja em um ponto do plano focal em vermelho. O campo magnético em função da posição do eixo, de cada solenoide, está na figura \ref{fig:campo_mag}.


% A figura  mostra um exemplo de calculo do valor do campo magnético, usando valores do sistema , para as componentes $x$, $y$ e $z$. É nítido que o campo em $z$ permanece praticamente constante e próximo das extremidades da bobina (linha vertical tracejada preta), $B_x$ e $B_y$ crescem em módulo, fazendo com que o solenoide funcione como uma lente delgada.

\begin{figure}[H]
    \centering
    \includegraphics[scale = 0.95]{figs/Foco2S.png}
    \caption{Simulação computacional do sistema RIBRAS, onde as partículas carregas em laranja surgem do ponto à direita daq figura e passam pelos dois solenoides em verde, para então serem focalizada em um ponto do plano focal em vermelho. A parte em preto dos solenoides corresponde aos limites físicos da bobina\cite{ribras_leo}.}
    \label{fig:sim_ribras}
\end{figure}

\begin{figure}[H]
    \centering
    \includegraphics[scale = 0.85]{figs/Campo_mag.png}
    \caption{Valor do campo magnético $B$ em tesla em função da posição em $z$ em centímetro da bobina. A linha vertical tracejada preta indica o limite físico da bobina. O campo foi calculado à uma distância de 8 cm do eixo do solenoide. É possível ver claramente o efeito de borda que há em um solenoide finito\cite{magnetic_field}.}
    \label{fig:campo_mag}
\end{figure}

\par A trajetória das partículas dentro do solenoide é helicoidal e, como os solenoides funcionam como uma lente óptica, existe um ponto focal. Em uma aproximação de um solenoide como uma lente grossa, o foco depende da rigidez magnética da partícula através da relação\cite{KOLATA1989503, zamora_mater}:

\begin{equation}
    \frac{1}{f} = \frac{B_z ^2}{(B\rho)^2},
\end{equation}

onde $f$ é o ponto focal, $B_z$ a componente $z$ do campo magnético, e $B\rho$ é dado por:

\begin{equation}
    B\rho = \frac{mv}{q} = \frac{\sqrt{2mE}}{q},
\end{equation}

onde $E$ é a energia, $m$ sua massa e $q$ seu estado de carga.
% A figura \ref{fig:ribras_sol_sim_a} mostra uma simulação feita com o GEANT4\cite{geant4}, usando novamente valores do RIBRAS, das trajetórias das partículas dentro do solenoide e em \ref{fig:ribras_sol_sim_b} pode-se ver que o feixe é focalizado em um ponto.

% \begin{figure}[H]
% \centering
%     \begin{subfigure}[t]{0.49\textwidth}
%         \centering
%         \includegraphics[scale=0.5]{figs/ribras_1.png}
%         \caption{Simulação que mostra as trajetórias helicoidais das partículas em laranja dentro do solenoide de cor verde até colidirem em um plano vermelho.}
%         \label{fig:ribras_sol_sim_a}
%     \end{subfigure}%
%     \hfill
%     \begin{subfigure}[t]{0.49\textwidth}
%         \centering
%         \includegraphics[scale=0.6]{figs/ribras_2.png}
%         \caption{Partículas em laranja passam pelo solenoide de cor verde que funciona como lente, sendo focalizadas em ponto da linha vermelha à esquerda da figura.}
%         \label{fig:ribras_sol_sim_b} 
%     \end{subfigure}
% \caption{Simulações computacionais usando o GEANT4 para entender o comportamento de partículas carregas que passam por um solenoide supercondutor.}
% \label{fig:sim_ribras}
% \end{figure}

% Figura a 1 sol, figura b mostra ponto focal

\par Para produzir $^{17}$F, uma célula gasosa localizada na câmara alvo de produção\cite{twinsol} preenchida com deutério foi bombardeada pelo feixe primário. Além de $^{17}$F, outras partículas também foram geradas, como o $^{17}$O, além de $^{16}$O do feixe espalhado.

% \par Não há só a reação que produz $^{17}$F, temos as reações: $^{16}$O(d, n)$^{17}$F, $^{16}$O(d, p)$^{17}$O e também o feixe de $^{16}$O pode ser apenas espalhado.

\par Mesmo para partículas diferentes, o $B_{\rho}$ pode ser muito próximo ou igual. Isso faz com que não seja possível obter um feixe de $^{17}$F com 100\% de pureza, e sim um coquetel de partículas\cite{zamora_mater}. O coquetel de partículas produzido possui 54\% de $^{17}$F, 41\% de $^{16}$O e cerca de 5\% de $^{17}$O. A figura \ref{fig:PID_17F} mostra o espectro biparamétrico de identificação de partículas, onde é possível identificar as partículas que estão presentes no feixe (coquetel de partículas). Por fim, o feixe  produzido pelo TWINSOL é conduzido até o pAT-TPC.

\begin{figure}[H]
    \centering
    \includegraphics[scale = 0.12]{figs/pid_17F.png}
    \caption{Espectro biparamétrico $\Delta E$ - $E$ de identificação de partículas. Nele é possível identificar que o feixe possui a presença de $^{17}$F, $^{16}$O e uma pequena parte de $^{17}$O.}
    \label{fig:PID_17F}
\end{figure}

\section{O alvo ativo pAT-TPC}

\par Essa seção irá descrever brevemente propriedades geométricas e físicas relacionadas ao sistema de detecção usado neste experimento, o alvo ativo pAT-TPC.
%  O pAT-TPC é um detector preenchido com gás que é usado como alvo-ativo.
\par A figura \ref{subfig:pattpc} mostra o desenho esquemático do pAT-TPC. O detector possui uma cela cilíndrica de 50 cm de comprimento e 28 cm de diâmetro, onde o seu eixo é alinhado com o eixo do feixe\cite{pattpc}, que entra pelo duto central. A câmara é preenchida com o $^4$He gasoso puro à uma pressão de 350 Torr que serve tanto para alvo de reações nucleares, quanto para a própria medição e detecção dos produtos da reação\cite{pattpc, pattpc2}. Tanto o feixe quanto partículas originadas da reação ionizam o gás e os elétrons que surgem dessa ionização foram conduzidos por um campo elétrico de 1 kV/cm perpendicular ao eixo da câmara até o plano detector (\textit{pad plane}), o \textit{Micromegas}\cite{micromegas}, mostrado na figura \ref{subfig:micromegas}.

\begin{figure}[H]
\centering
    \begin{subfigure}[t]{0.49\textwidth}
        \centering
        \includegraphics[scale=0.30]{figs/pattpc.png}
        \caption{Visão transversal do pAT-TPC. O gás é preenchido dentro da cela que possui um campo elétrico perpendicular ao plano do \textit{Micromegas}, à direita da figura. O feixe incide na câmara entrando pelo duto de feixe à esquerda da figura.}
        \label{subfig:pattpc}
    \end{subfigure}%
    \hfill
    \begin{subfigure}[t]{0.49\textwidth}
        \centering
        \includegraphics[scale=0.28]{figs/micromegas.png}
        \caption{Foto do \textit{Micromegas}. O detector é multipixelado com uma maior densidade no centro, parte destacada na imagem. O \textit{pad} central tem diâmetro de 5 mm enquanto que as faixas coaxiais possuem passo de 2mm\cite{attpc, josh_bradt}.}
        \label{subfig:micromegas} 
    \end{subfigure}
\caption{Figura esquemática do pAT-TPC e o detector \textit{Micromegas}\cite{pattpc}.}
\label{fig:pattpc_e_micromegas}
\end{figure}

\par O \textit{Micromegas} é um dispositivo de amplificação de elétrons, que consiste em um plano detector com 2048 canais (\textit{pads}) triangulares com eletrônica independente, que usa o \textit{Generic Electronics for TPCs} (GET)\cite{GET}. Detalhes sobre a eletrônica podem ser encontrados nas Refs. \cite{GET, josh_bradt}. O formato triangular dos canais tem como objetivo maximizar a resolução espacial do detector\cite{attpc}. Cada canal possui uma posição ($x$, $y$) fixa e a terceira coordenada $z$ será determinada a partir do tempo de deriva dos elétrons no gás\cite{pattpc, pattpc2, attpc, josh_bradt}. Isso só é possível pois a velocidade de deriva (\textit{drift}) dos elétrons é constante\cite{drift_constant}, portanto a posição em $z$ da partícula é diretamente proporcional ao tempo de voo. Esse princípio que deu origem ao nome de \textit{Time Projection Chamber}, pois o evento é projetado no tempo de deriva dos elétrons no gás. A equação  \ref{eq:langevin} (equação de Langevin) descreve o movimento de um elétron com massa $m$ e carga $e$ é descrito por\cite{drift_constant}

\begin{equation}\label{eq:langevin}
    m\frac{d\vec{v}}{dt} = e\left(\vec{E} +\vec{v}\times \vec{B}\right) - \frac{m}{\tau}\vec{v},
\end{equation}

onde $\vec{E}$ é o vetor campo elétrico, $\vec{B}$ o vetor campo magnético, $\vec{v}$ é o vetor de velocidade do elétron e $\tau$ é o tempo de colisão médio, que depende das propriedades termodinâmicas do gás. No caso deste experimento, $\vec{B}$ é zero e a solução estacionária para a velocidade de \textit{drift} do elétron é

\begin{equation}\label{eq:v_tpc}
    \vec{v} = \frac{\tau}{m}e\vec{E}.
\end{equation}

\par A velocidade de deriva depende das propriedades termodinâmicas do gás (temperatura, pressão) e também de sua condutividade elétrica\cite{drift_constant}. Isso significa que a calibração da velocidade envolve não depende só do campo elétrico, mas também das propriedades do gás dentro do alvo ativo\cite{pattpc, drift_constant}. Para acharmos a coordenada $z$, basta integrar a equação \ref{eq:v_tpc} para obter

\begin{equation}
    z = \frac{\tau}{m}e\vec{E}(t - t_0),
\end{equation}

onde no tempo $t_0$ = 0 o elétron está no plano do detector ($z$ = 0).

\par O \textit{Micromegas} possui \textit{thick gems}. \textit{Thick gems} usam do fato de que, no momento em que o elétron passa para uma região de campo elétrico ordens de grandeza maior que de sua origem, ocorre a ionização secundária (quando o elétron ioniza o gás). Isso provoca o que é chamado de avalanche de elétrons, amplificando a intensidade do sinal recebido\cite{GET}. A figura \ref{fig:thick_gems} mostra a esquematização do \textit{Micromegas}, onde na eletrônica de saída é produzido um sinal em função do tempo.

\begin{figure}[H]
    \centering
    \includegraphics[scale = 0.40]{figs/thick_gems_2.png}
    \caption{Plano do \textit{Micromegas} com a esquematização das \textit{thick gems}. Os elétrons quando passam para um campo elétrico mais intenso ionizam o gás, produzindo ainda mais elétrons (evento chamado de avalanche de elétrons). Cada canal da eletrônica de saída produz um pulso como mostrado na parte de baixo da figura.}
    \label{fig:thick_gems}
\end{figure}

\par O sinal é discretizado no tempo levando em conta a velocidade de deriva dos elétrons, dividindo em 512 canais o tempo que o elétron leva para percorrer toda câmara do TPC\cite{josh_bradt, pattpc}. A velocidade de deriva do elétron no gás $^4$He é da ordem de 5 $mm$/$\mu s$\cite{pattpc}, onde o elétron percorre os 50 cm da câmara em cerca de 100 $\mu s$. Dividindo esse tempo pelos 512 canais tem-se que cada canal (\textit{time bucket}) possui 192 $ns$ de largura. Cada centroide detectado é um ponto de interação de uma partícula carrega com o gás. A carga acumulada $Q$ dessa interação é a área do pulso associado ao centroide. Cada centroide então representa um ponto no espaço ($x$, $y$, $t$, $Q$). Um exemplo de evento reconstruído está na figura \ref{fig:event_cap_exp}.

\begin{figure}[H]
    \centering
    \includegraphics[scale = 0.40]{figs/event_cap_exp.png}
    \caption{Evento reconstruído a partir da análise dos pulsos gerados pelo \textit{Micromegas}. A cor representa a carga integrada de cada ponto de interação com o gás.}
    \label{fig:event_cap_exp}
\end{figure}

\par Para reconstituir eventos como o da figura \ref{fig:event_cap_exp}, foram analisados cerca de 300 pulsos. Existem canais auxiliares que servem para evitar armazenar canais sem detecção. Caso haja detecção além do centro do \textit{Micromegas} então os sinais gerados pelo evento são armazenados\cite{josh_bradt, attpc}. O número de eventos reconstruídos é da ordem de milhões, portanto a quantidade de sinais que precisam ser analisados é muito grande, o que gera a necessidade de desenvolvimento de algoritmos extremamente eficientes em tempo para a análise. Para a análise completa do experimento foram seguidas as seguintes etapas:

% \par Nem todos os \textit{pads} do detector são ativados por evento. Existem canais auxiliares que servem para evitar armazenar canais sem detecção. Caso haja detecção além do centro do \textit{Micromegas} então os sinais gerados pelo evento são armazenados\cite{josh_bradt, attpc}. São gerados cerca de 300 sinais por evento, sendo que existem milhões de eventos, o que gera a necessidade de desenvolvimento de algoritmos extremamente eficientes em tempo para a análise. Para a análise completa do experimento precisamos seguir as seguintes etapas:

\begin{itemize}
    \item Análise dos pulsos de cada interação das partículas com o gás. Isso envolve remover o fundo, localizar os picos e obter os tempos e carga integrada de cada caso;
    \item Reconstruir eventos em 3D (nuvens de pontos) a partir da análise de sinais. As nuvens de pontos precisam ser analisadas com algoritmos de reconhecimento de padrões que permitem ajustar as trajetórias das partículas em 3D;
    \item Reconstruir a cinemática das partículas com as trajetórias e energia depositada no gás. Isto permite obter as distribuições angulares. 
\end{itemize}

% \begin{itemize}
%     \item Reconstruir os eventos tridimensionais (nuvens de pontos) a partir dos sinais gerados pelo \textit{micromegas}. Isso inclui remover o fundo, localizar todos os pontos de interação das partículas carregadas com o gás etc. Isso será mostrado em detalhes no capítulo \ref{chapter:sinais};
%     \item A partir das nuvens de pontos é necessário reconstruir a cinemática das reações, identificando trajetórias das partículas e o vértice de reação;
%     \item Com a cinemática reconstruída podemos associar as partículas com as trajetórias e finalmente construir as seções de choque. 
% \end{itemize}

\par A análise dos pulsos, reconstituição de eventos, reconstrução da cinemática, gráficos das distribuições angulares e resultados serão mostradas nos próximos capítulos.

% \chapter{Uma breve introdução ao \textit{Machine Learning}}\label{sec:ml}
\chapter{Desenvolvimento de ferramentas de \textit{machine learning} para analise de dados}\label{sec:ml}

\par Um dos objetivos desse trabalho está em desenvolver ferramentas baseadas em \textit{machine learning} para a análise do grande volume de dados gerada pelo alvo ativo. Nesse capítulo será explicada a metodologia usada para a implementação dessas ferramentas, e algumas aplicações na física nuclear.

\par \textit{Machine learning} é a área de estudo que desenvolve algoritmos para que eles possam aprender com os dados, sem serem explicitamente programados para isso\cite{mlbook}. Supõe-se que uma rede neural imite um sistema biológico, em que os neurônios interajam enviando sinais na forma de funções matemáticas entre as camadas. Isso inspirou um modelo matemático simples para um neurônio artificial\cite{curso}:

\begin{equation}\label{eq:model_n}
    y = f\left(\sum^{n}_{i = 1}\omega_i x_i + bi\right) = f(z),
\end{equation}

\par onde $y$ é a saída do neurônio, que corresponde à função de ativação $f$ que depende da soma ponderada, onde o peso é $\omega_i$, das entradas $x_i$ dos outros $n$ neurônios. O termo $b_i$ corresponde ao parâmetro \textit{bias}. A ideia é fazer um neurônio receber a informação de todos os outros neurônios da camada anterior, fazendo uma média ponderada (onde o peso que será estimado pelo algoritmo de \textit{machine learning}) e somando com um termo independente (\textit{bias}, que também é estimado). Os parâmetros $\omega_i$ e $b_i$ serão estimados através de um determinado procedimento, chamado de minimização (o treino da rede neural).

\section{Tipos de redes neurais}

\par Uma rede neural artificial, \textit{Artificial Neural Network} (ANN), é um modelo computacional que consiste de camadas de neurônios. Muitas ANNs foram desenvolvidas para o estudo de inteligência artificial\cite{mlbook, mldiverso}, mas grande parte consiste em uma camada de entrada (\textit{input layer}), uma camada de saída (\textit{output layer}) e eventuais camadas entre essas duas, chamadas de camadas ocultas (\textit{hidden layers}). Os tipos mais comuns são:

\subsubsection*{Feed-Forward Neural Networks}

\par A \textit{Feed-forward neural networks} (FFNN) é a primeira e mais simples rede neural desenvolvida\cite{talent_ml, bishop2016pattern}. Nessa rede a informação se move apenas para frente através de camadas (da camada de entrada até a camada de saída). A figura \ref{fig:FFNN} mostra uma representação de rede, onde os neurônios são representados por círculos, enquanto que as linhas mostram as conexões entre os neurônios. Cada neurônio recebe informação de todos os neurônios da camada anterior, portanto a rede é chamada de totalmente conectada, \textit{fully-connected} (FC), FFNN. 

\begin{figure}[H]
    \centering
    \includegraphics[scale = 0.55]{figs/FFNN.png}
    \caption{Exemplo de FFNN. A camada de entrada na esquerda propaga a informação para a direita (camada de saída). Todos os neurônios entre camadas estão conectados entre si.}
    \label{fig:FFNN}
\end{figure}

\subsubsection*{Convolutional Neural Network}

\par Uma variante da FFNN é a chamada de rede neural convolucional, \textit{convolutional neural network} (CNN). Do ponto de vista matemático sobre convoluções, a convolução descrita como $(f*g)(t)$ de uma função $f(t)$ e outra $g(t)$ é definida como:

\begin{equation}\label{eq:conv_cont}
    (f*g)(t) \equiv \int^{\infty}_{\infty} f(\tau)g(t - \tau)d\tau.
\end{equation}

Para o caso discreto, com $g$ sendo uma função resposta finita de tamanho $2M$, temos

\begin{equation}\label{eq:conv_disc}
    (f*g)[n] = \sum^{M}_{m = -M} f[n - m]g[m]. 
\end{equation}

\par Convoluções são invariantes sobre rotação e translação, portanto são muito utilizadas para processamento de sinais e imagens\cite{signal_book}. Para a convolução discreta se escolhe um filtro que irá atuar no vetor desejado. Para ilustrar o que significa isso, no caso discreto e unidimensional, a figura \ref{fig:conv_valid} mostra o processo de convolução de um vetor de tamanho 9 e um filtro de tamanho 3.

\begin{figure}[H]
    \centering
    \includegraphics[scale = 0.38]{figs/conv_valid.png}
    \caption{Processo de convolução entre sinal azul em cima e o filtro em verde, resultando no sinal azul embaixo. A multiplicação é feita ponto a ponto e está indicada na caixa azul-clara.}
    \label{fig:conv_valid}
\end{figure}

\par Percebe-se que o sinal resultante tem dimensão menor que o sinal original. O filtro (também chamado de \textit{kernel}) atua em um ponto que possua vizinhos o suficiente para o restante do filtro poder fazer a multiplicação ponto a ponto. Esse tipo de convolução tem o chamado emparelhamento válido (\textit{valid padding}). O tamanho $n_2$ resultante do vetor de saída é 
\begin{equation}
    n_2 = n_1 - m + 1,
\end{equation}

\par onde $n_1$ é o tamanho do vetor de entrada e $m$ o tamanho do filtro (\textit{kernel size}). Para que o vetor de saída tenha o mesmo tamanho do vetor de entrada, são acrescentados zeros em torno da entrada, de forma que a saída tenha o mesmo tamanho da entrada. Esse é o chamado emparelhamento igual (\textit{same padding}). A figura \ref{fig:conv_same} ilustra esse processo.

\begin{figure}[H]
    \centering
    \includegraphics[scale = 0.33]{figs/conv_same.png}
    \caption{Processo de convolução entre o sinal azul em cima e o filtro em verde, resultando no sinal azul embaixo. Agora são acrescentados zeros no inicio e no final do vetor para que o vetor saída tenha o mesmo tamanho do vetor de entrada (nesse caso 9).}
    \label{fig:conv_same}
\end{figure}
%Uma CNN é capaz de fazer convoluções.
\par Como estamos no contexto de \textit{machine learning} (inteligência artificial), \textit{a priori} não sabemos quais os valores dos filtros que devem ser aplicados, apenas seus tamanhos e como agem. A ideia é estimar os valores do filtro (através do treino da rede neural) que deve ser aplicado para se obter o resultado desejado.

\par Cada filtro aplicado gera um mapa característico (\textit{feature map}), que é o resultado da atuação do filtro em um vetor. Usualmente, em uma CNN se escolhe o tamanho do filtro, \textit{padding} (\textit{valid} ou \textit{same}) e quantos filtros serão aplicados (para saber quantos \textit{feature maps} serão gerados). Como temos vários mapas gerados por cada filtro, isso acarreta em um aumento de dimensionalidade. Para filtrar/selecionar os mapas é usado um critério, como por exemplo selecionar valores máximos dos mapas gerados dada uma janela de atuação (quantos mapas serão comparados para selecionar o máximo valor). O \textit{Max-Pooling} faz isso, selecionando valores máximos para uma determinada quantidade de mapas sendo comparados (\textit{pool size}).

% Os filtros têm seus valores estimados pelo treino da rede neural, gerando vários mapas (\textit{feature maps}).

\par Existem outros tipos de redes neurais que não serão discutidas aqui, porém podem ser encontradas nas referências \cite{rbfbook, RNN_fund}.

\section{Construção de uma rede neural}
%, ou seja, todos os dados devem ter o mesmo formato.
\par Para a construção de uma rede neural (nesse caso em específico de uma rede neural supervisionada, que será discutida mais para frente), precisamos primeiro entender sobre os dados que estamos trabalhando. Grande parte das redes neurais possuem um \textit{input} que deve ter dimensão fixa. O mesmo vale para o \textit{output}.

\par Para cada camada da arquitetura devemos escolher sua função de ativação. Tanto FNNNs quanto CNNs podem possuir funções de ativação (função $f$ da equação \ref{eq:model_n}). Dentre muitas funções de ativação podemos citar a \textit{Rectified Linear Units} (ReLU)\cite{RELU}, sigmoide\cite{sigmoid_act}, linear e tangente hiperbólica\cite{act_comp}. A figura \ref{fig:ativacoes} mostra os gráficos dessas funções de ativação, onde no eixo x é o argumento e no eixo y o resultado da função.

\begin{figure}[H]
    \centering
    \includegraphics[scale = 0.7]{figs/ativacoes.png}
    \caption{Funções de ativação e seus respectivos gráficos.}
    \label{fig:ativacoes}
\end{figure}

\par O próximo passo é definir a função custo (também chamada de \textit{loss}) e o otimizador. A função custo tem o papel de retornar valores altos para previsões erradas e valores baixos para previsões corretas. Por exemplo, se queremos treinar uma rede neural para classificação binária (que prevê duas saídas possíveis), devemos usar a função custo chamada de \textit{binary cross-entropy} dada por\cite{dl_book}

\begin{equation}\label{eq:binary_cross_entropy}
    C(p(y_i)) = -\frac{1}{N}\sum_{i = 1} ^N y_i \log(p(y_i)) + (1 - y_i)\log(1 - p(y_i)),
\end{equation}

\par onde $y_i$ é o rótulo (\textit{label}), $p(y_i)$ é a probabilidade do ponto $y_i$ ser 1 e $N$ é o número de pontos. O objetivo da rede neural é achar o mínimo da função $C(p(y_i))$, o que implica diretamente na melhor solução para o conjunto de dados. Isso é feito pelo método de retropropagação do erro (\textit{backpropagation}\cite{backpropagation}) por um otimizador. Outros exemplos de \textit{loss} são o erro quadrático médio ou \textit{categorical cross-entropy}\cite{MSE_CEF_review}.

\par O otimizador tem o objetivo de otimizar os parâmetros presentes na rede neural, buscando o mínimo global da função custo, o que nem sempre acontece, pois a minimização pode parar em um mínimo local da função. Existem diversos otimizadores, como por exemplo o \textit{Stochastic Gradient Descent} (SGD), ADAM, ADAMAX\cite{ADAMAX}, entre outros\cite{gradient_over}. Para o SGD, temos que a atualização de parâmetros é dada por

\begin{equation}\label{eq:SGD}
    \theta_j = \theta_{j} - \alpha \frac{\partial }{\partial \theta_j}C(\theta),
\end{equation}

\par onde $\theta_j$ é o parâmetro a ser atualizado, $\alpha$ é a \textit{learning rate} e $C(\theta)$ é a \textit{loss} que depende dos parâmetros $\theta$.

\par Para enfim treinar a rede neural, se escolhe o \textit{batch size}, que é o tamanho de amostras que será usada para o treino, por iteração em cada rodada de treino (\textit{epoch}). Por exemplo, se usamos 1000 dados para o treino, e o \textit{batch size} é 500, cada \textit{epoch} terá duas iterações. No geral, se usam \textit{batch sizes} pequenos, pois o consumo de memória é mais eficiente.

\par Para avaliação do modelo se usam dados de validação, que servem para verificar o comportamento da rede neural que está sendo treinada. Esse conjunto de dados usados não é usado para o treino, são usados apenas para verificar possíveis problemas como o \textit{overfit}. O \textit{overfit} ocorre quando a rede neural começa a se adequar perfeitamente aos dados de treino, perdendo a capacidade de previsão em dados que não estão sendo usados no treino pela rede neural.

\par Além dos dados de validação, podemos escolher métricas que auxiliam a visualização do comportamento da rede neural durante o treino e nos retornam informações importantes sobre sua qualidade. Exemplos importantes de métricas são: acurácia binária, erro médio absoluto e acurácia categórica\cite{metrics}. Por exemplo, caso seja necessário verificar se uma rede neural está fazendo previsões certas em um problema cuja classificação é binária, então a métrica deve ser a acurácia binária. Tudo depende do objetivo da rede neural.

\section{Sistemas de \textit{machine learning}}

Podemos dividir os sistemas de \textit{machine learning} em quatro tipos:

\subsubsection*{Aprendizado Supervisionado}
% \begin{enumerate}
%     \item aprendizado supervisionado;
%     \item aprendizado não supervisionado;
%     \item aprendizado semi supervisionado;
%     \item aprendizado por reforço.
% \end{enumerate}

\par Aprendizado supervisionado é quando fornecemos para a rede neural um conjunto de dados para o treino com a solução desejada (chamados de \textit{labels}). Um uso típico é para problemas de classificação\cite{mlbook}. Por exemplo, classificação de imagens (identificação de figuras), previsão de valores numéricos etc. Exemplos de de algoritmos supervisionados são:

\begin{itemize}
    \item \textit{k-Nearest Neighbors}\cite{knn}
    \item Regressão linear
    \item \textit{Support Vector Machines} (SVMs)
    \item \textit{Decision Trees} and \textit{Random Forests}
    \item Redes neurais
\end{itemize}

\subsubsection*{Aprendizado não supervisionado}

\par Aprendizado não supervisionado é quando fornecemos o conjunto de dados para o treino, porém sem solução. A ideia é aprender sem supervisão. Um problema comum, por exemplo, é quando queremos identificar \textit{clusters} em um conjunto de dados (\textit{clustering})\cite{unsupervised, knn_uns}.

\subsubsection*{Aprendizado semi supervisionado}

\par Aprendizado supervisionado é quando apenas parte do conjunto de dados para o treino possui \textit{labels}. Isso é comum quando se obtém conjuntos de dados diferentes e apenas parte deles foi classificado\cite{semi_supervised}.

\subsubsection*{Aprendizado por reforço}

\par Aprendizado por reforço é quando um sistema, chamado de \textit{agente} nesse contexto, aprende através do ambiente, realizando ações que maximizam sua recompensa. Por exemplo, caso o sistema realize uma ação incorreta, ele recebe uma penalidade, fazendo com que procure outra maneira de realizar a ação, dessa vez de maneira correta, para poder ganhar uma recompensa\cite{reinforcement}. Esse tipo de sistema é muito usado, por exemplo, em automatização robótica, como carros que pilotam sozinhos, robôs que aprendem a andar etc\cite{robot_ml}.

\section{Aplicações de \textit{machine learning} na física nuclear}

\par Em física nuclear, o uso de técnicas de \textit{machine learning} tem se mostrado cada vez mais importante. Na continuação serão citados alguns exemplos.

\subsection{Análise de espectros para identificação de partículas (\textit{particle identification}, PID)}

\par O uso de técnicas de aprendizado não supervisionado pode ser usado para a identificar de partículas em espectros $\Delta E$ - $E$. Neste tipo de espectros, as partículas são detectadas e separadas por regiões que facilmente podem ser identificadas visualmente\cite{DETC_TELE}. No entanto, a identificação visual desses espectros com os sistemas de detecção com milhares de canais é inviável de forma manual. Assim, ferramentas de \textit{machine learning} (ou inteligencia artificial) são de grande relevância para esse tipo de analise.

Os resultados apresentados nessa seção foram obtidos na etapa inicial desta dissertação. Para o presente estudo, o espectro biparamétrico ($\Delta E$-$E$) mostrado na Figura \ref{subfig:espectro_1} será analisado. Os dados correspondem a reação de espalhamento e transferência no sistema $^{6}$Li +  $^{12}$C.


% O espectro biparamétrico ($\Delta E$ - $E$) mostrado pela figura \ref{subfig:espectro_1} é o resultado obtido do espalhamento do $^6$Li em $^{12}$C por um detector telescópio\cite{DETC_TELE}.

\par O objetivo é identificar conjuntos (ou \textit{clusters}) diferentes a partir do espectro. É claro para o olho humano que existem conjuntos diferentes, e eles podem ser identificados usando algoritmos de aprendizado não supervisionado, como por exemplo o \textit{density-based spatial clustering of applications with noise} (DBSCAN)\cite{dbscan}. O DBSCAN consegue identificar \textit{clusters} apenas com as informações características de densidade existente pelos conjuntos que existem nos dados. O resultado da aplicação do algoritmo está na figura \ref{subfig:espectro_2}, onde é mostrado que o espectro agora está separado por diferentes conjuntos.
%\newpage

\begin{figure}[H]
\centering
    \begin{subfigure}[t]{0.48\textwidth}
        \centering
        \includegraphics[scale=0.21]{figs/espectro_1.png}
        \caption{Espectro biparamétrico $\Delta E$ x $E$ (energia no referencial do centro de massa).}
        \label{subfig:espectro_1}
    \end{subfigure}%
    \hfill
    \begin{subfigure}[t]{0.48\textwidth}
        \centering
        \includegraphics[scale=0.52]{figs/espectro_2.png}
        \caption{Espectro biparamétrico $\Delta E$ x $E$ (energia no referencial do centro de massa) com os conjuntos identificados com o uso do DBSCAN.}
        \label{subfig:espectro_2}
    \end{subfigure}
\caption{Espectros biparamétricos $\Delta E$ x $E$, em \ref{subfig:espectro_1} o espectro cru com uma pré marcação das partículas, e em \ref{subfig:espectro_2} os conjuntos identificados usando \textit{machine learning}.}
\label{fig:espectros_ex_ml}
\end{figure}

\par Usando o conjunto 1, por exemplo, é possível criar o histograma de identificação de partículas dado pela figura \ref{fig:PID_hist}.

\begin{figure}[H]
    \centering
    \includegraphics[scale=0.11]{figs/espectro_3.png}
    \caption{Histograma de identificação de partículas, a partir do conjunto 1 identificado na figura \ref{subfig:espectro_2}. Em \textbf{a} temos o primeiro pico que corresponde ao estamos fundamental do $^{14}$N e em \textbf{b} temos o primeiro estado excitado do $^{14}$N.}
    \label{fig:PID_hist}
\end{figure}

\subsection{Estimativa de raios e massas nucleares}

\par Frequentemente é necessário calcular com alta precisão observáveis que ainda não foram medidos, para contribuir com dados já existentes. É possível estimar propriedades de núcleos usando modelos e dados experimentais já existentes em conjunto com técnicas de \textit{machine learning}.

\par É possível usar redes neurais que façam previsão de massa nucleares com informações de massa já disponíveis. Isso foi feito usando o algoritmo \textit{Light Gradient Boosting Machine} (LightGBM)\cite{LightGBM}. O algoritmo é uma rede neural cujo aprendizado é supervisionado e minimiza o erro entre a energia de ligação teórica e a energia de ligação experimental dos núcleos (essa diferença é chamada de resíduo), usando 10 quantidades físicas como dados de entrada (\textit{input})\cite{nuclear_mass}. O desvio quadrático médio para a massa dos núcleos é de 0.234 $\pm$ 0.022 MeV, valor melhor que muitos modelos físicos de massa nuclear. A figura \ref{fig:nuclei_mass_ml} mostra os resíduos calculados entre a energia de ligação calculada pelo rede neural e a energia de ligação determinada experimentalmente.

\begin{figure}[H]
    \centering
    \includegraphics[scale = 0.60]{figs/nuclei_mass.png}
    \caption{Painel acima mostra a localização dos núcleos usados para o treino da rede neural. No painel de baixo temos o erro entre a previsão e o valor experimental da energia de ligação do núcleo para os núcleos usados como dados de validação. $\sigma$ é o erro quadrático médio da rede neural.}
    \label{fig:nuclei_mass_ml}
\end{figure}

\par Outro exemplo de uso é para a estimativa de raio de carga nuclear. Usando uma rede neural (chamada de \textit{Bayesian neural network extended liquid drop} - BNN-ELD) FC com duas variáveis de entrada, o número de prótons Z e o número de massa A, com a condição de que Z $\geq$ 20 e A $\geq$ 40. A rede é supervisionada e usa 722 núcleos para dados de treino e 98 núcleos como validação. Resultados para o desvio quadrático médio do raio de carga são de cerca de 0.02 fm para os 820 núcleos utilizados. A figura \ref{fig:radii} mostra a diferença entre o raio de carga calculado teoricamente e o determinado experimentalmente para isótopos do chumbo, usando diferentes neurais e modelos teóricos. Mais detalhes podem ser encontrados na Ref. \cite{raio_carga}.

\begin{figure}[H]
    \centering
    \includegraphics[scale = 0.60]{figs/radii.png}
    \caption{Previsões para o raio de carga para isótopos do chumbo (Z = 82) para diferentes modelos teóricos ou que usam redes neurais. A previsão que contém barras de erro é a que foi brevemente descrita no texto.}
    \label{fig:radii}
\end{figure}

% \par Dada a capacidade de redes neurais de aprenderem padrões não lineares nos dados, isso tem se mostrado muito útil uma alternativa para estimar propriedades dos núcleos como: raio de carga nuclear \cite{raio_carga}, massa de núcleos \cite{nuclear_mass} etc.

\subsection{Decaimento beta e processo $r$}

\par O decaimento $\beta$ é fundamental para entender a origem dos elementos pesados. Prever o tempo de meia vida do decaimento $\beta$ é de grande importância para simulações do processo $r$ (captura rápida de nêutrons). Com redes neurais é possível fazer previsões que levam em conta a física do problema, como visto na Ref. \cite{mlbetadecay}. O modelo de inteligência artificial leva em conta a teoria de Fermi para o decaimento beta, onde a função $f$ que é minimizada (função custo) é dada por

\begin{equation}\label{eq:half_life}
    f = \log_{10}(T^{a}_{1/2}/T^{b}_{1/2}),
\end{equation}

onde $T^{a}_{1/2}$ é o tempo de meia vida do decaimento beta medido experimentalmente e $T^{b}_{1/2}$ é o tempo de meia vida do decaimento beta determinado teoricamente. As variáveis de entrada são o número de prótons Z, o número de nêtrons N, a energia total do decaimento beta e o parâmetro de paridade $\delta$ = 1, 0, -1, para núcleos par-par, ímpar-par e ímpar-ímpar, respectivamente.

\par A figura \ref{fig:beta_decay} mostra a previsão do tempo de meia vida ($T_{1/2}$) do decaimento $\beta$ (em segundos) para isótonos com número de nêutrons N = 126 usando redes neurais.

\begin{figure}[H]
    \centering
    \includegraphics[scale = 0.35]{figs/beta_predict_2.png}
    \caption{Meias-vidas de decaimento $\beta$ para isótonos com N = 126. A região hachurada verde mostra as previsões de uma rede neural. A região hachurada em azul mostra os resultados da mesma rede neural, porém seus dados de aprendizado são estendidos para incluir três meias-vidas extras de decaimento $\beta$ para cada isótopo (indicado por círculos abertos) em direção à \textit{drip-line} de nêutrons\cite{mlbetadecay}.}
    \label{fig:beta_decay}
\end{figure}

\subsection{Alvos ativos}

\par Experimentos com alvos ativos geram enormes quantidades de dados. Uma semana de experimento pode gerar até 10Tb de dados\cite{KUCHERAML}, o que gera a necessidade do uso de algoritmos de \textit{machine learning} para diminuir o consumo de tempo da análise desses dados.

\par A análise dos dados envolve a reconstrução tridimensional dos eventos e posteriormente a reconstrução da cinemática, com a identificação de partículas e de reações nucleares. O uso de técnicas de \textit{machine learning} pode diminuir o consumo de tempo dessas etapas.

\par O uso de CNNs em imagens feitas a partir das projeções de eventos pode ser útil para a classificação de eventos, sem a necessidade de reconstruir a cinemática em uma etapa anterior. A figura \ref{fig:tpc_ml} mostra exemplos de projeções de trajetórias de partículas dentro do alvo ativo, onde a partir da projeção no plano $xy$ é possível identificar a partícula que a originou.

\par Primeiro, com o objetivo de classificar as projeções entre próton ou carbono, é possível construir uma rede neural para a classificação binária. Para isso foi usada uma arquitetura com camadas convolucionais seguidas de \textit{max-pooling} e por fim uma camada FC\cite{KUCHERAML}. A função a ser minimizada é a dada pela equação \ref{eq:binary_cross_entropy} e a métrica para entender o comportamento da rede neural é a acurácia binária.

\par Caso o objetivo seja classificar entre três ou mais possibilidades (como por exemplo próton, carbono e outros), então a classificação não será binária, passará a ser categórica, pois será necessário classificar a imagem em alguma categoria. A função custo passará a ser a \textit{categorical cross-entropy} e a métrica será a acurácia categórica.

\begin{figure}[H]
    \centering
    \includegraphics[scale = 0.5]{figs/tpc_ml.png}
    \caption{Projeções no plano $xy$ de partículas dentro do alvo ativo, onde quanto mais escura for a cor, mais carga tem o ponto, e quanto mais clara a cor, menos carga tem o ponto. O objetivo da rede neural pode ser classificar corretamente se a imagem à direita corresponde à uma trajetória de um próton, carbono ou outra partícula, ou pode ser uma rede neural para classificação binária caso seja necessário classificar apenas entre próton ou carbono\cite{KUCHERAML}.}
    \label{fig:tpc_ml}
\end{figure}

\par A rede criada para a classificação binária (entre próton e carbono) atingiu cerca de 90\% de acurácia. Somado ao fato de que redes neurais são muito eficientes em tempo, isso mostra uma possibilidade de que seja possível utilizar redes neurais como essa para o processamento em tempo real dos dados, com pouca supervisão humana durante o processo\cite{KUCHERAML}.

\par Outro problema que pode ser resolvido com técnicas de \textit{machine learning} é a remoção de ruído de eventos. Para isso, pode ser usado o algoritmo não supervisionado \textit{Hough transform}\cite{hough}, que é capaz de detectar padrões de interesse nos dados e evitar regiões ou pontos que não seguem o padrão desejado. 


% Além disso são geradas nuvens de pontos tridimensionais, o que demanda um grande trabalho do ponto de vista de visão computacional, com algoritmos capazes de identificar estruturas em três dimensões. A figura \ref{fig:tpc_ml} mostra um exemplo de \textit{clustering} aplicado na identificação de trajetórias tridimensionais.

% \begin{figure}[H]
%     \centering
%     \includegraphics[scale = 0.5]{figs/tpc_ml_2.png}
%     \caption{Resultado do algoritmo iterativo \textit{Hough transform}\cite{hough} que separa os pontos originados de interações de partículas carregadas com o alvo ativo. O algoritmo não é 100\% correto e neste evento alguns dos pontos atribuídos à trajetória 0 e à trajetória 3 na verdade pertencem à trajetória 2.\cite{DALITZ2019159}}
%     \label{fig:tpc_ml}
% \end{figure}

    % \caption{Projeções de partículas dentro do alvo ativo. O objetivo da rede neural é classificar corretamente se a imagem à direita corresponde à uma trajetória de um próton, carbono ou outra partícula.}

\par Com essa breve descrição sobre \textit{machine learning} e exemplos do seu uso em problemas de física nuclear, pode-se seguir adiante e entender seu uso dentro deste trabalho, que será feito nos próximos capítulos.

\chapter{Reconstrução de nuvens de pontos a partir de algoritmos de \textit{machine learning}}\label{chapter:sinais}

\par Esse capítulo irá descrever o procedimento usado para criar as nuvens de pontos (\textit{pointclouds}) a partir dos pulsos gerados por cada pixel do \textit{micromegas}, usando algoritmos de \textit{machine learning} supervisionado. Na seção \ref{sec:pulses_trad}, está mostrado como o banco de dados para o treino das redes neurais desenvolvidas foi criado, e a construção das redes neurais está na seção \ref{sec:pulsos_ml}.


% os sinais foram analisados a partir de métodos mais tradicionais e depois criando algoritmos de \textit{machine learning}, usando das soluções anteriores como base para o seu funcionamento.


% \par A quantidade de histogramas armazenados ultrapassa facilmente a casa das centenas de milhões, portanto é necessário o desenvolvimento de algoritmos extremamente rápidos para a análise.


\section{Construção do banco de dados para as redes neurais}\label{sec:pulses_trad}

\par Essa seção irá descrever o processo de construção do banco de dados que foi usado para o treino das redes neurais das subseções \ref{subsec:pulso_ml_fundo}, \ref{subsec:pulso_ml_deconv} e \ref{subsec:pulso_ml_peaks}. Os sinais são gerados pelo detector \textit{micromegas} e cada um de seus canais possui eletrônica independente e os sinais recebidos são como os mostrados na figura \ref{fig:exemplos_sinais}.

\begin{figure}[H]
\centering
    \begin{subfigure}[b]{0.48\textwidth}
        \centering
        \includegraphics[scale=0.395]{figs/ex_sinal_1.png}
        \caption{}
        \label{subfig:exemplos_sinais_1}
    \end{subfigure}%
    \hfill
    \begin{subfigure}[b]{0.48\textwidth}
        \centering
        \includegraphics[scale=0.395]{figs/ex_sinal_2.png}
        \caption{}
        \label{subfig:exemplos_sinais_2}
    \end{subfigure}
\caption{Exemplos de sinais produzidos pelos canais do detector. Em \ref{subfig:exemplos_sinais_1} o sinal possui apenas um pulso, enquanto em \ref{subfig:exemplos_sinais_2} há vários pulsos em sobreposição, formando um único pulso com largura maior que em \ref{subfig:exemplos_sinais_1}.}
\label{fig:exemplos_sinais}
\end{figure}

\par No eixo $x$, cada um dos 512 \textit{time buckets} possui largura de 195 $n s$. No eixo $y$ tem-se a carga acumulada no detector para cada \textit{time bucket}. Na figura \ref{subfig:exemplos_sinais_1} há um sinal com um pedestal (fundo ou \textit{baseline}) com altura entre 300 e 450, e um pulso estreito em cima. Como mostrado na seção \ref{PATTPC}, os elétrons que surgiram da ionização do gás foram conduzidos perpendicularmente pelo campo elétrico até o detector. A interação da partícula com o gás é evidenciada justamente pelo pulso presente em \ref{subfig:exemplos_sinais_1}. Cada pixel $i$ do detector está em uma posição ($x_i$, $y_i$), o centroide de cada gaussiana fornece a coordenada em $t$ (\textit{time bucket}) para então ser convertida na posição em $z$ da partícula detectada, e a carga acumulada $Q$ do ponto é a área do pulso sem o fundo (gaussiana com centroide $t$).

\par Para a figura \ref{subfig:exemplos_sinais_1} tem-se apenas um pulso estreito, o que corresponde à um feixe incidindo paralelamente àquele canal do plano detector (perpendicular ao campo elétrico), pois a projeção da interação da partícula com gás no tempo é uma distribuição estreita. No caso da figura \ref{subfig:exemplos_sinais_2}, há uma distribuição ampla do sinal do tempo, o que corresponde ao feixe incidindo perpendicularmente ao canal do plano detector. A ilustração desse processo está na figura \ref{fig:get_signal}, que mostra o processo da passagem de uma partícula carregada e como o sinal é gerado a partir disso.


% único pixel do detector ativado com coordenadas($x$, $y$, $z$). Já para \ref{subfig:exemplos_sinais_2} temos o que é chamado de mistura de gaussianas (\textit{gaussian mixture}), que é a presença de várias gaussianas sobrepostas. Esse tipo de sinal corresponde ao feixe indo perpendicularmente ao pixel do plano do detector, indicando. A ilustração desse processo está na figura \ref{fig:get_signal}, que mostra o processo da passagem de uma partícula carregada e como o sinal é gerado a partir disso.

% As gaussianas presentes em \ref{subfig:exemplos_sinais_2} devem ter a mesma largura da gaussiana presente em \ref{subfig:exemplos_sinais_1}. 

\begin{figure}[H]
    \centering
    \includegraphics[scale = 0.325]{figs/get.png}
    \caption{Ilustração que mostra a variação no formato da carga coletada a partir da passagem de uma partícula carregada dentro do TPC, onde o plano do detector está embaixo. No lado esquerdo de cada imagem, a distribuição do sinal coletado por um único pad (escuro) do plano de coleta é mostrado (o canal eletrônico de leitura é representado pela seta cinza em negrito). No caso de uma trajetória quase horizontal em relação ao plano do detector (a), o sinal é uma distribuição estreita, enquanto para uma trajetória próxima a uma direção vertical (ou perpendicular) em relação ao detector (b), a distribuição deve ser muito mais ampla (vários pontos de interação da partícula com o gás devem ser extraídos desse sinal). A última imagem ilustra o caso em mais de uma trajetória de partículas contribui para o sinal\cite{GET}.}
    \label{fig:get_signal}
\end{figure}

\par Para analisar os pulsos deve-se primeiro remover o fundo (pedestal ou \textit{baseline}) dos sinais. O sinal de fundo é complexo e pode variar por canal e também por evento. Desde flutuações causadas pelo circuito eletrônico até efeitos sistemáticos gerados pela memória de \textit{buffer} circular alteram o sinal, podendo o tornar não analítico \cite{FORTINO2022166497, GET}. Com o sinal sem o fundo, se faz a deconvolução para determinar todos os centroides e cargas acumuladas dos pontos, para a reconstrução da nuvem de pontos. Todo esse processo pôde ser feito com algoritmos de \textit{machine learning} supervisionado\cite{FORTINO2022166497}. Para isso, foi criado um banco de dados que irá servir de \textit{output} e/ou \textit{input} para o treino das redes neurais.

\par A criação dos dados para o treino das redes neural será mostrado na seção \ref{subsec:pulses_baseline}. A criação das redes neurais está mostradas na seção \ref{sec:pulsos_ml}. 

% O fundo não é trivial de se determinar, pois não é analítico, oscilando muito entre os canais.

\subsection{Estimativa do fundo}\label{subsec:pulses_baseline}

\par Essa subseção descreve a estimativa do fundo de cada, para formar o banco de dados da rede neural que estima o fundo descrita na subseção \ref{subsec:pulso_ml_fundo}.

\par A primeira tentativa de estimar o sinal sem o fundo é usando transformada de Fourier e um filtro passa-baixa, que é a função resposta do detector fornecida na Ref.\cite{GET}. Seja $f(t)$ uma função qualquer, sua transformada de Fourier é dada por

\begin{equation} \label{eq:fourier}
    \hat{f}(\nu)=\mathscr{F}[f(t)]=\int_{-\infty}^{\infty} f(t) e^{-2 \pi i \nu t} d t.
\end{equation}

\par Primeiro calculamos a transformada de Fourier $\hat{f}(\nu)$ do sinal, em seguida multiplicamos pela função resposta do detector $h(\nu)$ dada por\cite{GET}

\begin{equation}
    h(\nu) = A*\exp\left (\nu \tau \right)\left(\nu \tau\right)^3 \sin \left( \nu \tau \right) ,
\end{equation}

%\begin{equation}
%    \sinc x \equiv \frac{\sin (\pi x)}{\pi x},
%\end{equation}

\par onde $A$ está relacionado com o ganho de amplificação e $\tau$ é o tempo de pico (\textit{peaking time}), que é o tempo de modelagem da cadeia de amplificação \cite{GET}. Do teorema da convolução, temos que \cite{metodos_mat_aplicada}

%\par onde $a$ é um fator de escala. A função sinc foi escolhida para tirar vantagem do Teorema da Convolução, dado por\cite{metodos_mat_aplicada}

\begin{equation}
    \mathscr{F}^{-1}[\hat{f}(\nu) \hat{g}(\nu)]=(f * g)(t)=\int_{-\infty}^{\infty} f(\tau) g(t-\tau) d \tau, 
\end{equation}

\par onde $(f * g)(t)$ é a convolução entre $f(t)$ e $g(t)$. Multiplicar o sinal transformado por $h(\nu)$ e depois inverter inverter a transformação é o mesmo que convoluir o sinal original com a transformação inversa de $h(\nu)$, o que resulta no sinal sem o fundo\cite{josh_bradt, GET}. Resultados desse procedimento estão na figura \ref{fig:bs_fourier_exs}.

%Multiplicar o sinal transformado por $\sinc (\nu / a)$ e depois inverter inverter a transformação é o mesmo que convoluir o sinal original com a transformação inversa de $\sinc (\nu / a)$, pois

%\begin{equation}
%\mathscr{F}^{-1}[\sinc(\nu)]=\rect(t) \equiv \begin{cases}1, & -\frac{1}{2}<t<\frac{1}{2} \\ 0, & \text { qualquer outro }t\end{cases},
%\end{equation}
%
%é uma função que representa uma janela retangular, o que seria um exemplo de função resposta do detector. Caso saibamos essa função resposta, então é possível reconstruir completamente o sinal. Resultados desse procedimento estão na figura \ref{fig:bs_fourier_exs}.

\begin{figure}[H]
\centering
    \begin{subfigure}[c]{0.45\textwidth}
        % \centering
        \includegraphics[scale=0.45]{figs/bs_fourier_1.png}
        \caption{}
        \label{subfig:bs_fourier_1}
    \end{subfigure}%
    \hfill
    \begin{subfigure}[c]{0.45\textwidth}
        % \centering
        \includegraphics[scale=0.45]{figs/bs_fourier_2.png}
        \caption{}
        \label{subfig:bs_fourier_2}
    \end{subfigure}
    \begin{subfigure}[b]{0.45\textwidth}
        \centering
        \includegraphics[scale=0.45]{figs/bs_fourier_3.png}
        \caption{}
        \label{subfig:bs_fourier_3}
    \end{subfigure}%
    \hfill
    \begin{subfigure}[b]{0.45\textwidth}
        \centering
        \includegraphics[scale=0.45]{figs/bs_fourier_4.png}
        \caption{}
        \label{subfig:bs_fourier_4}
    \end{subfigure}
\caption{Histogramas com as respectivas \textit{baselines} (linhas tracejadas) estimadas pelo método da convolução. O espectro resultante (sem o fundo) está em verde.}
\label{fig:bs_fourier_exs}
\end{figure}

% Dos exemplo acima em muitos casos, acaba estimando o sinal original, na região do pulso, menor do que deveria ser, fazendo com que o sinal tenha menos carga do que deveria.

\par Fica claro que visualmente, por exemplo na figura \ref{subfig:bs_fourier_4}, que o filtro utilizado não é a melhor função resposta do detector. Poderia-se estimar essa função resposta empiricamente, porém os canais auxiliares chamados de \textit{Fixed Pattern Noise} (FPN)\cite{GET} usados para esta estimativa não foram armazenados. Portanto, a estimativa do fundo foi feita sinal por sinal\cite{FORTINO2022166497, GET}. Para isso, o fundo foi determinado usando o algoritmo \textit{background removal} da biblioteca \textit{TSpectrum} do \textit{ROOT} \cite{root}. A função tem a capacidade de separar o fundo dos picos presentes no espectro\cite{BKG_1, BKG_2, BKG_3}. Exemplos de estimativa do fundo estão na figura \ref{fig:ex_sinal_bkg}.

%, pois o fundo é não analítico\cite{GET}. Para um resultado mais adequado, o fundo será determinado usando o algoritmo \textit{background removal} da biblioteca \textit{TSpectrum} do \textit{ROOT} \cite{root}. A função tem a capacidade de separar o fundo dos picos presentes no espectro\cite{BKG_1, BKG_2, BKG_3}. Exemplos de estimativa do fundo estão na figura \ref{fig:ex_sinal_bkg}.

\begin{figure}[H]
\centering
    \begin{subfigure}[b]{0.48\textwidth}
        \centering
        \includegraphics[scale=0.395]{figs/ex_sinal_bkg_1.png}
        \caption{}
        \label{subfig:ex_sinal_bkg_1}
    \end{subfigure}%
    \hfill
    \begin{subfigure}[b]{0.48\textwidth}
        \centering
        \includegraphics[scale=0.395]{figs/ex_sinal_bkg_2.png}
        \caption{}
        \label{subfig:ex_sinal_bkg_2}
    \end{subfigure}
    \begin{subfigure}[b]{0.48\textwidth}
        \centering
        \includegraphics[scale=0.395]{figs/ex_sinal_bkg_3.png}
        \caption{}
        \label{subfig:ex_sinal_bkg_3}
    \end{subfigure}%
    \hfill
    \begin{subfigure}[b]{0.48\textwidth}
        \centering
        \includegraphics[scale=0.395]{figs/ex_sinal_bkg_4.png}
        \caption{}
        \label{subfig:ex_sinal_bkg_4}
    \end{subfigure}
\caption{Histogramas com as respectivas \textit{baselines} (linhas tracejadas) calculadas pelo \textit{TSpectrum}.}
\label{fig:ex_sinal_bkg}
\end{figure}

\par Os resultados do calculo do fundo de cada sinal foram armazenados e usados para a etapa de deconvolução do sinal, que deve ser feita com o sinal sem o fundo. Para evitar valores negativos após a remoção do fundo, o valor mínimo do sinal sem o fundo é zero. Sem o fundo podemos buscar por todos os picos e suas cargas correspondentes no sinal. Não podemos detectar diretamente todos os picos pois muitos deles estão em sobreposição. Para isso foi feita a deconvolução do sinal, descrita na subseção \ref{subsec:pulses_deconv}.

% \par Com o fundo podemos então subtraí-lo do espectro. Podem aparecer valores negativos após a retirada do fundo, então para evitar esse problema o valor mínimo do sinal, após a retirada do fundo, é zero.


\subsection{Deconvolução do sinal}\label{subsec:pulses_deconv}

\par Essa subseção descreve a etapa de deconvolução do sinal, que serve de banco de dados para o treinamento da rede neural descrita na subseção \ref{subsec:pulso_ml_deconv}.

\par Para aumentar a resolução dos picos foi usado o algoritmo \textit{gold deconvolution} presente na biblioteca \textit{TSpectrum} do \textit{ROOT}\cite{paper_gold_deconv}. O algoritmo tem como objetivo fazer a deconvolução do espectro, gerando uma função (nesse caso um sinal) resposta de acordo com o sigma esperado para os pulsos. O sinal resposta corresponde ao espectro com as gaussianas não sobrespostas. Isso significa que foi necessário determinar qual o valor de sigma dos pulsos para buscar a função resposta.

\par O sigma dos pulsos é o mesmo de um sinal que possui apenas um pico. Ou seja, o sigma dos pulsos foi determinado fazendo a análise de sinais que possuem apenas 1 pico, fazendo um ajuste pelo método dos mínimos quadrados (MMQ) de uma gaussiana. Para buscar espectros com apenas um pico, foi usado o algoritmo de detecção de picos \textit{peak\_finder}, presente na biblioteca do \textit{scipy}\cite{scipy}, e para o ajuste da gaussiana foi usada o pacote \textit{lmfit} \cite{lmfit}. O valor de sigma encontrado foi de 4.09 (17) \textit{time buckets}.

\par O sigma escolhido foi ligeiramente maior pois, verificando empiricamente, em alguns casos o algoritmo separava o que deveria ser uma única gaussiana em duas. O valor de sigma usado na deconvolução foi de 4.30 \textit{time buckets}. Foi determinado também o número de iterações do algoritmo de deconvolução. O número de iterações escolhido foi de 700, menos que isso o algoritmo não estava separando totalmente picos sobrepostos. O limiar para a escolha de um ponto como um pico foi definido como ter altura maior que 20\% do valor máximo do sinal. Resultados da deconvolução estão na figura \ref{fig:ex_sinal_deconv}.

\begin{figure}[H]
\centering
    \begin{subfigure}[b]{0.48\textwidth}
        \centering
        \includegraphics[scale=0.40]{figs/ex_deconv_1.png}
        \caption{}
        \label{subfig:ex_sinal_deconv_1}
    \end{subfigure}%
    \hfill
    \begin{subfigure}[b]{0.48\textwidth}
        \centering
        \includegraphics[scale=0.40]{figs/ex_deconv_2.png}
        \caption{}
        \label{subfig:ex_sinal_deconv_2}
    \end{subfigure}
    \begin{subfigure}[b]{0.48\textwidth}
        \centering
        \includegraphics[scale=0.40]{figs/ex_deconv_3.png}
        \caption{}
        \label{subfig:ex_sinal_deconv_3}
    \end{subfigure}%
    \hfill
    \begin{subfigure}[b]{0.48\textwidth}
        \centering
        \includegraphics[scale=0.40]{figs/ex_deconv_4.png}
        \caption{}
        \label{subfig:ex_sinal_deconv_4}
    \end{subfigure}
\caption{Histogramas sem as \textit{baselines} antes (em azul) e depois da deconvolução (em vermelho). Os picos (em verde) e o limiar (linha tracejada preta) de detecção também estão indicados.}
\label{fig:ex_sinal_deconv}
\end{figure}

\par O algoritmo de deconvolução também retorna a posição dos centroides encontrados, que indica a localização de um pico. Os mesmo picos podem ser obtidos com o \textit{peak\_finder}, com a vantagem de que o algoritmo possui muitos parâmetros diferentes para calibração, melhorando a detecção em comparação com os picos detectados pelo algoritmo do TSpectrum. A execução de 200.000 sinais, desde a estimativa e remoção do fundo, até a detecção dos centroides, demora cerca de 23.25 minutos, usando o processador Ryzen 5 3600X.

\par Para determinar a carga acumulada $Q$ de cada ponto é necessário calcular a área do centroide do pico detectado. A área do sinal antes e depois da deconvolução é a mesma, mesmo para a região dos pulsos, portanto pode-se analisar diretamente o sinal após a deconvolução. Para achar a área do pulso, foi calculado o sigma dos pulsos após a deconvolução, para determinar a área como uma simples integral gaussiana. O sigma dos pulsos após a deconvolução é $\sigma_{dd}$ = 1.1543 (44) \textit{time buckets}. Com isso foi calculada a carga acumulada para cada ponto descoberto do evento. A carga acumulada $Q$ para cada ponto $i$ é dada por:

\begin{equation}\label{eq:gauss_area}
    Q = \int^\infty _{-\infty} Ae^{-(t' - t_i)^2 / 2\sigma_{dd}^2} dt' = A\left |\sigma_{dd} \right|\sqrt{2\pi},
\end{equation}

\par onde $A$ é a amplitude do ponto com centroide $t_i$ e desvio padrão após a deconvolução $\sigma_{dd}$.

\par Com o banco de dados para a deconvolução e também para a detecção de picos, basta criar as redes neurais, que serão descritas na seção \ref{sec:pulsos_ml}.

%Com esse procedimento podemos então reconstituir os eventos (nuvens de pontos), obtendo, para cada evento, todas as coordenadas $x$, $y$, $t$ e $Q$ de cada ponto. A figura \ref{fig:ex_eventos} mostra exemplos de eventos reconstruídos.

% AQUI FIGURA DAS NUVENS DE PONTO
%\begin{figure}[H]
%\centering
%    \begin{subfigure}[b]{0.48\textwidth}
%        \centering
%        \includegraphics[scale=0.38]{figs/ex_ev_1.png}
%        \caption{}
%        \label{subfig:ex_ev_1}
%    \end{subfigure}%
%    \hfill
%    \begin{subfigure}[b]{0.48\textwidth}
%        \centering
%        \includegraphics[scale=0.38]{figs/ex_ev_2.png}
%        \caption{}
%        \label{subfig:ex_ev_2}
%    \end{subfigure}
%    \begin{subfigure}[b]{0.48\textwidth}
%        \centering
%        \includegraphics[scale=0.38]{figs/ex_ev_3.png}
%        \caption{}
%        \label{subfig:ex_ev_3}
%    \end{subfigure}%
%    \hfill
%    \begin{subfigure}[b]{0.48\textwidth}
%        \centering
%        \includegraphics[scale=0.38]{figs/ex_ev_4.png}
%        \caption{}
%        \label{subfig:ex_ev_4}
%    \end{subfigure}
%\caption{Exemplos de eventos reconstruídos através da análise dos sinais. A seta vermelha indica o sentido do feixe.}
%\label{fig:ex_eventos}
%\end{figure}

\section{Análise dos pulsos com \textit{machine learning}}\label{sec:pulsos_ml}

% \par Da mesma forma que na seção anterior, queremos fazer a reconstrução dos eventos a partir dos histogramas gerados pelos canais do detector. O que precisamos é da posição dos centroides presentes no espectro, porém não é tarefa que pode ser resolvida diretamente, é necessário quebrar em etapas, ou seja, deve-se remover o sinal de fundo, fazer a deconvolução do espectro (já sabendo qual o valor de sigma) para enfim buscar os picos.

% \par Com \textit{machine learning} temos a possibilidade de criar algoritmos extremamente complexos sem definir operações explícitas como na seção anterior. Da solução anterior, para um vetor de tamanho $n$, a complexidade do consumo de tempo é $\mathcal{O}$($n$) para determinar o fundo, $\mathcal{O}$($m\times n$) para $m$ iterações da deconvolução e $\mathcal{O}$($n$) para detecção de picos. Sendo o maior consumo de tempo a deconvolução, podemos construir uma rede neural que faça a deconvolução porém em menor tempo. Devido à eletrônica do detector, o fundo do sinal de cada espectro possui informações relevantes para entender o funcionamento de cada pixel do detector, sua estimativa é importante para determinação de incertezas. Somado ao fato que para fazer a deconvolução é preciso do sinal sem a \textit{baseline}, então a estratégia será criar uma rede neural para nos dar a \textit{baseline} do espectro e uma que faça a deconvolução.

\par Com \textit{machine learning} temos a possibilidade de criar algoritmos extremamente complexos sem definir operações explícitas. Usando os resultados das seções anteriores foram desenvolvidas três redes neurais, com o objetivo de: estimar o fundo (subseção \ref{subsec:pulso_ml_fundo}), fazer a deconvolução (\ref{subsec:pulso_ml_deconv}) e por fim detectar os picos (subseção \ref{subsec:pulso_ml_peaks}). 

%Os resultados dos algoritmos usados nas subseções \ref{subsec:pulses_baseline} e \ref{subsec:pulses_deconv} foram usados como \textit{outputs} para o aprendizado das redes neurais que foram desenvolvidas em cada etapa.

\subsection{Rede neural para o fundo}\label{subsec:pulso_ml_fundo}

\par O objetivo foi criar uma rede neural que reproduza o comportamento do algoritmo \textit{background removal} que estima o fundo do sinal, que foi discutido na seção \ref{subsec:pulses_baseline}, tentando reproduzir resultados muito similares. A rede neural neural é supervisionada, onde os dados de entrada são os sinais brutos e as saídas devem ser os fundos de cada sinal. A arquitetura está na figura \ref{fig:arq_source_to_bkg}.

% \par Estimar o fundo é uma tarefa muito complexa pois a eletrônica do detector faz com que o sinal do canal varie muito dependendo do evento, podendo muitas vezes fazer com que o fundo tenha saltos no espectro após receber o sinal de um pulso. Importante ressaltar que a retirada do fundo não precisa ser considerada perfeita. O próprio algoritmo analítico de remoção de fundo não é perfeito e no geral nunca coloca toda a parte que não é o pulso em 0, há muitas flutuações. O importante é tentar deixar o mais próximo de zero possível. 

% \par Estimar o fundo é uma tarefa muito complexa pois ele não é analítico, podendo muitas vezes fazer com que o fundo tenha saltos em diferentes \textit{time buckets}. 

\begin{figure}[H]
    \centering
    \includegraphics[scale = 0.28]{figs/Source to only bkg.png}
    \caption{Arquitetura da rede neural que faz a inferência do fundo. O vetor de entrada deve ter dimensionalidade 512 x 1. Todas as partes com convolução não possuem o parâmetro \textit{bias}.}
    \label{fig:arq_source_to_bkg}
\end{figure}

\par A entrada da rede é o sinal cru com dimensionalidade 521 x 1. Há duas convoluções seguidas (passagens \textit{a} e \textit{b}) com \textit{padding same}, seguida de uma camada com \textit{Max pooling}. Os dois canais restantes sofrem uma planificação (ou \textit{flat}), diminuindo sua dimensionalidade, para então passar por mais uma convolução com \textit{padding same} e filtros de tamanho 19 seguido de uma camada \textit{Max pooling} e uma camada \textit{Fully Connected} com função de ativação \textit{ReLU}. Toda a rede neural foi construída usando o TensorFlow 2 e possui um total de 545.536 parâmetros, todos treináveis. O tamanho dos filtros das convoluções levam em conta a largura do pulso, sendo no mínimo maior que a largura, a fim de que cada \textit{kernel} atue em um pulso completo na convolução\cite{FORTINO2022166497}.

\par A camada final com função de ativação \textit{ReLU} garante o valor mínimo de saída em 0 e, principalmente, pelo fato de não causar problemas à minimização do gradiente \cite{VGP}. Foram testadas diversas combinações e a mostrada na figura \ref{fig:arq_source_to_bkg} é a que obteve os melhores resultados\cite{FORTINO2022166497}. 
% Caso, por exemplo, na passagem \textit{c} da figura, o \textit{pool size} da camada de \textit{max pooling} fosse 32 (a fim de sobrarem 512 canais igual na entrada e na saída), a rede parece não entender as oscilações grandes do sinal de fundo.

\par Para o treino foram usados 160.000 sinais para treino e 40.000 para validação. O \textit{loss} foi escolhido como sendo o erro quadrático médio (equação \ref{eq:erro_abs_m}, o otimizador foi o \textit{ADAMAX} \cite{ADAMAX}, com \textit{learning rate} de 0.0005, e métrica para avaliação foi o erro médio absoluto, dado por

\begin{equation}\label{eq:erro_abs_m}
    E = \frac{1}{N}\sum_{i = 1}^{N} \left | x_i - \Hat{x}_i\right |,
\end{equation}

onde $E$ é o erro absoluto médio, $N$ é o número de pontos e $x_i$ o ponto da saída da rede para ser comparado com o ponto original $\Hat{x}_i$. Foram 30 \textit{epochs} e o \textit{batch-size} foi 8. O treino foi realizado no Google Colaboratory \cite{google_colab} usando a GPU (\textit{graphics processing unit}) NVIDIA Tesla P100 e durou cerca de 34 minutos. Os resultados do treino estão na figura \ref{fig:source_to_bkg_results}.

% \par Como dito na seção \ref{sec:ml} temos que especificar qual o tamanho do filtro das camadas convolucionais. Para determinar o fundo precisamos, ponto a ponto, determinar quantos canais na direita e na esquerda o filtro deve atuar. Por exemplo, caso o tamanho do filtro fosse 3, ele só estaria ``enxergando" um ponto à direita e um à esquerda, para então passar a informação a diante, porém claramente é um filtro muito pequeno, as oscilações podem variar mais do que 5 canais. Além disso, quando há o começo de um pulso no sinal, ele pode se estender por múltiplos canais, então é preciso escolher um tamanho de filtro grande o suficiente para enxergar todas essas diferenças. A rede consta com duas camadas convolucionais em sequência com filtros de tamanho 21 e 19, respectivamente.

\begin{figure}[H]
\centering
    \begin{subfigure}[t]{0.49\textwidth}
        \centering
        \includegraphics[scale=0.42]{figs/source_to_bkg_loss.png}
        \caption{\textit{Loss} dos dados de treino (linha contínua) e dos dados de validação (linha tracejada) em função da \textit{epoch} no treino da rede dada pela figura \ref{fig:arq_source_to_bkg}.}
        \label{subfig:source_to_bkg_loss}
    \end{subfigure}%
    \hfill
    \begin{subfigure}[t]{0.45\textwidth}
        \centering
        \includegraphics[scale=0.42]{figs/source_to_bkg_metric.png}
        \caption{Erro absoluto médio dos dados de treino (linha contínua) e dos dados de validação (linha tracejada) em função da \textit{epoch} no treino da rede dada pela figura \ref{fig:arq_source_to_bkg}.}
        \label{subfig:source_to_bkg_metric}
    \end{subfigure}
\caption{Resultados do treino da rede neural dada pela figura \ref{fig:arq_source_to_bkg}. A rede atingiu seu melhor resultado a partir da \textit{epoch} 20 aproximadamente, quando começa um 
platô no \textit{loss}.}
\label{fig:source_to_bkg_results}
\end{figure}

% \par A métrica do erro absoluto médio mede ponto a ponto qual o erro absoluto da previsão. 

\par A arquitetura da figura \ref{fig:arq_source_to_bkg} (assim como as próximas desse capítulo) foi determinada de forma empírica. Uma arquitetura com menos passagens e/ou menos parâmetros fornece resultados menos adequados em comparação com a arquitetura apresentada. No caso de mais parâmetros e /ou passagens (consequentemente com aumento no tempo de execução), a rede neural não demonstrou melhora substancial.

\par Exemplos de resultados de previsões da rede neural estão na figura \ref{fig:stb_examples}. A previsão do fundo possui um erro absoluto nos dados de treino de 6.5315 ADC Channels e nos dados de validação de 6.0783 ADC Channels. O sinal cru é subtraído do fundo, colocando o valor mínimo da subtração em 0. Comparando o erro médio absoluto de 200.000 sinais sem o respectivo fundo (resultante do algoritmo do TSpectrum) com o resultado da rede neural obtemos 4.5 ADC Channels.
% Esse erro significa que, por exemplo, para o menor pico detectado considerado, que possui amplitude de cerca de 60 unidades em y, a incerteza estimada da amplitude seria de 7.5\%, ou seja, na menor amplitude possível para um pico a incerteza propagada seria de 7.5\% da amplitude.

\par Rede neurais convolucionais têm a vantagem de usarem poucas variáveis e serem facilmente paralelizadas em sua execução\cite{mlbook}. Uma vantagem de redes neurais é o seu tempo de execução. Empiricamente a rede neural pode processar 200.000 sinais em apenas 8s (ou 25.000 sinais por segundo), sendo extremamente eficiente em tempo.

\begin{figure}[H]
\centering
    \begin{subfigure}[b]{0.49\textwidth}
        \centering
        \includegraphics[scale=0.43]{figs/stb_1.png}
        \caption{}
        \label{subfig:stb_ex1}
    \end{subfigure}%
    \hfill
    \begin{subfigure}[b]{0.46\textwidth}
        \centering
        \includegraphics[scale=0.43]{figs/stb_2.png}
        \caption{}
        \label{subfig:stb_ex2}
    \end{subfigure}
    \begin{subfigure}[b]{0.49\textwidth}
        \centering
        \includegraphics[scale=0.43]{figs/stb_3.png}
        \caption{}
        \label{subfig:stb_ex3}
    \end{subfigure}%
    \hfill
    \begin{subfigure}[b]{0.46\textwidth}
        \centering
        \includegraphics[scale=0.43]{figs/stb_4.png}
        \caption{}
        \label{subfig:stb_ex4}
    \end{subfigure}
\caption{Exemplos da rede neural dada pela figura \ref{fig:arq_source_to_bkg} em comparação com a saída do \textit{TSpectrum}.}
\label{fig:stb_examples}
\end{figure}

\par Nos exemplos mostrados nas figuras \ref{subfig:stb_ex1}, \ref{subfig:stb_ex2} e \ref{subfig:stb_ex1} os fundos dos sinais possuem grande flutuação e a rede neural se mostrou eficaz na previsão. No exemplo \ref{subfig:stb_ex4} a \textit{baseline} do sinal é extremamente complexa de se determinar pois o sinal, aproximadamente do canal 300 a 500, varia em cerca de 50 unidades em \textit{y}. Apesar da rede neural determinar o fundo acima do fundo original, ao subtrair o espectro do fundo e colocar o valor mínimo em 0, o pulso presente entre os canais 200 e 300 é praticamente inalterado.

\par Com os resultados obtidos pela rede neural que calcula o sinal de fundo, o próximo passo foi criar a rede neural que faz a deconvolução do espectro sem a \textit{baseline}.

\subsection{Rede neural para a deconvolução}\label{subsec:pulso_ml_deconv}

% \par O próximo passo é construir uma rede neural que dê a posição dos picos. A tarefa a princípio não parece complicada. Pensando de forma simples, podemos fazer uma arquitetura curta e, como precisamos de uma saída de tamanho fixo, classificamos ponto a ponto como sendo não pico ou pico.

% \par O problema nessa abordagem é o desbalanço de classe evidente nos sinais. Caso tenhamos que classificar ponto a ponto em que, por exemplo, 0 representa um ponto que não é o centroide e 1 como um ponto que é um centroide, colocando na ultima camada a função de ativação \textit{sigmoid} (para sair valores entre 0 e 1), há muito mais valores 0 que 1. Se, por exemplo, temos um sinal que possui apenas um pico, teríamos que acertar o único valor 1 dentre 511 zeros. Caso a rede assuma que são 512 zeros, ainda assim a acurácia binária seria de mais de 99\%.

% \par Outro problema é que há um \textit{overlap} de gaussianas. Não é tão evidente a posição dos centroides se o espectro não está deconvoluído. É muito mais complicado, mesmo olhando, dizer a posição. Portanto o problema será quebrado em mais uma etapa: construir um rede que faça a deconvolução.

\par A mesma abordagem da rede neural anterior foi usada, que é fazer uma sequencia de convoluções e por fim uma camada \textit{fully connected} com função de ativação \textit{ReLU}, pois precisamos ter o valor mínimo do espectro em 0. Os filtros das convoluções precisam ter tamanho mínimo de duas vezes o sigma das gaussianas para atuarem sobre cada pulso do espectro. A entrada da rede é o sinal com o fundo subtraído e com mínimo em 0. A saída é o sinal após deconvolução dada pelo algoritmo \textit{gold deconvolution} na biblioteca \textit{TSpectrum} do ROOT, já mostrado na subseção \ref{subsec:pulses_deconv}. A figura \ref{fig:source_to_deconv} mostra a arquitetura da rede de deconvolução.

\par A rede é a sequência de duas convoluções com 32 filtros, \textit{valid padding} e \textit{kernels} de tamanho 19 e 17, respectivamente, seguida de uma camada \textit{Max pooling} com \textit{pool size} igual à 16. No final há o \textit{flat} na camada para seguir com uma camada \textit{fully connected} com função de ativação \textit{ReLU}. O \textit{valid padding} se mostrou mais eficiente para a convergência da rede. Toda a rede foi construída usando o TensorFlow 2, possuindo 508.000 parâmetros treináveis\cite{FORTINO2022166497}.

\par Assim como na rede anterior, foram usados 160.000 sinais para treino e 40.000 para validação. O \textit{loss} foi escolhido como sendo o erro quadrático médio, o otimizador foi o \textit{ADAM}, com \textit{learning rate} de 0.0005 porém com o parâmetro \textit{clipnorm} igual a 0.45. A métrica para avaliação foi o erro médio absoluto. Foram 75 \textit{epochs} e o \textit{batch-size} foi 8. Os resultados do treino estão na figura \ref{fig:source_wo_bkg_to_deconv_results}.

\par Alterar a norma do gradiente (usar o parâmetro \textit{clipnorm} = 0.45) significa que, caso a norma do vetor do gradiente exceda 0.45, então o valor da norma é reajustado para o limiar (\textit{threshold}) escolhido (0.45)\cite{FORTINO2022166497}. Isso faz com que não ocorra problemas comuns como o gradiente sumir\cite{VGP, ADAMAX}, o que estava acontecendo especificamente nesse caso.

\par O treino foi realizado no Google Colaboratory \cite{google_colab} usando a GPU NVIDIA Tesla P100 e demorou cerca de 54 minutos. Os resultados do treino estão na figura \ref{fig:source_to_bkg_results}, onde eles indicam que, visualmente, a rede neural consegue distinguir muito bem diferentes centroides presentes no pulso. Empiricamente, a rede é capaz de executar 200.000 sinais em 5.4 segundos (ou 37.000 sinais por segundo).

\begin{figure}[H]
    \centering
    \includegraphics[scale = 0.28]{figs/source_wobkg_to_deconv.png}
    \caption{Arquitetura da rede neural que faz a inferência da deconvolução do espectro. O vetor de entrada deve ter dimensionalidade 512 x 1. Todas as partes com convolução não possuem o parâmetro \textit{bias}.}
    \label{fig:source_to_deconv}
\end{figure}

\begin{figure}[H]
\centering
    \begin{subfigure}[t]{0.49\textwidth}
        \centering
        \includegraphics[scale=0.42]{figs/source_wo_bkg_to_deconv_loss.png}
        \caption{\textit{Loss} dos dados de treino (linha contínua) e dos dados de validação (linha tracejada) em função da \textit{epoch} no treino da rede dada pela figura \ref{fig:source_to_deconv}.}
        \label{subfig:source_wo_bkg_to_deconv_loss}
    \end{subfigure}%
    \hfill
    \begin{subfigure}[t]{0.465\textwidth}
        \centering
        \includegraphics[scale=0.42]{figs/source_wo_bkg_to_deconv_metric.png}
        \caption{Erro absoluto médio dos dados de treino (linha contínua) e dos dados de validação (linha tracejada) em função da \textit{epoch} no treino da rede dada pela figura \ref{fig:source_to_deconv}.}
        \label{subfig:source_wo_bkg_to_deconv_metric}
    \end{subfigure}
\caption{Resultados do treino da rede neural dada pela figura \ref{fig:arq_source_to_bkg}.}
\label{fig:source_wo_bkg_to_deconv_results}
\end{figure}

% \par A mudança em relação à rede que faz a inferência do sinal de fundo é que o \textit{padding} das camadas agora é \textit{valid}. Agora não estamos verificando pontos onde seriam necessários adicionar zeros além do vetor de entrada, como explicado na seção \ref{sec:ml}. O \textit{padding} \textit{valid} se mostrou mais eficiente com relação à separação das gaussianas, dando melhor resolução para buscar os centroides. Além disso, ao colocar o \textit{pool size} de apenas 16, para poder dar um \textit{flat} na camada seguinte, fez com que a resolução de separação das gaussianas fosse inclusive melhor que a do algoritmo analítico. A figura \ref{fig:std_examples} mostra um exemplo da saída da rede.

\begin{figure}[H]
\centering
    \begin{subfigure}[b]{0.49\textwidth}
        \centering
        \includegraphics[scale=0.425]{figs/swbtd_1.png}
        \caption{}
        \label{subfig:std_ex1}
    \end{subfigure}%
    \hfill
    \begin{subfigure}[b]{0.465\textwidth}
        \centering
        \includegraphics[scale=0.425]{figs/swbtd_2.png}
        \caption{}
        \label{subfig:std_ex2}
    \end{subfigure}
    \begin{subfigure}[b]{0.49\textwidth}
        \centering
        \includegraphics[scale=0.425]{figs/swbtd_3.png}
        \caption{}
        \label{subfig:std_ex3}
    \end{subfigure}%
    \hfill
    \begin{subfigure}[b]{0.465\textwidth}
        \centering
        \includegraphics[scale=0.425]{figs/swbtd_4.png}
        \caption{}
        \label{subfig:std_ex4}
    \end{subfigure}
\caption{Exemplos de deconvolução da rede neural dada pela figura \ref{fig:arq_source_to_bkg}.}
\label{fig:std_examples}
\end{figure}

\par Com a rede neural para a deconvolução feita, a última etapa é a construção de uma rede neural para a detecção de picos, mostrada na \ref{subsec:pulso_ml_peaks}.

\subsection{Detecção de picos}\label{subsec:pulso_ml_peaks}

\par A última etapa dessa análise com \textit{machine learning} foi analisar possíveis soluções para a detecção de picos. Esse é um problema muito complexo, pois há uma variação muito grande na quantidade de picos por sinal e também há um desbalanço muito grande na quantidade de pontos comuns (aqueles que não são picos) e pontos que são picos. Por exemplo, caso haja um sinal que possui apenas um pico, devemos detectar uma posição, dar o valor de saída como 1, por exemplo, dentre 512 pontos, onde 511 terão o valor de saída como 0. Caso a rede determine que todos os pontos são não picos, ainda assim a acurácia binária seria maior que 99\%. Isso é conhecido como desbalanço de classe\cite{inproceedings}.

\par Para corrigir esse desbalanço, foram acrescentados pontos simetricamente em torno do pulso, de forma que a somatória das amplitudes das regiões é equivalente a carga $Q$ acumulada no ponto. Isso faz com que não seja preciso detectar um único ponto, mas sim uma região em torno do pico. Com isso podemos nos basear na ideia de segmentar o sinal para destacar regiões de interesse\cite{aly2011research}. Segmentar significa ter uma rede neural com a saída com o mesmo tamanho do vetor de entrada (512) e saída com valores entre 0 e 1, onde 1 indica uma região com um pico e 0 não. A figura \ref{fig:n_peaks_exs} mostra um exemplo de um sinal após a deconvolução onde há os picos detectados com o algoritmo \textit{peak\_finder} e os pontos acrescentados simetricamente em torno dos picos para representar as regiões dos pulsos.

% \par Para resolver esse problema podemos nos basear na ideia de recortar regiões de interesse, como feito pela rede U-Net\cite{unet}. Recortar regiões significa, nesse caso, ter uma rede neural com a saída com o mesmo tamanho do vetor de entrada (512) e saída com valores entre 0 e 1, onde 1 indica uma região com um pico e 0 não. Como temos as posições dos picos, basta acrescentar pontos simetricamente em torno do pulso. A figura \ref{fig:n_peaks_exs} mostra um exemplo de um sinal após a deconvolução onde há o pico detectado e os pontos acrescentados para representar a região do pulso.

\begin{figure}[H]
    \centering
    \includegraphics[scale = 0.6]{figs/np_ex1.png}
    \caption{Sinal após a deconvolução que mostra o pico detectado mais os pontos adicionais que irão facilitar o trabalho da rede neural (evitar o desbalanço de classe). Foram acrescentados 2 pontos à esquerda e à direita.}
    \label{fig:n_peaks_exs}
\end{figure}

\par A rede construída é a sequência de uma convolução com \textit{kernel} de tamanho 13 e \textit{same padding} seguida de \textit{Max-Pooling} e uma \textit{fully-connected} com função de ativação sigmoide, possuindo um total de 263.104 parâmetros treináveis. Para o treino foram usados sinais que possuíam entre 1 e 6 picos, resultantes da saída do algoritmo \textit{peak\_finder} do \textit{SciPy}, o que resultou em 120.024 de dados para o treino e 30.006 para validação. A escolha pelos picos detectados pelo \textit{peak\_finder} ao invés do algoritmo de detecção do TSpectrum é pela maior flexibilidade de ajuste fino do algoritmo, tornando a detecção de picos muito melhor\cite{FORTINO2022166497}. A função custo escolhida foi a \textit{binary cross-entropy} (dada pela equação \ref{eq:binary_cross_entropy}), o otimizador o \textit{ADAM} com \textit{learning rate} de 0.001. A métrica utilizada foi a acurácia binária. O treino também foi realizado por uma GPU NVIDIA Tesla P100 e durou cerca de 8 minutos com 12 \textit{epochs}. A arquitetura da rede está na figura \ref{fig:arq:n_peaks} e os resultados do treino estão na figura \ref{fig:n_peaks_results}.

\begin{figure}[H]
    \centering
    \includegraphics[scale = 0.35]{figs/n_peaks.png}
    \caption{Arquitetura da rede neural que faz o recorte das regiões com picos. O vetor de entrada deve ter dimensionalidade 512 x 1.}
    \label{fig:arq:n_peaks}
\end{figure}

\begin{figure}[H]
\centering
    \begin{subfigure}[t]{0.49\textwidth}
        \centering
        \includegraphics[scale=0.42]{figs/n_peaks_loss.png}
        \caption{\textit{Loss} dos dados de treino (linha contínua) e dos dados de validação (linha tracejada) em função da \textit{epoch}.}
        \label{subfig:n_peaks_loss}
    \end{subfigure}%
    \hfill
    \begin{subfigure}[t]{0.46\textwidth}
        \centering
        \includegraphics[scale=0.42]{figs/n_peaks_metric.png}
        \caption{Acurácia binária dos dados de treino (linha contínua) e dos dados de validação (linha tracejada) em função da \textit{epoch}.}
        \label{subfig:n_peaks_metric}
    \end{subfigure}
\caption{Resultados do treino da rede neural dada pela figura \ref{fig:arq:n_peaks}.}
\label{fig:n_peaks_results}
\end{figure}

\par A saída da rede neural é um vetor de tamanho 512 com valores entre 0 e 1, onde valores maiores que 0.5 são considerados como pertencentes à um pulso. Com as regiões identificadas podemos fazer uma média ponderada com o espectro de entrada para achar o centroide. O tempo de processamento da rede neural é de 150.030 sinais em cerca de 4.11 segundos (aproximadamente 36.500 sinais por segundo). Para determinar os picos a partir da saída da rede neural, o tempo é de aproximadamente 4.3 segundos, onde o algoritmo pode ser ainda mais rápido se for paralelizado. Resultados para picos detectados pela rede neural em comparação com o algoritmo \textit{peak\_finder} estão na figura \ref{fig:exs_n_peaks}.

\begin{figure}[H]
\centering
    \begin{subfigure}[b]{0.49\textwidth}
        \centering
        \includegraphics[scale=0.425]{figs/np_exs1.png}
        \caption{}
        \label{subfig:exs_n_peaks_1}
    \end{subfigure}%
    \hfill
    \begin{subfigure}[b]{0.465\textwidth}
        \centering
        \includegraphics[scale=0.425]{figs/np_exs2.png}
        \caption{}
        \label{subfig:exs_n_peaks_2}
    \end{subfigure}
    \begin{subfigure}[b]{0.49\textwidth}
        \centering
        \includegraphics[scale=0.425]{figs/np_exs3.png}
        \caption{}
        \label{subfig:exs_n_peaks_3}
    \end{subfigure}%
    \hfill
    \begin{subfigure}[b]{0.465\textwidth}
        \centering
        \includegraphics[scale=0.425]{figs/np_exs4.png}
        \caption{}
        \label{subfig:exs_n_peaks_4}
    \end{subfigure}
\caption{Exemplos de detecção de picos usando a rede neural, em comparação com a detecção feita pelo algoritmo presente no SciPy, mostrada na figura \ref{fig:arq:n_peaks}. Os centroides detectados pela rede neural estão em vermelho, e os centroides detectados pelo SciPy estão em verde. Em azul está o espectro sem fundo após a deconvolução, resultantes }
\label{fig:exs_n_peaks}
\end{figure}

% \par A acurácia binária acima de 98\% não é uma métrica muito adequada para identificar a qualidade da rede neural discutida. Para isso, foi feita a comparação com a quantidade de picos detectados pela rede neural, em comparação com o número de picos detectos pelo algoritmo \textit{peak\_finder}, limitando a detecção entre 0 e 6 picos detectados pelo SciPy (mais que isso é indicativo de um sinal com muito ruído). A comparação está na figura \ref{fig:hist_2d_peaks_model_vs_scipy}.  


%\begin{figure}[H]
%    \centering
%    \includegraphics[scale = 0.48]{figs/histograma_2d_picos_model_vs_scipy.png}
%    \caption{Histograma bidimensional que mostra em $x$ a contagem do número de picos detectados por sinal pelo algoritmo \textit{peak\_finder} presente no SciPy, e em $y$ o número de picos detectados por sinal pela rede neural mostrada na figura \ref{fig:arq:n_peaks}. O número de contagens está marcado em cima de cada \textit{bin}.}
%    \label{fig:hist_2d_peaks_model_vs_scipy}
%\end{figure}
%\par Unificando a rede neural da figura \ref{fig:arq:source_to_deconv} com a rede de segmentação da figura \ref{fig:arq:n_peaks} o tempo para processar 200 mil sinais e depois identificar os picos é de aproximadamente 16.5 segundos, usando a GPU Tesla P100 e o processador (para processar a segmentação) Ryzen 5 3600X. Há um pequeno ganho de tempo em relação ao algoritmo \textit{peak\_finder}. 

%\par A rede neural detecta em média menos picos em comparação ao algoritmo \textit{peak\_finder}, isso fica mais evidente quanto maior o número de picos detectados pelo algoritmo presente no SciPy. O número de detecções de até 4 picos é muito semelhante, isso é mais evidente para os casos em que há um único pico. Esse limite de deteção é uma possível limitação da rede neural.

\par Para determinar carga acumulada $Q_i$ de cada ponto associado à cada centroide $i$, foi usada a relação

\begin{equation}\label{eq:carga_acumulada_ml}
	Q_i = \sum_{i - 2}^{i + 2}f(t_i)\alpha_\sigma,
\end{equation}

\par onde $f(t_i)$ é a amplitude do espectro no \textit{time bucket} $t$ da posição $i$ e $\alpha_\sigma$ é uma constante real para calibrar o valor da área em função do sigma dos pulsos, que foi determinado empiricamente como $\alpha_\sigma$ = 1.2.

\par Com as três redes neurais criadas, é necessário verificar o acoplamento das redes, a fim de analisar a qualidade dos algoritmos e verificar os tempos de execução, para comparar com os algoritmos discutidos nas subseções \ref{subsec:pulses_baseline} e \ref{subsec:pulses_deconv}.


\subsection{Acoplando as redes neurais}

%\par Com a rede que prevê o fundo e a que faz a deconvolução funcionando podemos agora testar se acopladas elas geram resultados satisfatórios, verificando a qualidade da detecção de picos.
\par Usando novamente o TensorFlow 2 pode-se carregar as redes neurais discutidas nas subseções \ref{subsec:pulso_ml_fundo}, \ref{subsec:pulso_ml_deconv} e \ref{subsec:pulso_ml_peaks}, já treinadas e usar como se fosse uma única rede neural. Acoplando as três arquiteturas, temos a nova arquitetura mostrada na figura \ref{fig:arq:source_to_segmentation}. O \textit{input} da rede é o sinal cru o \textit{output} da rede neural é o um vetor de tamanho 1024, que possui a segmentação e o sinal após a deconvolução, pois as duas informações são necessárias para determinar os centroides.

% Primeiro, acoplando as redes neurais dadas pelas figuras \ref{fig:arq_source_to_bkg} e \ref{fig:source_to_deconv} para formar a arquitetura dada pela figura \ref{fig:arq:source_to_deconv}, para investigar melhor os resultados com algoritmo \textit{peak\_finder}.

\begin{figure}[H]
    \centering
    \includegraphics[scale = 0.35]{figs/arq_source_segmentation.png}
    \caption{Arquitetura da rede neural que faz a inferência da \textit{baseline}, em seguida faz a deconvolução do espectro sem o fundo e por fim faz a segmentação do sinal. O resultado da segmentação e da deconvolução são concatenados na parte final da rede neural. O vetor de entrada deve ter dimensionalidade 512 x 1.}
    \label{fig:arq:source_to_segmentation}
\end{figure}

%\begin{figure}[H]
%    \centering
%    \includegraphics[scale = 0.28]{figs/source_to_deconv.png}
%    \caption{Arquitetura da rede neural que faz a inferência da \textit{baseline} e depois faz a deconvolução do espectro. O vetor de entrada deve ter dimensionalidade 512 x 1.}
%    \label{fig:arq:source_to_deconv}
%\end{figure}

\par A rede é apenas a sequencia das redes anteriores, ou seja, o espectro passa pelo cálculo da \textit{baseline}, então o espectro original é subtraído dessa \textit{baseline} (colocando o valor mínimo em 0) para passar pela etapa da deconvolução e em seguida o sinal é segmentado. Na etapa final, o resultado da segmentação e da deconvolução são concatenados em um único vetor, para assim determinar os centroides. Pelo fato de cada parte ser treinada de modo separado não há necessidade de treinar a rede unificada, apenas carregar as variáveis das redes neurais treinadas de cada parte. A rede unificada possui 1.316.640 de parâmetros.

\par Foi comparada a saída da rede acoplada para o espectro após a deconvolução, com a saída de referência, que é o espectro após a deconvolução mostrado na subseção \ref{subsec:pulses_deconv}. Podemos usar novamente o erro médio absoluto para fins de comparação. O erro médio absoluto, para os 200.000 sinais, é de 7.45 ADC Channels. Com relação à incerteza, ela foi estimada como o desvio padrão da diferença de cada \textit{time bucket} do sinal tido como referência e o sinal após a rede neural acoplada. A incerteza é da ordem de 6\% da amplitude do sinal no ponto.

% A detecção será feita com o \textit{SciPy} que possui rotinas para a detecção. A função usada é a ``find\_peaks", onde é possível determinar altura mínima de detecção, distância mínima entre picos, dentre outros parâmetros, o que torna possível ajustar melhor a detecção em comparação com a saída pelo algoritmo presente no \textit{TSpectrum}.
% O objetivo será fazer a comparação entre os picos resultantes do algoritmo de deconvolução (na biblioteca \textit{TSpectrum} do ROOT) e os resultantes do algoritmo do \textit{SciPy}.

% \par A comparação se limitou a detecção de 0 a 6 picos detectados pela deconvolução, que é o intervalo que faz sentido físico (mais que 6 picos detectados indica um sinal muito ruidoso). Caso o \textit{SciPy} detecte mais que 6, então o espectro é descartado. Primeiro deve-se ver a comparação do número de espectros detectado por evento de cada algoritmo. O histograma da figura \ref{fig:hist_2d_peaks} mostra essa comparação.

% \par O histograma da figura \ref{fig:hist_2d_peaks} mostra que, apesar da variação no número de picos detectados pelo \textit{SciPy} (eixo y) em comparação ao número de picos detectados pelo \textit{TSpectrum}, na maioria dos casos o número de detecções é o mesmo, especialmente para os casos em que há um único pico.

%\begin{figure}[H]
%    \centering
%    \includegraphics[scale = 0.48]{figs/hitograma_2d_peaks.png}
%    \caption{Histograma bidimensional que mostra em $x$ a contagem do número de picos detectados por sinal (dado pelo \textit{TSpectrum}) pela deconvolução e em $y$ a contagem do número de picos detectados por sinal (resultante da rede neural) pelo \textit{scipy}. O número de contagens está marcado em cima de cada \textit{bin}.}
%    \label{fig:hist_2d_peaks}
%\end{figure}

% \par A linha tracejada serve de referência para os caso em que o número de picos detectador por ambos algoritmos foi igual. A dispersão de detecção se dá em torno dessa e se mostrou com uma diferença muita baixa, indicando que o processamento de sinal pela rede neural seguido pela detecção de picos com o \textit{scipy} é equivalente ao dado pelo uso dos algoritmos do \textit{TSpectrum}. Para os casos em que o número detectado de picos foi o mesmo, a diferença da posição dos centroide é de 0.11 canais, mostrando que os métodos são praticamente equivalentes.
%  O valor não é a soma direta dos tempos das redes anteriores pois em Python existe a questão da vetorização de operações
\par Com relação a eficiência em tempo, a rede neural da figura \ref{fig:arq:source_to_segmentation} processa 200.000 sinais em 12 segundos, usando a GPU NVIDIA Tesla P100. Somando esse tempo com a determinação dos centroides, que é de aproximadamente 4.3 segundos, o tempo total para processar 200 mil sinais é de cerca de 16.3 segundos, cerca de 90 vezes mais rápido que os métodos mostrados na seção \ref{sec:pulses_trad}. Exemplos da reconstrução das nuvens de pontos usando a rede neural da figura \ref{fig:arq:source_to_segmentation} estão na figura \ref{subfig:ex_ev_4}.

% a detecção de picos com o \textit{SciPy}, o tempo total para processar 200 mil sinais é de cerca de 25 s. Nem foi preciso uma rede neural para os picos e o consumo de tempo já é 60 vezes maior em comparação aos métodos tradicionais mostrados anteriormente.

\begin{figure}[H]
\centering
    \begin{subfigure}[b]{0.48\textwidth}
        \centering
        \includegraphics[scale=0.38]{figs/ex_ev_1.png}
        \caption{}
        \label{subfig:ex_ev_1}
    \end{subfigure}%
    \hfill
    \begin{subfigure}[b]{0.48\textwidth}
        \centering
        \includegraphics[scale=0.38]{figs/ex_ev_2.png}
        \caption{}
        \label{subfig:ex_ev_2}
    \end{subfigure}
    \begin{subfigure}[b]{0.48\textwidth}
        \centering
        \includegraphics[scale=0.38]{figs/ex_ev_3.png}
        \caption{}
        \label{subfig:ex_ev_3}
    \end{subfigure}%
    \hfill
    \begin{subfigure}[b]{0.48\textwidth}
        \centering
        \includegraphics[scale=0.38]{figs/ex_ev_4.png}
        \caption{}
        \label{subfig:ex_ev_4}
    \end{subfigure}
\caption{Exemplos de eventos reconstruídos através da análise dos sinais com \textit{machine learning}. A seta vermelha indica o sentido do feixe.}
\label{fig:ex_eventos}
\end{figure}



% \textit{machine learning} com detecção de picos pelo \textit{SciPy} estão na figura \ref{fig:ml_pc_exs}. É possível perceber uma pequena melhora no ruído dos eventos, pois o com o \textit{SciPy} podemos calibrar melhor a detecção. Além disso há uma correlação de amplitude do pico detectado entre espectro sem o fundo e o espectro sem o fundo após a deconvolução, como mostrado na figura \ref{fig:corr_amps}. Colocando a condição de que, para cada pico detectado, a razão entre a amplitude do espectro após a deconvolução e do espectro original para o ponto seja maior que 3.8 faz com que a quantidade de pontos ruidosos por evento caia em cerca de 35\% em relação aos métodos anteriores.

%\begin{figure}[H]
%\centering
%    \begin{subfigure}[b]{\textwidth}
%        \centering
%        \includegraphics[scale=0.4]{figs/ml_pc_ex1.png}
%        \caption{}
%        \label{subfig:ml_pc_ex1}
%    \end{subfigure}%
%    \vspace{0.2cm}
%    % \hfill
%    \begin{subfigure}[b]{\textwidth}
%        \centering
%        \includegraphics[scale=0.4]{figs/ml_pc_ex2.png}
%        \caption{}
%        \label{subfig:ml_pc_ex2}
%    \end{subfigure}
%\caption{Resultados da reconstrução de eventos por \textit{machine learning}.}
%\label{fig:ml_pc_exs}
%\end{figure}

%\begin{figure}[H]
%    \centering
%    \includegraphics[scale = 0.85]{figs/corr_pulso_deconv.png}
%    \caption{Histograma bidimensional que mostra a relação, de cada pico detectado, entre a amplitude do pico após a deconvolução no eixo $y$ e antes da deconvolução no eixo $x$. A linha tracejada indica a tendência da maior parte dos pontos. Já a linha sólida indica a região de corte dos pontos.}
%    \label{fig:corr_amps}
%\end{figure}

% \subsection{}


% \par Podemos usar o \textit{TensorFlow} que possui diversas ferramentas para construir a arquitetura de modo à simplesmente fazer a conta sequencialmente, ou seja, calcular o fundo, subtrair o fundo do espectro original e fazer para a deconvolução. Os valores de cada variável de cada arquitetura está salvo e basta carregar cada rede feita individualmente e colocar os pesos (valores) nas camadas corretas. A arquitetura unificada está mostrada na figura [xx].

% \par O consumo de tempo para a análise usando o \textit{TSpectrum} é de cerca de 10 minutos para 200 mil sinais. Boa parte desse tempo se dá pela deconvolução, que é um algoritmo que cresce quanto mais iterações são necessárias (foram usadas 300 iterações). A remoção do fundo e a detecção de picos são algoritmos muito mais rápidos, portanto só a deconvolução por redes neurais já pode economizar mais de 90\% do tempo. A rede processa 200 mil arrays em apenas 11 segundos. Podemos usar na sequencia um algoritmo de detecção de picos simples como o presente na biblioteca do \textit{scipy}. O tempo de execução desses arrays mais a detecção de picos (ou seja, a rede neural remove o fundo e faz a deconvolução, enquanto a detecção de picos é por algoritmos analíticos) é de apenas 24 segundos, mesmo sendo executado em \textit{Python}. O uso de redes neurais facilita muito a análise de quantidades massivas de dados, podendo inclusive permitir a análise em tempo real de um experimento.



\par Com as redes neurais desenvolvidas, o consumo de tempo para o processamento dos pulsos diminuiu entre cerca de 90 vezes em relação à algoritmos mais tradicionais mostrados na seção \ref{sec:pulses_trad}, abrindo possibilidades para a análise em tempo real de um experimento. Com as nuvens de pontos reconstruídas, pode-se extrair propriedades físicas dos eventos, processo que será descrito no capítulo \ref{chapter:point_cloud_analysis}.

\chapter{Análise das nuvens de pontos}\label{chapter:point_cloud_analysis}

\par Esse capítulo irá mostrar como foram analisadas as nuvens de pontos com algoritmos de \textit{machine learning}. O objetivo desta etapa, do ponto de vista computacional, foi de detectar \textit{clusters}, que nesse caso são retas tridimensionais. Com a informação da física das trajetórias (energia $E$, momento $\vec{p}$ e  comprimento $L$), foi possível distinguir cada reta detectada, e determinar o vértice de reação (entre uma partícula originada da reação nuclear entre o feixe e o gás). Após a detecção, foram selecionados os eventos que possuíam reações nucleares.

\par Na seção \ref{sec:forcabruta} será mostrado como foram identificadas as trajetórias (também chamado de \textit{tracking}) a partir de estimadores robustos. Na seção \ref{sec:ml_pc}, será mostrado como a mesma identificação foi feita a partir de algoritmos de \textit{machine learning}, identificando as diferenças em relação ao método anterior.


\section{Método com estimadores robustos}\label{sec:forcabruta}

\par Essa seção irá descrever o processo de \textit{tracking} usando estimadores robustos. As nuvens de pontos (eventos) que foram analisados são como o da figura \ref{subfig:exemplo_1}. Nessa figura é notável ao olho humano que os pontos formam estruturas de retas. Essas retas são identificadas com estimadores robustos\cite{artigo}, resultando na figura \ref{subfig:antes_clustering}, e seu resultado foi corrigido resultando na figura \ref{subfig:depois_clustering}. 

\begin{figure}[H]
\centering
    \begin{subfigure}[b]{\textwidth}
        \centering
        \includegraphics[scale = 0.4]{figs/Figure_1.png}
        \caption{Exemplo de evento analisado. Os pontos em azul são das partículas detectadas pelo TPC, a seta vermelha indicando o sentido do feixe e o TPC está representado pelo cilindro cinza.}
        \label{subfig:exemplo_1}
    \end{subfigure}%
    \hfill
    \begin{subfigure}[t]{0.45\textwidth}
        \centering
        \includegraphics[scale=0.25, width=.95\columnwidth]{figs/Figure_8.png}
        \caption{Evento sem a clusterização. As retas de cor amarela e verde são de um único \textit{cluster}.}
        \label{subfig:antes_clustering}
    \end{subfigure}%
    \hspace{0.5cm}
    \begin{subfigure}[t]{0.45\textwidth}
        \centering
        \includegraphics[scale=0.25, width=.95\columnwidth]{figs/fig_cluster.png}
        \caption{Evento após a clusterização. Agora o evento possui as duas retas corretas, a azul e a verde.}
        \label{subfig:depois_clustering}
    \end{subfigure}
\caption{Sequência de análise de um evento. Em \ref{subfig:exemplo_1} temos o evento que é recebido para ser analisado, em \ref{subfig:antes_clustering} temos o mesmo evento após o RANSAC (antes da clusterização) e \ref{subfig:depois_clustering} mostra depois da correção. As cores das retas são arbitrárias e servem apenas para a diferenciação.}
\label{fig:3d_examples}
\end{figure}


% \begin{figure}[H]
%     \centering
%     \includegraphics[scale = 0.6]{figs/Figure_1.png}
%     \caption{Exemplo de evento a ser analisado. Os pontos em azul são das partículas detectadas pelo TPC, a seta vermelha indicando o sentido do feixe e o TPC está representado pelo cilindro cinza.}
%     \label{fig:exemplo_1}
% \end{figure}
% \newpage
% \par A etapa anterior nos fornece arquivos no formato \textit{ROOT} para cada \textit{run} do experimento. O arquivo é dividido por eventos em que estão contidas informações, em \textit{trees}, como coordenadas x, y, z, t, carga, tempo de voo, além de canais adicionais (indicadores de partículas presentes além da parte central do \textit{micromegas}). Os arquivos foram lidos em Python usando a biblioteca Uproot\cite{uproot}.

% \par A figura \ref{subfig:exemplo_1} nos mostra um exemplo que parece indicar mais de uma trajetória dentro do TPC, e é preciso separar os pontos de cada trajetória identificada. O objetivo é separar os pontos de \textit{clusters} que contenham a informação do momento da partícula, obtida a partir das propriedades geométricas do versor das retas e da energia inicial da trajetória. A identificação pode ser feita sem o uso de algoritmos de machine learning, que chamarei de método usual, e usando machine learning.

\par O algoritmo de \textit{tracking} escolhido deve ser capaz de identificar estruturas geométricas especificas (que no caso são retas tridimensionais) mesmo na presença de ruído, que são pontos que não pertencem à nenhum conjunto de pontos de uma reta detectada (também chamados de \textit{outliers}). Para tanto, foram testados estimadores robustos que são variações do RANSAC (\textit{RANdom SAmple Consensus})\cite{ransac, artigo}.

\par Os algoritmos testados foram o RANSAC, MLESAC\cite{ref:MLESAC}, RRANSAC, PROSAC, dentre outros, que estão presentes na biblioteca PCL (\textit{Point Cloud Library})\cite{pcl}. Os algoritmos com os melhores resultados (melhor capacidade de \textit{tracking}) foram o RANSAC e o PROSAC, sem grandes diferenças nos resultados entre os dois. Para melhor ainda mais o resultado, foi feita uma pequena modificação no RANSAC, a fim de obter melhores resultados nos dados do pAT-TPC. O algoritmo \ref{ransac_algo} (que chamei de p-RANSAC - \textit{prototype}-RANSAC) mostra o funcionamento. O algoritmo seleciona dois pontos de modo aleatório (\textit{Random Sampling}) e determina o versor $\hat{v}$ e o ponto $P_b$ que descrevem a única reta $r$ que passa pelos dois pontos. A reta é selecionada caso tenha um número mínimo $tam_{min}$ de pontos que pertencem à reta (chamado de \textit{inliers}) e tenha o mínimo (com relação aos outros conjunto de pontos) da estimativa $C$ dada por

\begin{equation} \label{criterio_ransac}
    C = \sum_{i = 0}^{N} \frac{d_i ^2}{N},
\end{equation}

onde \textit{N} é o número total de pontos de uma reta e $d_i$ é a distância do i-ésimo ponto à reta.

% \par Para identificar as retas (\textit{clusters}) presentes em eventos como mostrados na figura \ref{subfig:exemplo_1}, devemos usar estimadores robustos que consigam fazer a detecção mesmo com uma presença grande de \textit{outliers}, pontos que são considerados ruídos. Existem muitos estimadores, que são apenas variações do  Random sample consensus (\textit{RANSAC}) \cite{ransac}, como o MLESAC\cite{ref:MLESAC}, \textit{RRANSAC}, \textit{PROSAC}, dentre outros, que estão presentes no \textit{Point Cloud Library} (PCL)\cite{pcl}. Todos esses se baseiam na ideia de estimar várias possíveis retas de forma aleatória e selecionam a melhor usando uma estimativa (por isso o nome \textit{consensus}). Existe ainda a opção do \textit{Hough Transform}\cite{hough}, porém é um algoritmo de difícil expansão e não demonstrou resultados bons.


% \par Os algoritmos do PCL foram testados e os que mostraram os melhores resultados foram o \textit{RANSAC} e o \textit{PROSAC}, não havendo muita diferença dentre os dois. O estimador usado foi uma variação do \textit{RANSAC}, desenvolvida para fazer melhores estimativas em dados como os do pAT-TPC\cite{artigo}. O algoritmo \ref{ransac_algo}, que chamei de \textit{Enhanced RANSAC} (Para você Juan: só quis diferenciar o nome, até pq eu quero colocar no pypi com esse nome, pode ser?), mostra o seu funcionamento. A melhor estimativa original (que retorna a melhor reta) era simplesmente o número total de \textit{inliers} de um \textit{cluster}. A nova estimativa \textit{C} passou a ser


\begin{algorithm}
    \caption{p-RANSAC}\label{ransac_algo}
    \KwData{pointcloud, $N$, $d_{min}$, $tam_{min}$}
    \For{cada iteração $ i =1,2,\ldots, N$}{
        Seleciona dois pontos da \textit{pointcloud} de modo aleatório (\textit{Random Sampling})\;
        Estima versor $v$ e um ponto $P_b$ que passe pela reta \textit{r} formada pelos dois pontos\;
        \For{cada ponto $P$}{
            Calcula a distância $d$ do ponto à reta \textit{r}\;
            \If{$d < d_{min}$}{
			    Guarda $P$ como pertencente à $r$\;
	        }
	    }
	    \If{Número de pontos de $r > tam_{min}$}{
	        Guarda $v$, $P_b$ e $C$\;
	    }
    }
    Ordena as retas do menor para o maior $C$\;
    \For{cada reta $r$ ordenada}{
        \If{Número de pontos de $r > tam_{min}$}{
            Guarda $v$, $P_b$ e pontos $P$ $\in r$\;
        }
    }
    \Return Retas $r$ selecionadas na última etapa\;
\end{algorithm}

\par O \textit{RANSAC}, originalmente, consegue identificar apenas uma reta. O algoritmo \ref{ransac_algo} se baseia na ideia de usar o RANSAC sequencialmente, ou seja, no momento que um \textit{cluster} tem um tamanho mínimo, então ele é guardado para depois, então, poder ser escolhidos dentre as melhores estimativas \textit{C}. Com isso podemos identificar todas as possíveis estruturas presentes na nuvem de pontos.

\par Uma mudança na eficiência do algoritmo \ref{ransac_algo} é na linha 2, chamada de \textit{Random Sampling}. Podemos nos beneficiar do \textit{Monte Carlo Rejecting} na escolha de dois pontos aleatórios. Por exemplo, caso sejam selecionados pontos muito próximos, o que poderia implicar em uma reta com poucos pontos \textit{inliers}, podemos descarta-los, aumentando a eficiência do \textit{random rampling}.

\par O consumo de tempo do p-RANSAC cresce linearmente com o tamanho da nuvem de pontos. Para diminuir o número de pontos a serem analisados pelo algoritmo, passamos a \textit{pointcloud} por dois filtros. O primeiro filtro exclui pontos baseado na sua carga, colocando um limiar onde pontos com carga $Q$ $<$ 110 são descartados. O segundo filtro, chamado de \textit{outlier removal}, elimina pontos considerados \textit{outliers} globais, de modo que, caso um ponto não possua um número mínimo de vizinhos $n_{or}$ = 4 em um raio de distância $d_{or}$ = 12 mm, então ele é descartado. O \textit{outlier removal} está presente na biblioteca Open3D\cite{open3d} no Python e funciona excluindo pontos muito isolados uns dos outros.

\par O p-RANSAC apresenta falhas no seu resultado, como visto na figura \ref{subfig:antes_clustering}, em que, por exemplo, um único \textit{cluster} acaba sendo dividido em dois \textit{clusters} muito próximos. A etapa de correção da saída do p-RANSAC é chamada de clusterização (não a mesma clusterização usada em \textit{machine learning}). A ideia é comparar, dois a dois, todos os \textit{clusters} resultantes do algoritmo e, se satisfazer um determinado critério, unificar os dois clusters\cite{artigo}. Do ponto de vista computacional, esse problema é abordado avaliando a semelhança entre dois \textit{clusters}. Existem métricas como a dsitância de Jaccard\cite{jaccard_distance} e o coeficiente de silhueta\cite{silhueta}.

\par. A abordagem feita se dá em duas etapas: primeiro comparando os versores entre duas retas e depois verificando uma condição. Caso a diferença absoluta entre os ângulos com relação ao versor (0, 0, 1) seja menor que 9º (determinado empiricamente), então as duas retas serão combinadas se obedecerem a condição dada pela equação \ref{criterio_clustering}, dada por

\begin{equation}\label{criterio_clustering}
    \sum_{i = 0}^{N_1}\frac{d_{i2}}{N_1} < \alpha \space d_{min}, 
\end{equation}

onde $N_1$ é o número de pontos da reta 1, $d_{i2}$ a distância do ponto $i$ da reta 1 em relação à reta 2, $\alpha$ é um parâmetro com valor a ser escolhido e $d_{min}$ é a distância mínima de ponto a reta usada no algoritmo \ref{ransac_algo}. Os valores foram determinados empiricamente, tais que $\alpha$ = 1.75 e $d_{min}$ = 15 mm. A figura \ref{subfig:depois_clustering} mostra o resultado da clusterização baseada nesses critérios no resultado anterior mostrado na figura \ref{subfig:antes_clustering}.

%\par O coeficiente de silhueta compara dois \textit{clusters} e retorna um valor entre -1 e 1 que informa se estão separados corretamente. O valor -1 informa que a separação dos \textit{clusters} estão errados, 0 indica sobreposição e 1 significa que a separação está correta. A métrica se mostra muito viável para análise de duas dimensões, porém para o caso de três dimensões essa métrica não se mostrou muito eficiente. Mesmo colocando um \textit{threshold} muito baixo o algoritmo indicava que deveria juntar \textit{clusters} mesmo que muito distantes um do outro.

% \par A abordagem correta se dá primeiro comparando o ângulo entre as duas retas. Caso a diferença absoluta entre os ângulos com relação ao versor (0, 0, 1) seja menor que um valor dado, que nesse caso foi considerado 9° (determinado empiricamente), então as duas retas serão combinadas se obedecerem a condição dada pela equação \ref{criterio_clustering}.




% \begin{figure}[H]
% \centering
%     \begin{subfigure}[b]{\textwidth}
%         \centering
%         \includegraphics[scale=0.85]{figs/Figure_8.png}
%         \caption{Evento sem a clusterização. As retas de cor azul e verde são de um único \textit{cluster}.}
%         \label{subfig:antes_clustering}
%     \end{subfigure}

%     \begin{subfigure}[b]{\textwidth}
%         \centering
%         \includegraphics[scale=0.85]{figs/Figure_9.png}
%         \caption{Evento após a clusterização. Agora o evento possui as duas retas corretas, a azul e vermelha.}
%         \label{subfig:depois_clustering}
%     \end{subfigure}
% \caption{Sequência de um evento em que em \ref{subfig:antes_clustering} mostra antes da clusterização e \ref{subfig:depois_clustering} mostra depois da correção. As cores das retas são arbitrárias apenas para a diferenciação.}
% \label{fig:clustering}
% \end{figure}
% \begin{figure}[H]
%     \begin{subfigure}
%         \centering
%         \includegraphics[scale=0.85]{Figure_8.png}
%         \caption{O algoritmo \ref{ransac_algo} retornou a reta vermelha, azul e verde, porém a azul e a verde devem ser uma única reta.}
%         \label{fig:exemplo_ransac_sem_clustering}
%     \end{subfigure}
%     \newline
%     \begin{subfigure}
%         \centering
%         \includegraphics[scale=0.85]{Figure_9.png}
%         \caption{Correção da saída mostrada na figura \ref{fig:exemplo_ransac_sem_clustering}. Agora temos as duas retas corretas.}
%         \label{fig:exemplo_ransac_com_clustering}
%     \end{subfigure}
% \end{figure}





% \begin{figure}[H]
%     \centering
%     \includegraphics[scale=0.85]{Figure_9.png}
%     \caption{Correção da saída mostrada na figura \ref{fig:exemplo_ransac_sem_clustering}. Agora temos as duas retas corretas.}
%     \label{fig:exemplo_ransac_com_clustering}
% \end{figure}

\par Após a etapa de clusterização, é necessário classificar cada reta como sendo ou o feixe, ou uma partícula originada de uma reação nuclear. O  incide na câmara com um ângulo muito pequeno com relação ao versor (0, 0, 1). Além disso, mesmo se o ângulo for muito pequeno, a reta do feixe cruza o plano da janela do TPC muito próximo do ponto mais provável da entrada o feixe. O ponto mais provável foi calculado usando a posição média da projeção dos pontos de um conjunto de eventos no plano \textit{x-y}. Disso obtemos que a posição inicial mais provável do feixe é tal que $x_f$ = -3.4 (6.7) mm e $y_f$ = -0.9 (6.3) mm. A incerteza é alta pois o feixe incide de modo bem distribuído da abertura da janela.

\par Portanto se o ângulo entre o versor $\hat{v}_i$ de uma reta $r_i$ for menor que 5° (determinado de modo empírico novamente) e a distância $d$ entre o ponto $P_i$ que intercepta o plano e o ponto ($x_f$, $y_f$, 0), for menor que 15 mm (pouco mais que duas vezes a incerteza de cada ponto), então a reta foi considerada como o feixe do evento. No caso de não satisfazer essas condições, então ela foi classificada como uma possível partícula originada da reação do feixe com o gás.

\par Importante notar que há eventos que não possuem o feixe detectado, como mostrado na figura \ref{fig:exemplo_sem_feixe}. Neste caso, foi necessário assumir as propriedades da reta mais provável para o feixe, ou seja, precisa passar pelo ponto ($x_f$, $y_f$, 0) e ter versor (0, 0, 1).

\begin{figure}[H]
    \centering
    \includegraphics[scale = 0.65]{figs/Figure_12.png}
    \caption{Evento em que não foi detectado o feixe, apenas a partícula espalhada. O triângulo azul é o local calculo do vértice de reação dado pela equação \ref{eq:vertice_reacao}.}
    \label{fig:exemplo_sem_feixe}
\end{figure}

\par Para completar a cinemática do evento, foi necessário calcular o vértice de reação, para cada evento, entre a partícula e o feixe. O vértice de reação é o ponto médio do segmento de reta que conecta a reta a ser analisada (partícula) e o feixe, no ponto de menor distância entre as retas. Ele permitiu fazer correções em etapas futuras e diferençar o começo e o fim de uma trajetória.

\par Para deduzir a equação para o vértice de reação, primeiro temos as seguintes equações das retas $\vec{P_1}$ e $\vec{P_2}$ como vetores:

\begin{equation}
\begin{split}
        &\vec{P_1} = \vec{A_1} + \vec{V_1} * t_1 \\
        &\vec{P_2} = \vec{A_2} + \vec{V_2} * t_2,
\end{split}
\end{equation}

onde $\vec{A_1}$ e $\vec{A_2}$ são pontos arbitrários que pertencem as retas 1 e 2, respectivamente, $\vec{V_1}$ e $\vec{V_2}$ são os versores, $t_1$ e  $t_2$ são os hiperparâmetros das retas.

\par A reta que conecta a menor distância possui versor

\begin{equation}\label{eq:versor_menor_dist}
    \vec{V_c} = \frac{\vec{V_1} \times \vec{V_2}}{\left | \vec{V_1} \times \vec{V_2} \right |}.
\end{equation}

Podemos então construir uma reta $\vec{P_3}$ que conecta $\vec{P_1}$ e $\vec{P_2}$. Essa reta deve começar no ponto de menor distância da reta 1 e terminar no ponto de menor distância da reta 2. Ou seja, temos o seguinte sistema linear:

\begin{equation*}
    \vec{A_2} + \vec{V_2} * \tilde{t_2} = \vec{A_1} + \vec{V_1} * \tilde{t_1} + \vec{V_c} * \tilde{t_3}.
\end{equation*}

Rearranjando temos que

\begin{equation}\label{eq:sistema_vertice}
    \vec{V_1} * \tilde{t_1} - \vec{V_2} * \tilde{t_2} + \vec{V_c} * \tilde{t_3} = \vec{A_2} - \vec{A_1},
\end{equation}

onde $\tilde{t_1}$, $\tilde{t_2}$ e $\tilde{t_3}$ são os hiperparâmetros a serem determinados. Caso $\vec{V_1}$ seja paralelo à $\vec{V_2}$, então não há solução (não há vértice de reação, ou seja, não há uma reação nuclear em comum entre a partícula e o feixe analisado). Achando os valores, achamos os pontos de menor distância nas duas retas:

\begin{equation}
\begin{split}
        &\vec{P_1} = \vec{A_1} + \vec{V_1} * \tilde{t_1} \\
        &\vec{P_2} = \vec{A_2} + \vec{V_2} * \tilde{t_2}.
\end{split}
\end{equation}

\par Com isso, conseguimos determinar que o vértice de reação $\vec{V_r}$ é dado por

\begin{equation} \label{eq:vertice_reacao}
    \vec{V_r} = \frac{1}{2}(\vec{P_1} + \vec{P_2}).
\end{equation}

\par Também podemos definir a distância de máxima aproximação $d_{max}$ das retas, dada pela equação \ref{eq:menor_dist_retas}.

\begin{equation} \label{eq:menor_dist_retas}
    d_{max} = \left | \vec{P_1} - \vec{P_2} \right |.
\end{equation}

% \par No caso em que $\vec{V_c}$ em \ref{eq:versor_menor_dist} é zero, então as retas analisadas são paralelas. Nesse caso não há vértice de reação e a distância de máxima aproximação é dada por

\begin{equation}
    d_{max} = \left | \left(\vec{P_1} - \vec{P_2}\right) \right |.
\end{equation}

\par Da equação \ref{eq:menor_dist_retas} podemos estabelecer um limite superior para $d_{max}$ tal que valores maiores que esse limite indicam uma reação nuclear em comum entre a partícula e o feixe. O valor foi determinado empiricamente e foi definido como $d_{max}^{sup}$ = 25 mm. Trajetórias cuja distância máxima de aproximação excedia $d_{max}^{sup}$, então a trajetória é descartada. A última condição para garantir que houve uma reação nuclear, é garantir que o vértice de reação está dentro da câmara, cujos limites são $|x| < $ 140mm, $|y| < $ 140mm e $|t| < $ 512.

\par Para completar essa etapa, é necessário guarda a energia $E$ da trajetória. Ela é definida como

\begin{equation}
	E = \sum_{i = 1}^{N} Q_i,
\end{equation}

\par onde $Q_i$ é a carga acumulada do i-ésimo ponto de um conjunto com $N$ pontos.

\par Com essa análise feita, tem-se todas as propriedades e informações necessárias dos eventos reconstruídos, sendo possível ir para a etapa de obtenção de resultados físicos mostrada no capítulo \ref{chapter:resultados}. A próxima seção irá mostrar como foi feita a mesma análise utilizando algoritmos de \textit{machine learning}.

%tal que uma reta pode ter do feixe, para então definir se houve realmente a reação no vértice de reação definido pelo equação \ref{eq:vertice_reacao}. Retas que possuíam $d_{max}$ maior ou igual que 25mm foram automaticamente descartadas.

%\par Existem ainda situação em que a reta e o feixe tem uma distância de máxima aproximação muito pequena, porém o vértice de reação estava foram dos limites do TPC ($|x| < $ 140mm, $|y| < $ 140mm e $|t| < $ 512), o que também indica que a reta deve ser descartada.

%\par Agora com apenas as retas certas selecionadas devemos apenas tomar cuidado de excluir eventos em que apenas o feixe foi detectado, pois neste caso não há o evento físico para ser analisado.



\section{Métodos com \textit{machine learning}}\label{sec:ml_pc}

\par Na seção \ref{sec:forcabruta} descrevemos o algoritmo completo de seleção de eventos para a análise. Agora o objetivo é aplicar algoritmos de \textit{machine learning} para melhorar o funcionamento de algo já existente, sugerir ou resolver o mesmo problema de uma maneira totalmente diferente.

\subsection{Clusterização hierárquica}


\par Nessa seção 
\par Como queremos classificar pontos, dando \textit{labels} que diferenciem os pontos de cada reta, podemos usar algoritmos de clusterização (\textit{machine learning} não supervisionado). Devemos escolher algoritmos que sejam muito rápidos pois a ideia é substituir o uso do RANSAC. O algoritmo usado é o \textit{Hierarchical Density-Based Spatial Clustering of Applications with Noise} (HDBSCAN)\cite{hdbscan1, hdbscan2}, pois é muito eficiente em tempo e consegue realizar \textit{clustering} mesmo em dados muito complexos. O algoritmo consegue também nos dar os \textit{outliers} da \textit{pointcloud}, porém para eliminação de \textit{outliers} será usado novamente o \textit{outlier removal} do Open3D. A figura \ref{fig:cl_exs} mostra resultados do uso do algoritmo, já com as retas ajustadas. Percebe-se casos em que a houve falha na clusterização.

\begin{figure}[H]
\centering
    \begin{subfigure}[b]{0.45\textwidth}
        \centering
        \includegraphics[scale=0.37]{figs/cl_ex1.png}
        \caption{}
        \label{subfig:cl_ex1}
    \end{subfigure}%
    \hfill
    \begin{subfigure}[b]{0.45\textwidth}
        \centering
        \includegraphics[scale=0.37]{figs/cl_ex2.png}
        \caption{}
        \label{subfig:cl_ex2}
    \end{subfigure}
    \begin{subfigure}[b]{0.45\textwidth}
        \centering
        \includegraphics[scale=0.37]{figs/cl_ex3.png}
        \caption{}
        \label{subfig:cl_ex3}
    \end{subfigure}%
    \hfill
    \begin{subfigure}[b]{0.45\textwidth}
        \centering
        \includegraphics[scale=0.37]{figs/cl_ex4.png}
        \caption{}
        \label{subfig:cl_ex4}
    \end{subfigure}
\caption{Exemplos dos resultados para o HDBSCAN. Percebe-se que em \ref{subfig:cl_ex3} e \ref{subfig:cl_ex4} o algoritmo falhou, juntando diferentes \textit{clusters} ou simplesmente detectando ruído junto da \textit{track}.}
\label{fig:cl_exs}
\end{figure}

\par A eficiência em relação ao RANSAC é menor, pois os \textit{clusters} são pouco densos, o que dificulta a seleção de \textit{clusters} pelo algoritmo. Além disso, o algoritmo tem dificuldade em separar \textit{clusters} que possuem pontos sobrepostos, ou seja, onde há vértice de reação (vide figuras \ref{subfig:cl_ex3} e \ref{subfig:cl_ex4}). A tabela \ref{tabela:ransacvshdbscan} mostra a comparação entre a taxa de acerto e a eficiência em tempo entre o HDBSCAN e do RANSAC. A taxa de acertos mostra mostra quantos foram corretamente solucionados (ou seja, todos os \textit{clusters} foram detectados corretamente), e a eficiência mostra a capacidade de processamento de eventos pelo algoritmo, medida em eventos por segundo. O \textit{benchmark} foi feito usando o processador Ryzen 5 3600X.

\begin{table}[H]
\centering
\caption{Comparação entre algoritmos usados para identificar tracks em eventos. O HDBSCAN acerta menos vezes em comparação com o RANSAC (cerca de 14\% menos), porém é quase 4 vezes mais rápido. }
\label{tabela:ransacvshdbscan}
\begin{tabular}{|c|c|c|}
\hline
Método  & Taxa de acertos (\%) & Eficiência (eventos/s) \\ \hline
RANSAC  & 78.2                 & 57                     \\ \hline
HDBSCAN & 64.3                 & 208                    \\ \hline
\end{tabular}
\end{table}

\par A clusterização se mostrou melhor em eventos que tinham uma separação clara entre os \textit{clusters}, sem pontos que coincidem duas \textit{tracks} diferentes\cite{TriplClust}, e também nos casos em que a densidade de pontos era muito significativo. Apesar da queda na taxa de acertos a velocidade de execução sobe significativamente, sendo uma possível escolha no lugar do RANSAC.

\chapter{Resultados}\label{chapter:resultados}

\par Após identificar todos os \textit{clusters} de cada evento devemos identificar quais são as partículas que originaram cada uma das trajetórias detectadas. Temos o comprimento de cada trajetória e sua energia, portanto o objetivo é, dada essas duas informações, identificar qual a partícula. A figura \ref{fig:coincidencia_detec} mostra um histograma bidimensional do comprimento de cada \textit{track} em função do ângulo de espalhamento no referencial do laboratório, usando apenas eventos que possuíam duas \textit{tracks} com o mesmo vértice de reação. É possível perceber as medidas coincidentes do $^{16}$O e do próton.

\begin{figure}[H]
    \centering
    \includegraphics[scale = 0.75]{figs/coincidencias_tpc.png}\
    \caption{Histograma de comprimento de \textit{track} no eixo y e ângulo de espalhamento no eixo x. O histograma foi feito coletando eventos que possuíam duas trajetórias com o mesmo vértice de reação, indicando a detecção simultânea do $^{16}$O e do próton.}
    \label{fig:coincidencia_detec}
\end{figure}

\par Para determinar qual é a partícula de cada \textit{track} podemos usar o LISE++\cite{lise++} para calcular o alcance (\textit{range}) das possíveis partículas ($^{17}$F, $^{16}$O e próton) dada as propriedades do alvo ($^4$He à uma pressão de 350 Torr). A figura \ref{fig:alcance_vs_energia} mostra o alcance em mm das partículas em função da energia em MeV.

% \begin{figure}[H]
%     \centering
%     \includegraphics[scale = 0.75]{figs/alcance_vs_energia_2.png}\
%     \caption{Alcance (mm) em função da energia (MeV) para o $^{17}$F, $^{16}$O e próton. A grande diferença está no próton que tem um alcance muito maior que os outros núcleos.}
%     \label{fig:alcance_vs_energia}
% \end{figure}


\par 

% \begin{figure}[H]
%     \centering
%     %% Creator: Matplotlib, PGF backend
%%
%% To include the figure in your LaTeX document, write
%%   \input{<filename>.pgf}
%%
%% Make sure the required packages are loaded in your preamble
%%   \usepackage{pgf}
%%
%% Figures using additional raster images can only be included by \input if
%% they are in the same directory as the main LaTeX file. For loading figures
%% from other directories you can use the `import` package
%%   \usepackage{import}
%%
%% and then include the figures with
%%   \import{<path to file>}{<filename>.pgf}
%%
%% Matplotlib used the following preamble
%%
\begingroup%
\makeatletter%
\begin{pgfpicture}%
\pgfpathrectangle{\pgfpointorigin}{\pgfqpoint{6.400000in}{4.800000in}}%
\pgfusepath{use as bounding box, clip}%
\begin{pgfscope}%
\pgfsetbuttcap%
\pgfsetmiterjoin%
\definecolor{currentfill}{rgb}{1.000000,1.000000,1.000000}%
\pgfsetfillcolor{currentfill}%
\pgfsetlinewidth{0.000000pt}%
\definecolor{currentstroke}{rgb}{1.000000,1.000000,1.000000}%
\pgfsetstrokecolor{currentstroke}%
\pgfsetdash{}{0pt}%
\pgfpathmoveto{\pgfqpoint{0.000000in}{0.000000in}}%
\pgfpathlineto{\pgfqpoint{6.400000in}{0.000000in}}%
\pgfpathlineto{\pgfqpoint{6.400000in}{4.800000in}}%
\pgfpathlineto{\pgfqpoint{0.000000in}{4.800000in}}%
\pgfpathclose%
\pgfusepath{fill}%
\end{pgfscope}%
\begin{pgfscope}%
\pgfsetbuttcap%
\pgfsetmiterjoin%
\definecolor{currentfill}{rgb}{1.000000,1.000000,1.000000}%
\pgfsetfillcolor{currentfill}%
\pgfsetlinewidth{0.000000pt}%
\definecolor{currentstroke}{rgb}{0.000000,0.000000,0.000000}%
\pgfsetstrokecolor{currentstroke}%
\pgfsetstrokeopacity{0.000000}%
\pgfsetdash{}{0pt}%
\pgfpathmoveto{\pgfqpoint{0.642863in}{0.580957in}}%
\pgfpathlineto{\pgfqpoint{6.250000in}{0.580957in}}%
\pgfpathlineto{\pgfqpoint{6.250000in}{4.650000in}}%
\pgfpathlineto{\pgfqpoint{0.642863in}{4.650000in}}%
\pgfpathclose%
\pgfusepath{fill}%
\end{pgfscope}%
\begin{pgfscope}%
\pgfpathrectangle{\pgfqpoint{0.642863in}{0.580957in}}{\pgfqpoint{5.607137in}{4.069043in}}%
\pgfusepath{clip}%
\pgfsetbuttcap%
\pgfsetroundjoin%
\definecolor{currentfill}{rgb}{1.000000,0.000000,0.000000}%
\pgfsetfillcolor{currentfill}%
\pgfsetlinewidth{1.003750pt}%
\definecolor{currentstroke}{rgb}{1.000000,0.000000,0.000000}%
\pgfsetstrokecolor{currentstroke}%
\pgfsetdash{}{0pt}%
\pgfsys@defobject{currentmarker}{\pgfqpoint{-0.006944in}{-0.006944in}}{\pgfqpoint{0.006944in}{0.006944in}}{%
\pgfpathmoveto{\pgfqpoint{0.000000in}{-0.006944in}}%
\pgfpathcurveto{\pgfqpoint{0.001842in}{-0.006944in}}{\pgfqpoint{0.003608in}{-0.006213in}}{\pgfqpoint{0.004910in}{-0.004910in}}%
\pgfpathcurveto{\pgfqpoint{0.006213in}{-0.003608in}}{\pgfqpoint{0.006944in}{-0.001842in}}{\pgfqpoint{0.006944in}{0.000000in}}%
\pgfpathcurveto{\pgfqpoint{0.006944in}{0.001842in}}{\pgfqpoint{0.006213in}{0.003608in}}{\pgfqpoint{0.004910in}{0.004910in}}%
\pgfpathcurveto{\pgfqpoint{0.003608in}{0.006213in}}{\pgfqpoint{0.001842in}{0.006944in}}{\pgfqpoint{0.000000in}{0.006944in}}%
\pgfpathcurveto{\pgfqpoint{-0.001842in}{0.006944in}}{\pgfqpoint{-0.003608in}{0.006213in}}{\pgfqpoint{-0.004910in}{0.004910in}}%
\pgfpathcurveto{\pgfqpoint{-0.006213in}{0.003608in}}{\pgfqpoint{-0.006944in}{0.001842in}}{\pgfqpoint{-0.006944in}{0.000000in}}%
\pgfpathcurveto{\pgfqpoint{-0.006944in}{-0.001842in}}{\pgfqpoint{-0.006213in}{-0.003608in}}{\pgfqpoint{-0.004910in}{-0.004910in}}%
\pgfpathcurveto{\pgfqpoint{-0.003608in}{-0.006213in}}{\pgfqpoint{-0.001842in}{-0.006944in}}{\pgfqpoint{0.000000in}{-0.006944in}}%
\pgfpathclose%
\pgfusepath{stroke,fill}%
}%
\begin{pgfscope}%
\pgfsys@transformshift{0.897733in}{0.765914in}%
\pgfsys@useobject{currentmarker}{}%
\end{pgfscope}%
\begin{pgfscope}%
\pgfsys@transformshift{1.097042in}{0.899189in}%
\pgfsys@useobject{currentmarker}{}%
\end{pgfscope}%
\begin{pgfscope}%
\pgfsys@transformshift{1.360513in}{1.081166in}%
\pgfsys@useobject{currentmarker}{}%
\end{pgfscope}%
\begin{pgfscope}%
\pgfsys@transformshift{1.495658in}{1.171307in}%
\pgfsys@useobject{currentmarker}{}%
\end{pgfscope}%
\begin{pgfscope}%
\pgfsys@transformshift{1.559821in}{1.212608in}%
\pgfsys@useobject{currentmarker}{}%
\end{pgfscope}%
\begin{pgfscope}%
\pgfsys@transformshift{1.676409in}{1.284838in}%
\pgfsys@useobject{currentmarker}{}%
\end{pgfscope}%
\begin{pgfscope}%
\pgfsys@transformshift{1.759130in}{1.333875in}%
\pgfsys@useobject{currentmarker}{}%
\end{pgfscope}%
\begin{pgfscope}%
\pgfsys@transformshift{1.823293in}{1.370637in}%
\pgfsys@useobject{currentmarker}{}%
\end{pgfscope}%
\begin{pgfscope}%
\pgfsys@transformshift{1.875718in}{1.397701in}%
\pgfsys@useobject{currentmarker}{}%
\end{pgfscope}%
\begin{pgfscope}%
\pgfsys@transformshift{1.920042in}{1.417655in}%
\pgfsys@useobject{currentmarker}{}%
\end{pgfscope}%
\begin{pgfscope}%
\pgfsys@transformshift{1.958438in}{1.433337in}%
\pgfsys@useobject{currentmarker}{}%
\end{pgfscope}%
\begin{pgfscope}%
\pgfsys@transformshift{1.992306in}{1.446229in}%
\pgfsys@useobject{currentmarker}{}%
\end{pgfscope}%
\begin{pgfscope}%
\pgfsys@transformshift{2.022601in}{1.457177in}%
\pgfsys@useobject{currentmarker}{}%
\end{pgfscope}%
\begin{pgfscope}%
\pgfsys@transformshift{2.050007in}{1.466716in}%
\pgfsys@useobject{currentmarker}{}%
\end{pgfscope}%
\begin{pgfscope}%
\pgfsys@transformshift{2.075026in}{1.475183in}%
\pgfsys@useobject{currentmarker}{}%
\end{pgfscope}%
\begin{pgfscope}%
\pgfsys@transformshift{2.119351in}{1.489717in}%
\pgfsys@useobject{currentmarker}{}%
\end{pgfscope}%
\begin{pgfscope}%
\pgfsys@transformshift{2.157747in}{1.501915in}%
\pgfsys@useobject{currentmarker}{}%
\end{pgfscope}%
\begin{pgfscope}%
\pgfsys@transformshift{2.191614in}{1.512442in}%
\pgfsys@useobject{currentmarker}{}%
\end{pgfscope}%
\begin{pgfscope}%
\pgfsys@transformshift{2.221910in}{1.521716in}%
\pgfsys@useobject{currentmarker}{}%
\end{pgfscope}%
\begin{pgfscope}%
\pgfsys@transformshift{2.274334in}{1.537529in}%
\pgfsys@useobject{currentmarker}{}%
\end{pgfscope}%
\begin{pgfscope}%
\pgfsys@transformshift{2.318659in}{1.550823in}%
\pgfsys@useobject{currentmarker}{}%
\end{pgfscope}%
\begin{pgfscope}%
\pgfsys@transformshift{2.357055in}{1.562412in}%
\pgfsys@useobject{currentmarker}{}%
\end{pgfscope}%
\begin{pgfscope}%
\pgfsys@transformshift{2.390922in}{1.572787in}%
\pgfsys@useobject{currentmarker}{}%
\end{pgfscope}%
\begin{pgfscope}%
\pgfsys@transformshift{2.421218in}{1.582275in}%
\pgfsys@useobject{currentmarker}{}%
\end{pgfscope}%
\begin{pgfscope}%
\pgfsys@transformshift{2.485381in}{1.603230in}%
\pgfsys@useobject{currentmarker}{}%
\end{pgfscope}%
\begin{pgfscope}%
\pgfsys@transformshift{2.537806in}{1.621434in}%
\pgfsys@useobject{currentmarker}{}%
\end{pgfscope}%
\begin{pgfscope}%
\pgfsys@transformshift{2.582131in}{1.637631in}%
\pgfsys@useobject{currentmarker}{}%
\end{pgfscope}%
\begin{pgfscope}%
\pgfsys@transformshift{2.620526in}{1.652357in}%
\pgfsys@useobject{currentmarker}{}%
\end{pgfscope}%
\begin{pgfscope}%
\pgfsys@transformshift{2.654394in}{1.665975in}%
\pgfsys@useobject{currentmarker}{}%
\end{pgfscope}%
\begin{pgfscope}%
\pgfsys@transformshift{2.684689in}{1.678710in}%
\pgfsys@useobject{currentmarker}{}%
\end{pgfscope}%
\begin{pgfscope}%
\pgfsys@transformshift{2.737114in}{1.702050in}%
\pgfsys@useobject{currentmarker}{}%
\end{pgfscope}%
\begin{pgfscope}%
\pgfsys@transformshift{2.781439in}{1.723159in}%
\pgfsys@useobject{currentmarker}{}%
\end{pgfscope}%
\begin{pgfscope}%
\pgfsys@transformshift{2.819835in}{1.742467in}%
\pgfsys@useobject{currentmarker}{}%
\end{pgfscope}%
\begin{pgfscope}%
\pgfsys@transformshift{2.853702in}{1.760412in}%
\pgfsys@useobject{currentmarker}{}%
\end{pgfscope}%
\begin{pgfscope}%
\pgfsys@transformshift{2.883998in}{1.777318in}%
\pgfsys@useobject{currentmarker}{}%
\end{pgfscope}%
\begin{pgfscope}%
\pgfsys@transformshift{2.948161in}{1.815957in}%
\pgfsys@useobject{currentmarker}{}%
\end{pgfscope}%
\begin{pgfscope}%
\pgfsys@transformshift{3.000586in}{1.850554in}%
\pgfsys@useobject{currentmarker}{}%
\end{pgfscope}%
\begin{pgfscope}%
\pgfsys@transformshift{3.044910in}{1.881982in}%
\pgfsys@useobject{currentmarker}{}%
\end{pgfscope}%
\begin{pgfscope}%
\pgfsys@transformshift{3.083306in}{1.910820in}%
\pgfsys@useobject{currentmarker}{}%
\end{pgfscope}%
\begin{pgfscope}%
\pgfsys@transformshift{3.117174in}{1.937483in}%
\pgfsys@useobject{currentmarker}{}%
\end{pgfscope}%
\begin{pgfscope}%
\pgfsys@transformshift{3.147469in}{1.962295in}%
\pgfsys@useobject{currentmarker}{}%
\end{pgfscope}%
\begin{pgfscope}%
\pgfsys@transformshift{3.199894in}{2.007688in}%
\pgfsys@useobject{currentmarker}{}%
\end{pgfscope}%
\begin{pgfscope}%
\pgfsys@transformshift{3.244219in}{2.047970in}%
\pgfsys@useobject{currentmarker}{}%
\end{pgfscope}%
\begin{pgfscope}%
\pgfsys@transformshift{3.282614in}{2.084205in}%
\pgfsys@useobject{currentmarker}{}%
\end{pgfscope}%
\begin{pgfscope}%
\pgfsys@transformshift{3.316482in}{2.117150in}%
\pgfsys@useobject{currentmarker}{}%
\end{pgfscope}%
\begin{pgfscope}%
\pgfsys@transformshift{3.346777in}{2.147355in}%
\pgfsys@useobject{currentmarker}{}%
\end{pgfscope}%
\begin{pgfscope}%
\pgfsys@transformshift{3.374183in}{2.175248in}%
\pgfsys@useobject{currentmarker}{}%
\end{pgfscope}%
\begin{pgfscope}%
\pgfsys@transformshift{3.399202in}{2.201156in}%
\pgfsys@useobject{currentmarker}{}%
\end{pgfscope}%
\begin{pgfscope}%
\pgfsys@transformshift{3.422218in}{2.225346in}%
\pgfsys@useobject{currentmarker}{}%
\end{pgfscope}%
\begin{pgfscope}%
\pgfsys@transformshift{3.443527in}{2.248031in}%
\pgfsys@useobject{currentmarker}{}%
\end{pgfscope}%
\begin{pgfscope}%
\pgfsys@transformshift{3.463365in}{2.269387in}%
\pgfsys@useobject{currentmarker}{}%
\end{pgfscope}%
\begin{pgfscope}%
\pgfsys@transformshift{3.481923in}{2.289560in}%
\pgfsys@useobject{currentmarker}{}%
\end{pgfscope}%
\begin{pgfscope}%
\pgfsys@transformshift{3.499355in}{2.308675in}%
\pgfsys@useobject{currentmarker}{}%
\end{pgfscope}%
\begin{pgfscope}%
\pgfsys@transformshift{3.515790in}{2.326836in}%
\pgfsys@useobject{currentmarker}{}%
\end{pgfscope}%
\begin{pgfscope}%
\pgfsys@transformshift{3.531337in}{2.344133in}%
\pgfsys@useobject{currentmarker}{}%
\end{pgfscope}%
\begin{pgfscope}%
\pgfsys@transformshift{3.546086in}{2.360644in}%
\pgfsys@useobject{currentmarker}{}%
\end{pgfscope}%
\begin{pgfscope}%
\pgfsys@transformshift{3.573491in}{2.391572in}%
\pgfsys@useobject{currentmarker}{}%
\end{pgfscope}%
\begin{pgfscope}%
\pgfsys@transformshift{3.598511in}{2.420068in}%
\pgfsys@useobject{currentmarker}{}%
\end{pgfscope}%
\begin{pgfscope}%
\pgfsys@transformshift{3.621526in}{2.446485in}%
\pgfsys@useobject{currentmarker}{}%
\end{pgfscope}%
\begin{pgfscope}%
\pgfsys@transformshift{3.642836in}{2.471102in}%
\pgfsys@useobject{currentmarker}{}%
\end{pgfscope}%
\begin{pgfscope}%
\pgfsys@transformshift{3.662674in}{2.494146in}%
\pgfsys@useobject{currentmarker}{}%
\end{pgfscope}%
\begin{pgfscope}%
\pgfsys@transformshift{3.681231in}{2.515804in}%
\pgfsys@useobject{currentmarker}{}%
\end{pgfscope}%
\begin{pgfscope}%
\pgfsys@transformshift{3.698663in}{2.536232in}%
\pgfsys@useobject{currentmarker}{}%
\end{pgfscope}%
\begin{pgfscope}%
\pgfsys@transformshift{3.715099in}{2.555560in}%
\pgfsys@useobject{currentmarker}{}%
\end{pgfscope}%
\begin{pgfscope}%
\pgfsys@transformshift{3.730645in}{2.573899in}%
\pgfsys@useobject{currentmarker}{}%
\end{pgfscope}%
\begin{pgfscope}%
\pgfsys@transformshift{3.745394in}{2.591345in}%
\pgfsys@useobject{currentmarker}{}%
\end{pgfscope}%
\begin{pgfscope}%
\pgfsys@transformshift{3.759423in}{2.607978in}%
\pgfsys@useobject{currentmarker}{}%
\end{pgfscope}%
\begin{pgfscope}%
\pgfsys@transformshift{3.772800in}{2.623871in}%
\pgfsys@useobject{currentmarker}{}%
\end{pgfscope}%
\begin{pgfscope}%
\pgfsys@transformshift{3.785582in}{2.639085in}%
\pgfsys@useobject{currentmarker}{}%
\end{pgfscope}%
\begin{pgfscope}%
\pgfsys@transformshift{3.797819in}{2.653674in}%
\pgfsys@useobject{currentmarker}{}%
\end{pgfscope}%
\begin{pgfscope}%
\pgfsys@transformshift{3.809557in}{2.667689in}%
\pgfsys@useobject{currentmarker}{}%
\end{pgfscope}%
\begin{pgfscope}%
\pgfsys@transformshift{3.831687in}{2.694159in}%
\pgfsys@useobject{currentmarker}{}%
\end{pgfscope}%
\begin{pgfscope}%
\pgfsys@transformshift{3.852234in}{2.718785in}%
\pgfsys@useobject{currentmarker}{}%
\end{pgfscope}%
\begin{pgfscope}%
\pgfsys@transformshift{3.871411in}{2.741805in}%
\pgfsys@useobject{currentmarker}{}%
\end{pgfscope}%
\begin{pgfscope}%
\pgfsys@transformshift{3.889388in}{2.763411in}%
\pgfsys@useobject{currentmarker}{}%
\end{pgfscope}%
\begin{pgfscope}%
\pgfsys@transformshift{3.906307in}{2.783764in}%
\pgfsys@useobject{currentmarker}{}%
\end{pgfscope}%
\begin{pgfscope}%
\pgfsys@transformshift{3.929954in}{2.812235in}%
\pgfsys@useobject{currentmarker}{}%
\end{pgfscope}%
\begin{pgfscope}%
\pgfsys@transformshift{3.951803in}{2.838559in}%
\pgfsys@useobject{currentmarker}{}%
\end{pgfscope}%
\begin{pgfscope}%
\pgfsys@transformshift{3.972108in}{2.863031in}%
\pgfsys@useobject{currentmarker}{}%
\end{pgfscope}%
\begin{pgfscope}%
\pgfsys@transformshift{3.991074in}{2.885889in}%
\pgfsys@useobject{currentmarker}{}%
\end{pgfscope}%
\begin{pgfscope}%
\pgfsys@transformshift{4.008866in}{2.907330in}%
\pgfsys@useobject{currentmarker}{}%
\end{pgfscope}%
\begin{pgfscope}%
\pgfsys@transformshift{4.036271in}{2.940344in}%
\pgfsys@useobject{currentmarker}{}%
\end{pgfscope}%
\begin{pgfscope}%
\pgfsys@transformshift{4.061291in}{2.970460in}%
\pgfsys@useobject{currentmarker}{}%
\end{pgfscope}%
\begin{pgfscope}%
\pgfsys@transformshift{4.084306in}{2.998136in}%
\pgfsys@useobject{currentmarker}{}%
\end{pgfscope}%
\begin{pgfscope}%
\pgfsys@transformshift{4.105615in}{3.023730in}%
\pgfsys@useobject{currentmarker}{}%
\end{pgfscope}%
\begin{pgfscope}%
\pgfsys@transformshift{4.125454in}{3.047527in}%
\pgfsys@useobject{currentmarker}{}%
\end{pgfscope}%
\begin{pgfscope}%
\pgfsys@transformshift{4.144011in}{3.069758in}%
\pgfsys@useobject{currentmarker}{}%
\end{pgfscope}%
\begin{pgfscope}%
\pgfsys@transformshift{4.161443in}{3.090610in}%
\pgfsys@useobject{currentmarker}{}%
\end{pgfscope}%
\begin{pgfscope}%
\pgfsys@transformshift{4.177879in}{3.110242in}%
\pgfsys@useobject{currentmarker}{}%
\end{pgfscope}%
\begin{pgfscope}%
\pgfsys@transformshift{4.193425in}{3.128785in}%
\pgfsys@useobject{currentmarker}{}%
\end{pgfscope}%
\begin{pgfscope}%
\pgfsys@transformshift{4.208174in}{3.146350in}%
\pgfsys@useobject{currentmarker}{}%
\end{pgfscope}%
\begin{pgfscope}%
\pgfsys@transformshift{4.222203in}{3.163033in}%
\pgfsys@useobject{currentmarker}{}%
\end{pgfscope}%
\begin{pgfscope}%
\pgfsys@transformshift{4.235580in}{3.178915in}%
\pgfsys@useobject{currentmarker}{}%
\end{pgfscope}%
\begin{pgfscope}%
\pgfsys@transformshift{4.248361in}{3.194068in}%
\pgfsys@useobject{currentmarker}{}%
\end{pgfscope}%
\begin{pgfscope}%
\pgfsys@transformshift{4.260599in}{3.208555in}%
\pgfsys@useobject{currentmarker}{}%
\end{pgfscope}%
\begin{pgfscope}%
\pgfsys@transformshift{4.272337in}{3.222430in}%
\pgfsys@useobject{currentmarker}{}%
\end{pgfscope}%
\begin{pgfscope}%
\pgfsys@transformshift{4.283615in}{3.235740in}%
\pgfsys@useobject{currentmarker}{}%
\end{pgfscope}%
\begin{pgfscope}%
\pgfsys@transformshift{4.294466in}{3.248530in}%
\pgfsys@useobject{currentmarker}{}%
\end{pgfscope}%
\begin{pgfscope}%
\pgfsys@transformshift{4.304924in}{3.260836in}%
\pgfsys@useobject{currentmarker}{}%
\end{pgfscope}%
\begin{pgfscope}%
\pgfsys@transformshift{4.315014in}{3.272693in}%
\pgfsys@useobject{currentmarker}{}%
\end{pgfscope}%
\begin{pgfscope}%
\pgfsys@transformshift{4.324762in}{3.284132in}%
\pgfsys@useobject{currentmarker}{}%
\end{pgfscope}%
\begin{pgfscope}%
\pgfsys@transformshift{4.334190in}{3.295309in}%
\pgfsys@useobject{currentmarker}{}%
\end{pgfscope}%
\begin{pgfscope}%
\pgfsys@transformshift{4.343319in}{3.306100in}%
\pgfsys@useobject{currentmarker}{}%
\end{pgfscope}%
\begin{pgfscope}%
\pgfsys@transformshift{4.352168in}{3.316526in}%
\pgfsys@useobject{currentmarker}{}%
\end{pgfscope}%
\begin{pgfscope}%
\pgfsys@transformshift{4.360752in}{3.326609in}%
\pgfsys@useobject{currentmarker}{}%
\end{pgfscope}%
\begin{pgfscope}%
\pgfsys@transformshift{4.369087in}{3.336372in}%
\pgfsys@useobject{currentmarker}{}%
\end{pgfscope}%
\begin{pgfscope}%
\pgfsys@transformshift{4.377187in}{3.345832in}%
\pgfsys@useobject{currentmarker}{}%
\end{pgfscope}%
\begin{pgfscope}%
\pgfsys@transformshift{4.385065in}{3.355008in}%
\pgfsys@useobject{currentmarker}{}%
\end{pgfscope}%
\begin{pgfscope}%
\pgfsys@transformshift{4.392733in}{3.363915in}%
\pgfsys@useobject{currentmarker}{}%
\end{pgfscope}%
\begin{pgfscope}%
\pgfsys@transformshift{4.400203in}{3.372568in}%
\pgfsys@useobject{currentmarker}{}%
\end{pgfscope}%
\begin{pgfscope}%
\pgfsys@transformshift{4.407482in}{3.380982in}%
\pgfsys@useobject{currentmarker}{}%
\end{pgfscope}%
\begin{pgfscope}%
\pgfsys@transformshift{4.421512in}{3.397136in}%
\pgfsys@useobject{currentmarker}{}%
\end{pgfscope}%
\begin{pgfscope}%
\pgfsys@transformshift{4.434888in}{3.412467in}%
\pgfsys@useobject{currentmarker}{}%
\end{pgfscope}%
\begin{pgfscope}%
\pgfsys@transformshift{4.447670in}{3.427051in}%
\pgfsys@useobject{currentmarker}{}%
\end{pgfscope}%
\begin{pgfscope}%
\pgfsys@transformshift{4.459907in}{3.440955in}%
\pgfsys@useobject{currentmarker}{}%
\end{pgfscope}%
\begin{pgfscope}%
\pgfsys@transformshift{4.471645in}{3.454237in}%
\pgfsys@useobject{currentmarker}{}%
\end{pgfscope}%
\begin{pgfscope}%
\pgfsys@transformshift{4.482923in}{3.466947in}%
\pgfsys@useobject{currentmarker}{}%
\end{pgfscope}%
\begin{pgfscope}%
\pgfsys@transformshift{4.493775in}{3.479128in}%
\pgfsys@useobject{currentmarker}{}%
\end{pgfscope}%
\begin{pgfscope}%
\pgfsys@transformshift{4.504232in}{3.490825in}%
\pgfsys@useobject{currentmarker}{}%
\end{pgfscope}%
\begin{pgfscope}%
\pgfsys@transformshift{4.514322in}{3.502068in}%
\pgfsys@useobject{currentmarker}{}%
\end{pgfscope}%
\begin{pgfscope}%
\pgfsys@transformshift{4.524070in}{3.512893in}%
\pgfsys@useobject{currentmarker}{}%
\end{pgfscope}%
\begin{pgfscope}%
\pgfsys@transformshift{4.533499in}{3.523326in}%
\pgfsys@useobject{currentmarker}{}%
\end{pgfscope}%
\begin{pgfscope}%
\pgfsys@transformshift{4.542628in}{3.533394in}%
\pgfsys@useobject{currentmarker}{}%
\end{pgfscope}%
\begin{pgfscope}%
\pgfsys@transformshift{4.551476in}{3.543119in}%
\pgfsys@useobject{currentmarker}{}%
\end{pgfscope}%
\begin{pgfscope}%
\pgfsys@transformshift{4.560060in}{3.552523in}%
\pgfsys@useobject{currentmarker}{}%
\end{pgfscope}%
\begin{pgfscope}%
\pgfsys@transformshift{4.568395in}{3.561626in}%
\pgfsys@useobject{currentmarker}{}%
\end{pgfscope}%
\begin{pgfscope}%
\pgfsys@transformshift{4.576495in}{3.570445in}%
\pgfsys@useobject{currentmarker}{}%
\end{pgfscope}%
\begin{pgfscope}%
\pgfsys@transformshift{4.584374in}{3.578995in}%
\pgfsys@useobject{currentmarker}{}%
\end{pgfscope}%
\begin{pgfscope}%
\pgfsys@transformshift{4.592042in}{3.587292in}%
\pgfsys@useobject{currentmarker}{}%
\end{pgfscope}%
\begin{pgfscope}%
\pgfsys@transformshift{4.599511in}{3.595350in}%
\pgfsys@useobject{currentmarker}{}%
\end{pgfscope}%
\begin{pgfscope}%
\pgfsys@transformshift{4.606791in}{3.603180in}%
\pgfsys@useobject{currentmarker}{}%
\end{pgfscope}%
\begin{pgfscope}%
\pgfsys@transformshift{4.613891in}{3.610796in}%
\pgfsys@useobject{currentmarker}{}%
\end{pgfscope}%
\begin{pgfscope}%
\pgfsys@transformshift{4.620820in}{3.618206in}%
\pgfsys@useobject{currentmarker}{}%
\end{pgfscope}%
\begin{pgfscope}%
\pgfsys@transformshift{4.627586in}{3.625423in}%
\pgfsys@useobject{currentmarker}{}%
\end{pgfscope}%
\begin{pgfscope}%
\pgfsys@transformshift{4.634196in}{3.632453in}%
\pgfsys@useobject{currentmarker}{}%
\end{pgfscope}%
\begin{pgfscope}%
\pgfsys@transformshift{4.640658in}{3.639306in}%
\pgfsys@useobject{currentmarker}{}%
\end{pgfscope}%
\begin{pgfscope}%
\pgfsys@transformshift{4.646978in}{3.645991in}%
\pgfsys@useobject{currentmarker}{}%
\end{pgfscope}%
\begin{pgfscope}%
\pgfsys@transformshift{4.653162in}{3.652514in}%
\pgfsys@useobject{currentmarker}{}%
\end{pgfscope}%
\begin{pgfscope}%
\pgfsys@transformshift{4.659216in}{3.658884in}%
\pgfsys@useobject{currentmarker}{}%
\end{pgfscope}%
\begin{pgfscope}%
\pgfsys@transformshift{4.665145in}{3.665106in}%
\pgfsys@useobject{currentmarker}{}%
\end{pgfscope}%
\begin{pgfscope}%
\pgfsys@transformshift{4.670954in}{3.671186in}%
\pgfsys@useobject{currentmarker}{}%
\end{pgfscope}%
\begin{pgfscope}%
\pgfsys@transformshift{4.682231in}{3.682945in}%
\pgfsys@useobject{currentmarker}{}%
\end{pgfscope}%
\begin{pgfscope}%
\pgfsys@transformshift{4.693083in}{3.694204in}%
\pgfsys@useobject{currentmarker}{}%
\end{pgfscope}%
\begin{pgfscope}%
\pgfsys@transformshift{4.703540in}{3.704999in}%
\pgfsys@useobject{currentmarker}{}%
\end{pgfscope}%
\begin{pgfscope}%
\pgfsys@transformshift{4.713631in}{3.715366in}%
\pgfsys@useobject{currentmarker}{}%
\end{pgfscope}%
\begin{pgfscope}%
\pgfsys@transformshift{4.723379in}{3.725333in}%
\pgfsys@useobject{currentmarker}{}%
\end{pgfscope}%
\begin{pgfscope}%
\pgfsys@transformshift{4.732807in}{3.734929in}%
\pgfsys@useobject{currentmarker}{}%
\end{pgfscope}%
\begin{pgfscope}%
\pgfsys@transformshift{4.741936in}{3.744178in}%
\pgfsys@useobject{currentmarker}{}%
\end{pgfscope}%
\begin{pgfscope}%
\pgfsys@transformshift{4.750784in}{3.753101in}%
\pgfsys@useobject{currentmarker}{}%
\end{pgfscope}%
\begin{pgfscope}%
\pgfsys@transformshift{4.759368in}{3.761720in}%
\pgfsys@useobject{currentmarker}{}%
\end{pgfscope}%
\begin{pgfscope}%
\pgfsys@transformshift{4.767703in}{3.770051in}%
\pgfsys@useobject{currentmarker}{}%
\end{pgfscope}%
\begin{pgfscope}%
\pgfsys@transformshift{4.775804in}{3.778114in}%
\pgfsys@useobject{currentmarker}{}%
\end{pgfscope}%
\begin{pgfscope}%
\pgfsys@transformshift{4.783682in}{3.785922in}%
\pgfsys@useobject{currentmarker}{}%
\end{pgfscope}%
\begin{pgfscope}%
\pgfsys@transformshift{4.791350in}{3.793489in}%
\pgfsys@useobject{currentmarker}{}%
\end{pgfscope}%
\begin{pgfscope}%
\pgfsys@transformshift{4.798819in}{3.800830in}%
\pgfsys@useobject{currentmarker}{}%
\end{pgfscope}%
\begin{pgfscope}%
\pgfsys@transformshift{4.806099in}{3.807955in}%
\pgfsys@useobject{currentmarker}{}%
\end{pgfscope}%
\begin{pgfscope}%
\pgfsys@transformshift{4.813199in}{3.814876in}%
\pgfsys@useobject{currentmarker}{}%
\end{pgfscope}%
\begin{pgfscope}%
\pgfsys@transformshift{4.820128in}{3.821603in}%
\pgfsys@useobject{currentmarker}{}%
\end{pgfscope}%
\begin{pgfscope}%
\pgfsys@transformshift{4.826894in}{3.828146in}%
\pgfsys@useobject{currentmarker}{}%
\end{pgfscope}%
\begin{pgfscope}%
\pgfsys@transformshift{4.833505in}{3.834513in}%
\pgfsys@useobject{currentmarker}{}%
\end{pgfscope}%
\begin{pgfscope}%
\pgfsys@transformshift{4.839967in}{3.840713in}%
\pgfsys@useobject{currentmarker}{}%
\end{pgfscope}%
\begin{pgfscope}%
\pgfsys@transformshift{4.846287in}{3.846754in}%
\pgfsys@useobject{currentmarker}{}%
\end{pgfscope}%
\begin{pgfscope}%
\pgfsys@transformshift{4.852470in}{3.852642in}%
\pgfsys@useobject{currentmarker}{}%
\end{pgfscope}%
\begin{pgfscope}%
\pgfsys@transformshift{4.858524in}{3.858385in}%
\pgfsys@useobject{currentmarker}{}%
\end{pgfscope}%
\begin{pgfscope}%
\pgfsys@transformshift{4.864453in}{3.863988in}%
\pgfsys@useobject{currentmarker}{}%
\end{pgfscope}%
\begin{pgfscope}%
\pgfsys@transformshift{4.870262in}{3.869457in}%
\pgfsys@useobject{currentmarker}{}%
\end{pgfscope}%
\begin{pgfscope}%
\pgfsys@transformshift{4.875956in}{3.874798in}%
\pgfsys@useobject{currentmarker}{}%
\end{pgfscope}%
\begin{pgfscope}%
\pgfsys@transformshift{4.881540in}{3.880017in}%
\pgfsys@useobject{currentmarker}{}%
\end{pgfscope}%
\begin{pgfscope}%
\pgfsys@transformshift{4.887017in}{3.885120in}%
\pgfsys@useobject{currentmarker}{}%
\end{pgfscope}%
\begin{pgfscope}%
\pgfsys@transformshift{4.892392in}{3.890107in}%
\pgfsys@useobject{currentmarker}{}%
\end{pgfscope}%
\begin{pgfscope}%
\pgfsys@transformshift{4.897668in}{3.894987in}%
\pgfsys@useobject{currentmarker}{}%
\end{pgfscope}%
\begin{pgfscope}%
\pgfsys@transformshift{4.902849in}{3.899761in}%
\pgfsys@useobject{currentmarker}{}%
\end{pgfscope}%
\begin{pgfscope}%
\pgfsys@transformshift{4.907938in}{3.904435in}%
\pgfsys@useobject{currentmarker}{}%
\end{pgfscope}%
\begin{pgfscope}%
\pgfsys@transformshift{4.912939in}{3.909013in}%
\pgfsys@useobject{currentmarker}{}%
\end{pgfscope}%
\begin{pgfscope}%
\pgfsys@transformshift{4.917854in}{3.913496in}%
\pgfsys@useobject{currentmarker}{}%
\end{pgfscope}%
\begin{pgfscope}%
\pgfsys@transformshift{4.922687in}{3.917889in}%
\pgfsys@useobject{currentmarker}{}%
\end{pgfscope}%
\begin{pgfscope}%
\pgfsys@transformshift{4.927440in}{3.922194in}%
\pgfsys@useobject{currentmarker}{}%
\end{pgfscope}%
\begin{pgfscope}%
\pgfsys@transformshift{4.932116in}{3.926416in}%
\pgfsys@useobject{currentmarker}{}%
\end{pgfscope}%
\begin{pgfscope}%
\pgfsys@transformshift{4.936716in}{3.930557in}%
\pgfsys@useobject{currentmarker}{}%
\end{pgfscope}%
\begin{pgfscope}%
\pgfsys@transformshift{4.941245in}{3.934619in}%
\pgfsys@useobject{currentmarker}{}%
\end{pgfscope}%
\begin{pgfscope}%
\pgfsys@transformshift{4.945703in}{3.938605in}%
\pgfsys@useobject{currentmarker}{}%
\end{pgfscope}%
\begin{pgfscope}%
\pgfsys@transformshift{4.950093in}{3.942517in}%
\pgfsys@useobject{currentmarker}{}%
\end{pgfscope}%
\begin{pgfscope}%
\pgfsys@transformshift{4.954417in}{3.946359in}%
\pgfsys@useobject{currentmarker}{}%
\end{pgfscope}%
\begin{pgfscope}%
\pgfsys@transformshift{4.958677in}{3.950131in}%
\pgfsys@useobject{currentmarker}{}%
\end{pgfscope}%
\begin{pgfscope}%
\pgfsys@transformshift{4.962874in}{3.953836in}%
\pgfsys@useobject{currentmarker}{}%
\end{pgfscope}%
\begin{pgfscope}%
\pgfsys@transformshift{4.967012in}{3.957477in}%
\pgfsys@useobject{currentmarker}{}%
\end{pgfscope}%
\begin{pgfscope}%
\pgfsys@transformshift{4.971091in}{3.961055in}%
\pgfsys@useobject{currentmarker}{}%
\end{pgfscope}%
\begin{pgfscope}%
\pgfsys@transformshift{4.975112in}{3.964572in}%
\pgfsys@useobject{currentmarker}{}%
\end{pgfscope}%
\begin{pgfscope}%
\pgfsys@transformshift{4.979078in}{3.968029in}%
\pgfsys@useobject{currentmarker}{}%
\end{pgfscope}%
\begin{pgfscope}%
\pgfsys@transformshift{4.982990in}{3.971430in}%
\pgfsys@useobject{currentmarker}{}%
\end{pgfscope}%
\begin{pgfscope}%
\pgfsys@transformshift{4.986850in}{3.974774in}%
\pgfsys@useobject{currentmarker}{}%
\end{pgfscope}%
\begin{pgfscope}%
\pgfsys@transformshift{4.990659in}{3.978064in}%
\pgfsys@useobject{currentmarker}{}%
\end{pgfscope}%
\begin{pgfscope}%
\pgfsys@transformshift{4.994417in}{3.981302in}%
\pgfsys@useobject{currentmarker}{}%
\end{pgfscope}%
\begin{pgfscope}%
\pgfsys@transformshift{4.998128in}{3.984488in}%
\pgfsys@useobject{currentmarker}{}%
\end{pgfscope}%
\begin{pgfscope}%
\pgfsys@transformshift{5.001791in}{3.987624in}%
\pgfsys@useobject{currentmarker}{}%
\end{pgfscope}%
\begin{pgfscope}%
\pgfsys@transformshift{5.005408in}{3.990712in}%
\pgfsys@useobject{currentmarker}{}%
\end{pgfscope}%
\begin{pgfscope}%
\pgfsys@transformshift{5.008980in}{3.993753in}%
\pgfsys@useobject{currentmarker}{}%
\end{pgfscope}%
\begin{pgfscope}%
\pgfsys@transformshift{5.012508in}{3.996747in}%
\pgfsys@useobject{currentmarker}{}%
\end{pgfscope}%
\begin{pgfscope}%
\pgfsys@transformshift{5.015993in}{3.999696in}%
\pgfsys@useobject{currentmarker}{}%
\end{pgfscope}%
\begin{pgfscope}%
\pgfsys@transformshift{5.019437in}{4.002603in}%
\pgfsys@useobject{currentmarker}{}%
\end{pgfscope}%
\begin{pgfscope}%
\pgfsys@transformshift{5.022840in}{4.005466in}%
\pgfsys@useobject{currentmarker}{}%
\end{pgfscope}%
\begin{pgfscope}%
\pgfsys@transformshift{5.031175in}{4.012444in}%
\pgfsys@useobject{currentmarker}{}%
\end{pgfscope}%
\begin{pgfscope}%
\pgfsys@transformshift{5.039275in}{4.019179in}%
\pgfsys@useobject{currentmarker}{}%
\end{pgfscope}%
\begin{pgfscope}%
\pgfsys@transformshift{5.047153in}{4.025685in}%
\pgfsys@useobject{currentmarker}{}%
\end{pgfscope}%
\begin{pgfscope}%
\pgfsys@transformshift{5.054822in}{4.031974in}%
\pgfsys@useobject{currentmarker}{}%
\end{pgfscope}%
\begin{pgfscope}%
\pgfsys@transformshift{5.062291in}{4.038058in}%
\pgfsys@useobject{currentmarker}{}%
\end{pgfscope}%
\begin{pgfscope}%
\pgfsys@transformshift{5.069571in}{4.043950in}%
\pgfsys@useobject{currentmarker}{}%
\end{pgfscope}%
\begin{pgfscope}%
\pgfsys@transformshift{5.083600in}{4.055193in}%
\pgfsys@useobject{currentmarker}{}%
\end{pgfscope}%
\begin{pgfscope}%
\pgfsys@transformshift{5.096976in}{4.065777in}%
\pgfsys@useobject{currentmarker}{}%
\end{pgfscope}%
\begin{pgfscope}%
\pgfsys@transformshift{5.109758in}{4.075765in}%
\pgfsys@useobject{currentmarker}{}%
\end{pgfscope}%
\begin{pgfscope}%
\pgfsys@transformshift{5.121996in}{4.085212in}%
\pgfsys@useobject{currentmarker}{}%
\end{pgfscope}%
\begin{pgfscope}%
\pgfsys@transformshift{5.133734in}{4.094165in}%
\pgfsys@useobject{currentmarker}{}%
\end{pgfscope}%
\begin{pgfscope}%
\pgfsys@transformshift{5.145011in}{4.102668in}%
\pgfsys@useobject{currentmarker}{}%
\end{pgfscope}%
\begin{pgfscope}%
\pgfsys@transformshift{5.155863in}{4.110757in}%
\pgfsys@useobject{currentmarker}{}%
\end{pgfscope}%
\begin{pgfscope}%
\pgfsys@transformshift{5.166320in}{4.118465in}%
\pgfsys@useobject{currentmarker}{}%
\end{pgfscope}%
\begin{pgfscope}%
\pgfsys@transformshift{5.176410in}{4.125821in}%
\pgfsys@useobject{currentmarker}{}%
\end{pgfscope}%
\begin{pgfscope}%
\pgfsys@transformshift{5.186159in}{4.132852in}%
\pgfsys@useobject{currentmarker}{}%
\end{pgfscope}%
\begin{pgfscope}%
\pgfsys@transformshift{5.204716in}{4.146028in}%
\pgfsys@useobject{currentmarker}{}%
\end{pgfscope}%
\begin{pgfscope}%
\pgfsys@transformshift{5.222148in}{4.158156in}%
\pgfsys@useobject{currentmarker}{}%
\end{pgfscope}%
\begin{pgfscope}%
\pgfsys@transformshift{5.238584in}{4.169366in}%
\pgfsys@useobject{currentmarker}{}%
\end{pgfscope}%
\begin{pgfscope}%
\pgfsys@transformshift{5.254130in}{4.179770in}%
\pgfsys@useobject{currentmarker}{}%
\end{pgfscope}%
\begin{pgfscope}%
\pgfsys@transformshift{5.268879in}{4.189459in}%
\pgfsys@useobject{currentmarker}{}%
\end{pgfscope}%
\begin{pgfscope}%
\pgfsys@transformshift{5.282908in}{4.198511in}%
\pgfsys@useobject{currentmarker}{}%
\end{pgfscope}%
\begin{pgfscope}%
\pgfsys@transformshift{5.296285in}{4.206993in}%
\pgfsys@useobject{currentmarker}{}%
\end{pgfscope}%
\begin{pgfscope}%
\pgfsys@transformshift{5.309066in}{4.214961in}%
\pgfsys@useobject{currentmarker}{}%
\end{pgfscope}%
\begin{pgfscope}%
\pgfsys@transformshift{5.321304in}{4.222465in}%
\pgfsys@useobject{currentmarker}{}%
\end{pgfscope}%
\begin{pgfscope}%
\pgfsys@transformshift{5.333042in}{4.229548in}%
\pgfsys@useobject{currentmarker}{}%
\end{pgfscope}%
\begin{pgfscope}%
\pgfsys@transformshift{5.347071in}{4.237866in}%
\pgfsys@useobject{currentmarker}{}%
\end{pgfscope}%
\begin{pgfscope}%
\pgfsys@transformshift{5.360448in}{4.245647in}%
\pgfsys@useobject{currentmarker}{}%
\end{pgfscope}%
\begin{pgfscope}%
\pgfsys@transformshift{5.385467in}{4.259810in}%
\pgfsys@useobject{currentmarker}{}%
\end{pgfscope}%
\begin{pgfscope}%
\pgfsys@transformshift{5.408483in}{4.272389in}%
\pgfsys@useobject{currentmarker}{}%
\end{pgfscope}%
\begin{pgfscope}%
\pgfsys@transformshift{5.429792in}{4.283655in}%
\pgfsys@useobject{currentmarker}{}%
\end{pgfscope}%
\begin{pgfscope}%
\pgfsys@transformshift{5.449630in}{4.293816in}%
\pgfsys@useobject{currentmarker}{}%
\end{pgfscope}%
\begin{pgfscope}%
\pgfsys@transformshift{5.468187in}{4.303036in}%
\pgfsys@useobject{currentmarker}{}%
\end{pgfscope}%
\begin{pgfscope}%
\pgfsys@transformshift{5.485619in}{4.311451in}%
\pgfsys@useobject{currentmarker}{}%
\end{pgfscope}%
\begin{pgfscope}%
\pgfsys@transformshift{5.502055in}{4.319165in}%
\pgfsys@useobject{currentmarker}{}%
\end{pgfscope}%
\begin{pgfscope}%
\pgfsys@transformshift{5.517601in}{4.326269in}%
\pgfsys@useobject{currentmarker}{}%
\end{pgfscope}%
\begin{pgfscope}%
\pgfsys@transformshift{5.532350in}{4.332837in}%
\pgfsys@useobject{currentmarker}{}%
\end{pgfscope}%
\begin{pgfscope}%
\pgfsys@transformshift{5.559756in}{4.344601in}%
\pgfsys@useobject{currentmarker}{}%
\end{pgfscope}%
\begin{pgfscope}%
\pgfsys@transformshift{5.584775in}{4.354850in}%
\pgfsys@useobject{currentmarker}{}%
\end{pgfscope}%
\begin{pgfscope}%
\pgfsys@transformshift{5.607791in}{4.363871in}%
\pgfsys@useobject{currentmarker}{}%
\end{pgfscope}%
\begin{pgfscope}%
\pgfsys@transformshift{5.629100in}{4.371882in}%
\pgfsys@useobject{currentmarker}{}%
\end{pgfscope}%
\begin{pgfscope}%
\pgfsys@transformshift{5.648938in}{4.379052in}%
\pgfsys@useobject{currentmarker}{}%
\end{pgfscope}%
\begin{pgfscope}%
\pgfsys@transformshift{5.667496in}{4.385511in}%
\pgfsys@useobject{currentmarker}{}%
\end{pgfscope}%
\begin{pgfscope}%
\pgfsys@transformshift{5.684928in}{4.391363in}%
\pgfsys@useobject{currentmarker}{}%
\end{pgfscope}%
\begin{pgfscope}%
\pgfsys@transformshift{5.701363in}{4.396694in}%
\pgfsys@useobject{currentmarker}{}%
\end{pgfscope}%
\begin{pgfscope}%
\pgfsys@transformshift{5.716910in}{4.401573in}%
\pgfsys@useobject{currentmarker}{}%
\end{pgfscope}%
\begin{pgfscope}%
\pgfsys@transformshift{5.731659in}{4.406056in}%
\pgfsys@useobject{currentmarker}{}%
\end{pgfscope}%
\begin{pgfscope}%
\pgfsys@transformshift{5.745688in}{4.410192in}%
\pgfsys@useobject{currentmarker}{}%
\end{pgfscope}%
\begin{pgfscope}%
\pgfsys@transformshift{5.759064in}{4.414021in}%
\pgfsys@useobject{currentmarker}{}%
\end{pgfscope}%
\begin{pgfscope}%
\pgfsys@transformshift{5.771846in}{4.417575in}%
\pgfsys@useobject{currentmarker}{}%
\end{pgfscope}%
\begin{pgfscope}%
\pgfsys@transformshift{5.784084in}{4.420885in}%
\pgfsys@useobject{currentmarker}{}%
\end{pgfscope}%
\begin{pgfscope}%
\pgfsys@transformshift{5.795822in}{4.423976in}%
\pgfsys@useobject{currentmarker}{}%
\end{pgfscope}%
\begin{pgfscope}%
\pgfsys@transformshift{5.807099in}{4.426868in}%
\pgfsys@useobject{currentmarker}{}%
\end{pgfscope}%
\begin{pgfscope}%
\pgfsys@transformshift{5.817951in}{4.429582in}%
\pgfsys@useobject{currentmarker}{}%
\end{pgfscope}%
\begin{pgfscope}%
\pgfsys@transformshift{5.828408in}{4.432134in}%
\pgfsys@useobject{currentmarker}{}%
\end{pgfscope}%
\begin{pgfscope}%
\pgfsys@transformshift{5.838499in}{4.434537in}%
\pgfsys@useobject{currentmarker}{}%
\end{pgfscope}%
\begin{pgfscope}%
\pgfsys@transformshift{5.848247in}{4.436804in}%
\pgfsys@useobject{currentmarker}{}%
\end{pgfscope}%
\begin{pgfscope}%
\pgfsys@transformshift{5.866804in}{4.440976in}%
\pgfsys@useobject{currentmarker}{}%
\end{pgfscope}%
\begin{pgfscope}%
\pgfsys@transformshift{5.884236in}{4.444728in}%
\pgfsys@useobject{currentmarker}{}%
\end{pgfscope}%
\begin{pgfscope}%
\pgfsys@transformshift{5.900672in}{4.448119in}%
\pgfsys@useobject{currentmarker}{}%
\end{pgfscope}%
\begin{pgfscope}%
\pgfsys@transformshift{5.916218in}{4.451200in}%
\pgfsys@useobject{currentmarker}{}%
\end{pgfscope}%
\begin{pgfscope}%
\pgfsys@transformshift{5.930967in}{4.454011in}%
\pgfsys@useobject{currentmarker}{}%
\end{pgfscope}%
\begin{pgfscope}%
\pgfsys@transformshift{5.944996in}{4.456588in}%
\pgfsys@useobject{currentmarker}{}%
\end{pgfscope}%
\begin{pgfscope}%
\pgfsys@transformshift{5.958373in}{4.458957in}%
\pgfsys@useobject{currentmarker}{}%
\end{pgfscope}%
\begin{pgfscope}%
\pgfsys@transformshift{5.971155in}{4.461143in}%
\pgfsys@useobject{currentmarker}{}%
\end{pgfscope}%
\begin{pgfscope}%
\pgfsys@transformshift{5.983392in}{4.463166in}%
\pgfsys@useobject{currentmarker}{}%
\end{pgfscope}%
\begin{pgfscope}%
\pgfsys@transformshift{5.995130in}{4.465044in}%
\pgfsys@useobject{currentmarker}{}%
\end{pgfscope}%
\end{pgfscope}%
\begin{pgfscope}%
\pgfpathrectangle{\pgfqpoint{0.642863in}{0.580957in}}{\pgfqpoint{5.607137in}{4.069043in}}%
\pgfusepath{clip}%
\pgfsetbuttcap%
\pgfsetroundjoin%
\definecolor{currentfill}{rgb}{0.121569,0.466667,0.705882}%
\pgfsetfillcolor{currentfill}%
\pgfsetlinewidth{1.003750pt}%
\definecolor{currentstroke}{rgb}{0.121569,0.466667,0.705882}%
\pgfsetstrokecolor{currentstroke}%
\pgfsetdash{}{0pt}%
\pgfsys@defobject{currentmarker}{\pgfqpoint{-0.035355in}{-0.058926in}}{\pgfqpoint{0.035355in}{0.058926in}}{%
\pgfpathmoveto{\pgfqpoint{0.000000in}{-0.058926in}}%
\pgfpathlineto{\pgfqpoint{0.035355in}{0.000000in}}%
\pgfpathlineto{\pgfqpoint{0.000000in}{0.058926in}}%
\pgfpathlineto{\pgfqpoint{-0.035355in}{0.000000in}}%
\pgfpathclose%
\pgfusepath{stroke,fill}%
}%
\begin{pgfscope}%
\pgfsys@transformshift{3.047317in}{1.638833in}%
\pgfsys@useobject{currentmarker}{}%
\end{pgfscope}%
\end{pgfscope}%
\begin{pgfscope}%
\pgfsetbuttcap%
\pgfsetroundjoin%
\definecolor{currentfill}{rgb}{0.000000,0.000000,0.000000}%
\pgfsetfillcolor{currentfill}%
\pgfsetlinewidth{0.803000pt}%
\definecolor{currentstroke}{rgb}{0.000000,0.000000,0.000000}%
\pgfsetstrokecolor{currentstroke}%
\pgfsetdash{}{0pt}%
\pgfsys@defobject{currentmarker}{\pgfqpoint{0.000000in}{-0.048611in}}{\pgfqpoint{0.000000in}{0.000000in}}{%
\pgfpathmoveto{\pgfqpoint{0.000000in}{0.000000in}}%
\pgfpathlineto{\pgfqpoint{0.000000in}{-0.048611in}}%
\pgfusepath{stroke,fill}%
}%
\begin{pgfscope}%
\pgfsys@transformshift{1.407244in}{0.580957in}%
\pgfsys@useobject{currentmarker}{}%
\end{pgfscope}%
\end{pgfscope}%
\begin{pgfscope}%
\definecolor{textcolor}{rgb}{0.000000,0.000000,0.000000}%
\pgfsetstrokecolor{textcolor}%
\pgfsetfillcolor{textcolor}%
\pgftext[x=1.407244in,y=0.483735in,,top]{\color{textcolor}\rmfamily\fontsize{10.000000}{12.000000}\selectfont \(\displaystyle {10^{-1}}\)}%
\end{pgfscope}%
\begin{pgfscope}%
\pgfsetbuttcap%
\pgfsetroundjoin%
\definecolor{currentfill}{rgb}{0.000000,0.000000,0.000000}%
\pgfsetfillcolor{currentfill}%
\pgfsetlinewidth{0.803000pt}%
\definecolor{currentstroke}{rgb}{0.000000,0.000000,0.000000}%
\pgfsetstrokecolor{currentstroke}%
\pgfsetdash{}{0pt}%
\pgfsys@defobject{currentmarker}{\pgfqpoint{0.000000in}{-0.048611in}}{\pgfqpoint{0.000000in}{0.000000in}}{%
\pgfpathmoveto{\pgfqpoint{0.000000in}{0.000000in}}%
\pgfpathlineto{\pgfqpoint{0.000000in}{-0.048611in}}%
\pgfusepath{stroke,fill}%
}%
\begin{pgfscope}%
\pgfsys@transformshift{2.731420in}{0.580957in}%
\pgfsys@useobject{currentmarker}{}%
\end{pgfscope}%
\end{pgfscope}%
\begin{pgfscope}%
\definecolor{textcolor}{rgb}{0.000000,0.000000,0.000000}%
\pgfsetstrokecolor{textcolor}%
\pgfsetfillcolor{textcolor}%
\pgftext[x=2.731420in,y=0.483735in,,top]{\color{textcolor}\rmfamily\fontsize{10.000000}{12.000000}\selectfont \(\displaystyle {10^{1}}\)}%
\end{pgfscope}%
\begin{pgfscope}%
\pgfsetbuttcap%
\pgfsetroundjoin%
\definecolor{currentfill}{rgb}{0.000000,0.000000,0.000000}%
\pgfsetfillcolor{currentfill}%
\pgfsetlinewidth{0.803000pt}%
\definecolor{currentstroke}{rgb}{0.000000,0.000000,0.000000}%
\pgfsetstrokecolor{currentstroke}%
\pgfsetdash{}{0pt}%
\pgfsys@defobject{currentmarker}{\pgfqpoint{0.000000in}{-0.048611in}}{\pgfqpoint{0.000000in}{0.000000in}}{%
\pgfpathmoveto{\pgfqpoint{0.000000in}{0.000000in}}%
\pgfpathlineto{\pgfqpoint{0.000000in}{-0.048611in}}%
\pgfusepath{stroke,fill}%
}%
\begin{pgfscope}%
\pgfsys@transformshift{4.055597in}{0.580957in}%
\pgfsys@useobject{currentmarker}{}%
\end{pgfscope}%
\end{pgfscope}%
\begin{pgfscope}%
\definecolor{textcolor}{rgb}{0.000000,0.000000,0.000000}%
\pgfsetstrokecolor{textcolor}%
\pgfsetfillcolor{textcolor}%
\pgftext[x=4.055597in,y=0.483735in,,top]{\color{textcolor}\rmfamily\fontsize{10.000000}{12.000000}\selectfont \(\displaystyle {10^{3}}\)}%
\end{pgfscope}%
\begin{pgfscope}%
\pgfsetbuttcap%
\pgfsetroundjoin%
\definecolor{currentfill}{rgb}{0.000000,0.000000,0.000000}%
\pgfsetfillcolor{currentfill}%
\pgfsetlinewidth{0.803000pt}%
\definecolor{currentstroke}{rgb}{0.000000,0.000000,0.000000}%
\pgfsetstrokecolor{currentstroke}%
\pgfsetdash{}{0pt}%
\pgfsys@defobject{currentmarker}{\pgfqpoint{0.000000in}{-0.048611in}}{\pgfqpoint{0.000000in}{0.000000in}}{%
\pgfpathmoveto{\pgfqpoint{0.000000in}{0.000000in}}%
\pgfpathlineto{\pgfqpoint{0.000000in}{-0.048611in}}%
\pgfusepath{stroke,fill}%
}%
\begin{pgfscope}%
\pgfsys@transformshift{5.379773in}{0.580957in}%
\pgfsys@useobject{currentmarker}{}%
\end{pgfscope}%
\end{pgfscope}%
\begin{pgfscope}%
\definecolor{textcolor}{rgb}{0.000000,0.000000,0.000000}%
\pgfsetstrokecolor{textcolor}%
\pgfsetfillcolor{textcolor}%
\pgftext[x=5.379773in,y=0.483735in,,top]{\color{textcolor}\rmfamily\fontsize{10.000000}{12.000000}\selectfont \(\displaystyle {10^{5}}\)}%
\end{pgfscope}%
\begin{pgfscope}%
\definecolor{textcolor}{rgb}{0.000000,0.000000,0.000000}%
\pgfsetstrokecolor{textcolor}%
\pgfsetfillcolor{textcolor}%
\pgftext[x=3.446432in,y=0.304723in,,top]{\color{textcolor}\rmfamily\fontsize{10.000000}{12.000000}\selectfont Energia \(\displaystyle ^{17}\)F (MeV)}%
\end{pgfscope}%
\begin{pgfscope}%
\pgfsetbuttcap%
\pgfsetroundjoin%
\definecolor{currentfill}{rgb}{0.000000,0.000000,0.000000}%
\pgfsetfillcolor{currentfill}%
\pgfsetlinewidth{0.803000pt}%
\definecolor{currentstroke}{rgb}{0.000000,0.000000,0.000000}%
\pgfsetstrokecolor{currentstroke}%
\pgfsetdash{}{0pt}%
\pgfsys@defobject{currentmarker}{\pgfqpoint{-0.048611in}{0.000000in}}{\pgfqpoint{-0.000000in}{0.000000in}}{%
\pgfpathmoveto{\pgfqpoint{-0.000000in}{0.000000in}}%
\pgfpathlineto{\pgfqpoint{-0.048611in}{0.000000in}}%
\pgfusepath{stroke,fill}%
}%
\begin{pgfscope}%
\pgfsys@transformshift{0.642863in}{1.196102in}%
\pgfsys@useobject{currentmarker}{}%
\end{pgfscope}%
\end{pgfscope}%
\begin{pgfscope}%
\definecolor{textcolor}{rgb}{0.000000,0.000000,0.000000}%
\pgfsetstrokecolor{textcolor}%
\pgfsetfillcolor{textcolor}%
\pgftext[x=0.344444in, y=1.147876in, left, base]{\color{textcolor}\rmfamily\fontsize{10.000000}{12.000000}\selectfont \(\displaystyle {10^{1}}\)}%
\end{pgfscope}%
\begin{pgfscope}%
\pgfsetbuttcap%
\pgfsetroundjoin%
\definecolor{currentfill}{rgb}{0.000000,0.000000,0.000000}%
\pgfsetfillcolor{currentfill}%
\pgfsetlinewidth{0.803000pt}%
\definecolor{currentstroke}{rgb}{0.000000,0.000000,0.000000}%
\pgfsetstrokecolor{currentstroke}%
\pgfsetdash{}{0pt}%
\pgfsys@defobject{currentmarker}{\pgfqpoint{-0.048611in}{0.000000in}}{\pgfqpoint{-0.000000in}{0.000000in}}{%
\pgfpathmoveto{\pgfqpoint{-0.000000in}{0.000000in}}%
\pgfpathlineto{\pgfqpoint{-0.048611in}{0.000000in}}%
\pgfusepath{stroke,fill}%
}%
\begin{pgfscope}%
\pgfsys@transformshift{0.642863in}{2.081564in}%
\pgfsys@useobject{currentmarker}{}%
\end{pgfscope}%
\end{pgfscope}%
\begin{pgfscope}%
\definecolor{textcolor}{rgb}{0.000000,0.000000,0.000000}%
\pgfsetstrokecolor{textcolor}%
\pgfsetfillcolor{textcolor}%
\pgftext[x=0.344444in, y=2.033339in, left, base]{\color{textcolor}\rmfamily\fontsize{10.000000}{12.000000}\selectfont \(\displaystyle {10^{3}}\)}%
\end{pgfscope}%
\begin{pgfscope}%
\pgfsetbuttcap%
\pgfsetroundjoin%
\definecolor{currentfill}{rgb}{0.000000,0.000000,0.000000}%
\pgfsetfillcolor{currentfill}%
\pgfsetlinewidth{0.803000pt}%
\definecolor{currentstroke}{rgb}{0.000000,0.000000,0.000000}%
\pgfsetstrokecolor{currentstroke}%
\pgfsetdash{}{0pt}%
\pgfsys@defobject{currentmarker}{\pgfqpoint{-0.048611in}{0.000000in}}{\pgfqpoint{-0.000000in}{0.000000in}}{%
\pgfpathmoveto{\pgfqpoint{-0.000000in}{0.000000in}}%
\pgfpathlineto{\pgfqpoint{-0.048611in}{0.000000in}}%
\pgfusepath{stroke,fill}%
}%
\begin{pgfscope}%
\pgfsys@transformshift{0.642863in}{2.967026in}%
\pgfsys@useobject{currentmarker}{}%
\end{pgfscope}%
\end{pgfscope}%
\begin{pgfscope}%
\definecolor{textcolor}{rgb}{0.000000,0.000000,0.000000}%
\pgfsetstrokecolor{textcolor}%
\pgfsetfillcolor{textcolor}%
\pgftext[x=0.344444in, y=2.918801in, left, base]{\color{textcolor}\rmfamily\fontsize{10.000000}{12.000000}\selectfont \(\displaystyle {10^{5}}\)}%
\end{pgfscope}%
\begin{pgfscope}%
\pgfsetbuttcap%
\pgfsetroundjoin%
\definecolor{currentfill}{rgb}{0.000000,0.000000,0.000000}%
\pgfsetfillcolor{currentfill}%
\pgfsetlinewidth{0.803000pt}%
\definecolor{currentstroke}{rgb}{0.000000,0.000000,0.000000}%
\pgfsetstrokecolor{currentstroke}%
\pgfsetdash{}{0pt}%
\pgfsys@defobject{currentmarker}{\pgfqpoint{-0.048611in}{0.000000in}}{\pgfqpoint{-0.000000in}{0.000000in}}{%
\pgfpathmoveto{\pgfqpoint{-0.000000in}{0.000000in}}%
\pgfpathlineto{\pgfqpoint{-0.048611in}{0.000000in}}%
\pgfusepath{stroke,fill}%
}%
\begin{pgfscope}%
\pgfsys@transformshift{0.642863in}{3.852489in}%
\pgfsys@useobject{currentmarker}{}%
\end{pgfscope}%
\end{pgfscope}%
\begin{pgfscope}%
\definecolor{textcolor}{rgb}{0.000000,0.000000,0.000000}%
\pgfsetstrokecolor{textcolor}%
\pgfsetfillcolor{textcolor}%
\pgftext[x=0.344444in, y=3.804263in, left, base]{\color{textcolor}\rmfamily\fontsize{10.000000}{12.000000}\selectfont \(\displaystyle {10^{7}}\)}%
\end{pgfscope}%
\begin{pgfscope}%
\definecolor{textcolor}{rgb}{0.000000,0.000000,0.000000}%
\pgfsetstrokecolor{textcolor}%
\pgfsetfillcolor{textcolor}%
\pgftext[x=0.288889in,y=2.615479in,,bottom,rotate=90.000000]{\color{textcolor}\rmfamily\fontsize{10.000000}{12.000000}\selectfont Range (\(\displaystyle mm\))}%
\end{pgfscope}%
\begin{pgfscope}%
\pgfsetrectcap%
\pgfsetmiterjoin%
\pgfsetlinewidth{0.803000pt}%
\definecolor{currentstroke}{rgb}{0.000000,0.000000,0.000000}%
\pgfsetstrokecolor{currentstroke}%
\pgfsetdash{}{0pt}%
\pgfpathmoveto{\pgfqpoint{0.642863in}{0.580957in}}%
\pgfpathlineto{\pgfqpoint{0.642863in}{4.650000in}}%
\pgfusepath{stroke}%
\end{pgfscope}%
\begin{pgfscope}%
\pgfsetrectcap%
\pgfsetmiterjoin%
\pgfsetlinewidth{0.803000pt}%
\definecolor{currentstroke}{rgb}{0.000000,0.000000,0.000000}%
\pgfsetstrokecolor{currentstroke}%
\pgfsetdash{}{0pt}%
\pgfpathmoveto{\pgfqpoint{6.250000in}{0.580957in}}%
\pgfpathlineto{\pgfqpoint{6.250000in}{4.650000in}}%
\pgfusepath{stroke}%
\end{pgfscope}%
\begin{pgfscope}%
\pgfsetrectcap%
\pgfsetmiterjoin%
\pgfsetlinewidth{0.803000pt}%
\definecolor{currentstroke}{rgb}{0.000000,0.000000,0.000000}%
\pgfsetstrokecolor{currentstroke}%
\pgfsetdash{}{0pt}%
\pgfpathmoveto{\pgfqpoint{0.642863in}{0.580957in}}%
\pgfpathlineto{\pgfqpoint{6.250000in}{0.580957in}}%
\pgfusepath{stroke}%
\end{pgfscope}%
\begin{pgfscope}%
\pgfsetrectcap%
\pgfsetmiterjoin%
\pgfsetlinewidth{0.803000pt}%
\definecolor{currentstroke}{rgb}{0.000000,0.000000,0.000000}%
\pgfsetstrokecolor{currentstroke}%
\pgfsetdash{}{0pt}%
\pgfpathmoveto{\pgfqpoint{0.642863in}{4.650000in}}%
\pgfpathlineto{\pgfqpoint{6.250000in}{4.650000in}}%
\pgfusepath{stroke}%
\end{pgfscope}%
\end{pgfpicture}%
\makeatother%
\endgroup%

%     \caption{Caption}
%     \label{fig:my_label}
% \end{figure}

\chapter{Conclusão}





% \par As duas condição que precisam ser satisfeitas para uma reta ser considerada um feixe são dadas por [XX].

% \begin{equation}\label{criterio_feixe_p}
%     \arccos{(\hat{v}_i \cdot (0, 0, 1))} < \beta, 
% \end{equation}

% onde $\hat{v}_i$ é o versor de uma reta $i$ e $\beta$ é um limiar dado em graus a ser determinado empiricamente, que neste caso foi considerado como 5°.

\printbibliography[title={Referências}]
\end{document}
